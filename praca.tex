\documentclass[12pt,notitlepage]{aghdpl}
% \documentclass[en,11pt]{aghdpl}  % praca w języku angielskim

% Lista wszystkich języków stanowiących języki pozycji bibliograficznych użytych w pracy.
% (Zgodnie z zasadami tworzenia bibliografii każda pozycja powinna zostać utworzona zgodnie z zasadami języka, w którym dana publikacja została napisana.)
\usepackage[english,polish]{babel}

% Użyj polskiego łamania wyrazów (zamiast domyślnego angielskiego).
\usepackage{polski}

\usepackage[utf8]{inputenc}

% dodatkowe pakiety
\usepackage{mathtools}
\usepackage{amsfonts}
\usepackage{amsmath}
\usepackage{hyperref}
\usepackage{amsthm}


% --- < bibliografia > ---

\usepackage[
style=numeric,
sorting=none,
%
% Zastosuj styl wpisu bibliograficznego właściwy językowi publikacji.
language=autobib,
autolang=other,
% Zapisuj datę dostępu do strony WWW w formacie RRRR-MM-DD.
urldate=iso8601,
% Nie dodawaj numerów stron, na których występuje cytowanie.
backref=false,
% Podawaj ISBN.
isbn=true,
% Nie podawaj URL-i, o ile nie jest to konieczne.
url=true,
%
% Ustawienia związane z polskimi normami dla bibliografii.
maxbibnames=3,
% Jeżeli używamy BibTeXa:
backend=bibtex
]{biblatex}

\usepackage{csquotes}
% Ponieważ `csquotes` nie posiada polskiego stylu, można skorzystać z mocno zbliżonego stylu chorwackiego.
\DeclareQuoteAlias{croatian}{polish}

\addbibresource{praca_pliki/bibliografia.bib}
\renewcommand{\labelitemii}{$\bullet$}
\usepackage{tabto}

% Nie wyświetlaj wybranych pól.
%\AtEveryBibitem{\clearfield{note}}


% ------------------------
% --- < listingi > ---

% Użyj czcionki kroju Courier.
\usepackage{courier}

\usepackage{listings}
\lstloadlanguages{TeX}

\lstset{
	literate={ą}{{\k{a}}}1
           {ć}{{\'c}}1
           {ę}{{\k{e}}}1
           {ó}{{\'o}}1
           {ń}{{\'n}}1
           {ł}{{\l{}}}1
           {ś}{{\'s}}1
           {ź}{{\'z}}1
           {ż}{{\.z}}1
           {Ą}{{\k{A}}}1
           {Ć}{{\'C}}1
           {Ę}{{\k{E}}}1
           {Ó}{{\'O}}1
           {Ń}{{\'N}}1
           {Ł}{{\L{}}}1
           {Ś}{{\'S}}1
           {Ź}{{\'Z}}1
           {Ż}{{\.Z}}1,
	basicstyle=\footnotesize\ttfamily,
}

% ------------------------

\AtBeginDocument{
	\renewcommand{\tablename}{Tabela}
	\renewcommand{\figurename}{Rys.}
}

% ------------------------
% --- < tabele > ---
\usepackage{relsize}
\usepackage{array}
\usepackage{tabularx}
\usepackage{multirow}
\usepackage{booktabs}
\usepackage{makecell}
\usepackage[flushleft]{threeparttable}
\usepackage{graphicx}
\usepackage[table,x11names]{xcolor}
\usepackage[toc,page]{appendix}	
% defines the X column to use m (\parbox[c]) instead of p (`parbox[t]`)
\newcolumntype{C}[1]{>{\hsize=#1\hsize\centering\arraybackslash}X}

%---------------------------------------------------------------------------

\author{Miłosz Mach}
\shortauthor{M. Mach}

%\titlePL{Przygotowanie bardzo długiej i pasjonującej pracy dyplomowej w~systemie~\LaTeX}
%\titleEN{Preparation of a very long and fascinating bachelor or master thesis in \LaTeX}

\titlePL{Wbudowany system wizyjny do śledzenia obiektów dla potrzeb nawigacji bezzałogowego statku powietrznego (UAV)}
\titleEN{Embedded vision-based tracking system for unmanned aerial vehicle (UAV)}


\shorttitlePL{Wbudowany system wizyjny do śledzenia obiektów dla potrzeb nawigacji bezzałogowego statku powietrznego (UAV)} % skrócona wersja tytułu jeśli jest bardzo długi
\shorttitleEN{Embedded vision-based tracking system for unmanned aerial vehicle (UAV)}

\thesistype{Praca dyplomowa magisterska}
%\thesistype{Master of Science Thesis}

\supervisor{dr inż. Tomasz Kryjak}
%\supervisor{Tomasz Kryjak PhD}

\degreeprogramme{Automatyka i Robotyka}
%\degreeprogramme{Automatics and Robotics}

\date{2017}

\department{Katedra Automatyki i Inżynierii Biomedycznej}
%\department{Department of Applied Computer Science}

\faculty{Wydział Elektrotechniki, Automatyki,\protect\\[-1mm] Informatyki i Inżynierii Biomedycznej}
%\faculty{Faculty of Electrical Engineering, Automatics, Computer Science and Biomedical Engineering}

\acknowledgements{Serdecznie dziękuję mojemu promotorowi Panu dr inż. Tomaszowi Kryjakowi, za wsparcie merytoryczne oraz bliskim za cierpliwość w trakcie pisania tej pracy.}


\setlength{\cftsecnumwidth}{10mm}

%---------------------------------------------------------------------------
\setcounter{secnumdepth}{4}
\brokenpenalty=10000\relax

\begin{document}
\graphicspath{{praca_pliki/}}
\nocite{*}

\titlepages

% Ponowne zdefiniowanie stylu `plain`, aby usunąć numer strony z pierwszej strony spisu treści i poszczególnych rozdziałów.
\fancypagestyle{plain}
{
	% Usuń nagłówek i stopkę
	\fancyhf{}
	% Usuń linie.
	\renewcommand{\headrulewidth}{0pt}
	\renewcommand{\footrulewidth}{0pt}
}

\setcounter{tocdepth}{2}
\tableofcontents
\clearpage

\begin{abstract}
W tej pracy przedstawiona została sprzętowa realizacja detekcji i śledzenia osoby w wielu skalach, przy wykorzystaniu deskryptora HoG i klasyfikatora SVM, z niezależnym wsparciem algorytmu MeanShift. Zastosowanie potokowego przetwarzania obrazu pozwoliło na realizację obliczeń w czasie rzeczywistym dla sygnału wideo o rozdzielczości $720 \times 1280$ i $60$ klatkach na sekundę dla algorytmu MeanShift i $30$ klatkach na sekundę dla metody HoG+SVM. Przedstawiany projekt został napisany w języku opisu sprzętu System Verilog i C++, oraz uruchomiony w układzie ZYNQ SoC znajdującym się na karcie uruchomieniowej PYNQ.

Moduł śledzenia postaci został zrealizowany na potrzeby wbudowanego systemu wizyjnego do sterowania bezzałogowym statkiem latającym realizującym zadanie nadążania za osobą. Projekt systemu również został opisany w tej pracy.
\end{abstract}

\selectlanguage{english} 
\begin{abstract}
This document describes hardware realization of multi-scale human detection and tracking, which was based on HoG descriptor and SVM classifier, with an independent support of a MeanShift algorithm. Pipelined image processing allowed to compute $720 \times 1280$, $60$fps video signal in real time for MeanShift, and $30$fps for HoG+SVM method. Project was created in hardware description language - System Verilog - and C++, and ran on ZYNQ SoC which is the part of a PYNQ board.

Human tracking model was created for an embedded vision system to control an unmanned aerial vehicle with an intention of following the detected person. That system has been described in this document as well.
\end{abstract}
\selectlanguage{polish} 
\chapter{Wstęp}
\label{cha:wstep}

%TODO Przed tym proszę umieścić abstrakt PL i EN na jednej stronie (po kilka zdań).
%TODO Jakoś nie uważam, aby takie spisy tablic, rysunków itp. były potrzebne w takiej pracy - usunąć. %ODP OK
Bezzałogowe statki powietrzne, potocznie nazywane dronami, zyskują w ostatnich latach coraz większe zainteresowanie. 
Według raportu portalu Business Insider z czerwca 2016 roku \cite{BInsider}, światowy rynek bezzałogowych statków powietrznych powiększy swoją wartość z niecałych 6 mld \$ w 2013 roku do prawie 13 mld \$ w roku 2024. 
Biorąc pod uwagę podstawowy podział maszyn -- ze względu na przeznaczenie -- raport ten wskazuje na rosnący trend zarówno w przypadku przemysłu obronnego, jak i cywilnego. W obrębie drugiej grupy można uwzględnić zastosowania w następujących branżach:
\begin{itemize}
	\item transport -- dostarczanie przesyłek
	\item przemysł filmowy -- realizacja stabilnych ujęć lotniczych
	\item rolnictwo -- stosowanie oprysków, nadzór nad wypasem zwierząt
	\item górnictwo -- skanowanie otoczenia, ocena infrastruktury kopalni, monitorowanie szybów, transport części do morskich platform wydobywczych
	\item ekologia -- pomiar zanieczyszczeń na danym obszarze
	\item ratownictwo medyczne -- transport krwi lub urządzeń ratujących życie, poszukiwanie osób zaginionych
\end{itemize}
Tak gwałtowny wzrost popularności niesie ze sobą potrzebę tworzenia nowych rozwiązań, którym sprzyja ogólny rozwój technologiczny i dostępność wielu elementów elektronicznych -- czujników, modułów komunikacyjnych, autopilotów; stopniowej poprawie ulega też kwestia zasilania bateryjnego i związany z nią maksymalny czas pracy. %TODO tj. 

Jedno z głównych zadań, jakie stoi przed inżynierami, to autonomizacja dronów, czyli nadanie im możliwości pracy bez nadzoru człowieka.
Oznacza to wyposażenie ich w odpowiednie urządzenia elektroniczne i rozwiązania służące do realizacji określonego sterowania w oparciu o zebrane informacje. Przykładem takiego zadania jest śledzenie obiektu na podstawie informacji wizyjnej. Po inicjalizacji, którą jest wskazanie celu (lub jego detekcja) następuje określenie jego położenia w kolejnych klatkach obrazu. Na tej podstawie generowane jest sterowanie ruchem drona, które w efekcie dąży do utrzymania obiektu w określonej pozycji na obrazie.

Biorąc pod uwagę główne systemy elektroniczne drona, większość platform bazuje na podejściu sekwencyjnym, realizowanym na mikrokontrolerach lub procesorach sygnałowych. Dodanie nowych funkcjonalności - zwłaszcza związanych z autonomizacją - często wykracza poza możliwości obliczeniowe wspomnianych układów. Ponadto ograniczenia związane maksymalną ładownością drona i pojemnością baterii wykluczają wykorzystanie większych i często energochłonnych jednostek.  Z tego względu naukowcy i firmy związane z rynkiem maszyn bezzałogowych coraz chętniej rozważają stosowanie układów FPGA.

%TODO mówiąc uczciwie - trzeba to ciut rozbudować. Omówić różne zastosowania dronów, poszerzyć temat autonomiczności, pokazać na czym polega śledznie i poruszyć temat platform obliczeniowych.

\section{Cel pracy}

%TODO rozbudować zgodnie ze specyfikację (załączona do maila, tekst latex ponieżej
%TODO - tylko przerobić i na czas przeszły
%ODP poprawione
Celem pracy była realizacja systemu nawigacji bezzałogowego, 
autonomicznego statku powietrznego opartego o informację wizyjną. 
Założeniem zadania było śledzenie zdefinowanego podczas inicjalizacji obiektu. 
Pierwszym etapem prac był przegląd artykułów związanych z algorytmami śledzenia 
zaimplementowanymi we wbudowanych systemach rekonfigurowalnych 
(FPGA/ZYNQ), przy czym preferowane były rozwiązania zbliżone do tematu pracy. 

Na tej podstawie wybrano dwa podejścia i zaimplementowano je programowo, 
analizując ich pracę na materiale wideo zbliżonym do docelowego. 
Kolejnym etapem była sprzętowa implementacja algorytmów w układzie, 
której proces był zgodny z typową metodologią obejmującą: stworzenie modelu programowego, 
rozwój wymaganych modułów, testy symulacyjne oraz testy w sprzęcie.
Ostatnie zadanie stanowiła integracja układu z platformą UAV, 
zrealizowanie komunikacji pomiędzy modułem wizyjnym, 
układem reprogramowalnym i autopilotem. 
Docelowo miał powstać prototyp bezzałogowego statku powietrznego, zdolnego podążać za wybraną osobą.





\section{Struktura pracy}

Rozdział drugi poświęcono opisowi dostępnych algorytmów, a następnie skoncentrowano się na dokładnym przedstawieniu zagadnień związanych z HoG+SVM oraz MeanShift. %TODO ale jakich algorytmów ? wszystkich ? %ODP poprawiono
W rozdziale trzecim pokrótce przedstawiono zrealizowane modele programowe, które następnie przeniesiono na system wbudowany -- co zostało opisane w rozdziale 4. 
Część 5 zawiera informacje o konfiguracji sprzętowej oraz o wyższej warstwie decyzyjnej. 
Część 6 prezentuje proces testów poświęconych tworzonemu rozwiązaniu. 
Ostatni rozdział stanowi przedstawienie wniosków i wskazuje możliwe kierunki rozwoju pracy.













\chapter{Koncepcja systemu śledzenia obiektów}
\label{sec:koncepcja}
Realizacja projektu śledzenia obiektów dla potrzeb nawigacji bezzałogowego statku powietrznego dotyczy detekcji osoby w postawie stojącej. Wymagane było tu zdefiniowanie kilku założeń:
\begin{itemize}
	\item detekcja osoby znajdującej się jedynie w bezpośrednim otoczeniu drona, wewnątrz okręgu o promieniu do 7 metrów
	\item śledzenie poprzez ruch całej platformy -- kamera jest nieruchoma, a jej pozycję dodatkowo stabilizuje gimbal kompensując wychylenia drona,
	\item śledzenie w przestrzeni trójwymiarowej, z zadaną pozycją:
	\begin{itemize}
		\item przesunięcie względem osoby: $0$m, tj. osoba powinna znajdować się w centrum obrazu rejestrowanego przez kamerę
		\item wysokość: około $1.5$m od ziemi
		\item odległość: około $4$m od osoby
	\end{itemize} 
	\item automatyzacja misji - do zadań użytkownika należy jedynie wydanie rozkazu startu z ziemi i zakończenia pracy, skutkującego lądowaniem
	\item możliwość awaryjnego przejęcia manualnej kontroli nad dronem
	\item brak funkcjonalności, która zachowałoby ciągłość detekcji w przypadku pojawienia się drugiej osoby w otoczeniu głównego celu
\end{itemize}

%TODO Brakuje tu takiego rozdziału przedstawiającego koncepcję co Pan chce zrobić. Częściowo będzie to we wstępnie, ale raczej pobieżnie. Wydaje mi się, że dobrze by było w tym rozdziale opisać koncepcję systemu (rysunek), drona i potem przejść do omówienia poszczególnych komponentów. %ODP OK
Dodatkowym warunkiem było oparcie prac na konstrukcji opisanej w kolejnych podrozdziałach.

\section{Platforma UAV}
Platforma, na której realizowany jest projekt, składa się z następujących elementów:
\begin{itemize}
	\item typ obiektu: hexacopter (rama DJI F550),
	\item rodzaj śmigieł: wzmocnione śmigła o oznaczeniu 9050, czyli o średnicy śmigła równej $9.0"$ ($~22.86$cm) oraz skoku śmigła $5.0"$ ($~12.7$cm).,
	\item silniki: DJI 2312/960KV sterowane kontrolerami 420 LITE,
	\item zasilanie: czterokomorowa bateria LiPo o nominalnym napięciu $14.8$V (maksymalnym $16.8$V) oraz o pojemności $6450$mAh,
	\item kamera: Xiaomi Yi,
	\item gimbal: Tarot T-2D,
	\item aparatura radiowa: FrSky Taranis X9D Plus,
	\item odbiornik: FrSky X8D,
	\item autopilot: 3DR Pixhawk.
\end{itemize}

Wybór komponentów, montaż platformy i jej kalibracja były częścią tej pracy i w ramach projektu SKN AVADER były dofinansowane ze środków Grantu Rektorskiego AGH 2015 oraz funduszy wydziału EAIiIB.
%TODO do rozdziału 1 %ODP OK
%TODO napisać wprost, że wybór komponentów, montaż itp. były a) elementem tej pracy, b) też że projektem rektorskim SKN AVADER. %ODP

\section{Autopilot}

Każdy dron byłby bezużyteczną konstrukcją, gdyby nie serce maszyny -- tzw. autopilot. 
W tym przypadku postanowiono wykorzystać urządzenie Pixhawk. 
Jest to zgodny ze standardami przemysłowymi moduł na otwartej licencji, stworzony przy współpracy z firmą 3D Robotics oraz ArduPilot Group. 
Posiada następujące parametry:
\begin{itemize}
	\item procesor Cortex-M4F taktowany zegarem 168 MHz,
	\item sensory: trzyosiowy akcelerometr, żyroskop, kompas magnetyczny, barometr i zewnętrzny GPS,
	\item slot na kartę microSD,
	\item możliwość połączenia peryferiów (interfejsy: UART, I2C, CAN),
	\item 14 wyjść PWM (8 głównych z zabezpieczeniami + 6 dodatkowych).
\end{itemize}

\begin{figure}[h]
	\centering
	\includegraphics[width=8cm]{5_pixhawk.jpg}
	\caption{Autopilot Pixhawk -- widok na panel główny oraz we/wy PWM}
	\label{fig:pixhawk}
\end{figure}

Powyższy sprzęt jednak nie jest w pełni skonfigurowany do pracy po wyjęciu z pudełka -- szczególnie, że jako produkt uniwersalny, bywa montowany na konstrukcjach o szerokim rozrzucie parametrów. 
Może zapewnić sterowanie śmigłowcom (tzw. multicopterom), samolotom modelarskim oraz nawet łazikom. %TODO mam wątpliwość, czy słowo "kopter" jest poprawne. Skoro helikopter jest niepoprawny (był taki słynny przetarg..) to trzeba śmigłowiec. %ODP OK, do nawiasu wrzucono potoczną nazwę.
W przypadku dwóch pierwszych grup konfiguracja wiąże się ze zdefiniowaniem odpowiedniej liczby śmigieł i ich rozstawienia, typu aparatury radiowej oraz zewnętrznych urządzeń geolokalizacyjnych. 
Tę dość dużą elastyczność mogą zapewnić dwa główne systemy, które są wczytywane z pamięci SD i pracują w czasie rzeczywistym. 
Dedykowany, PX4 Flight Stack jest stworzony przez twórców modułu, oraz ArduPilot Copter (ArduCopter) -- niezależny, otwarty system, który został dostosowany do platformy Pixhawk z wykorzystaniem dostępnych narzędzi deweloperskich. 
Ze względu na większą bazę użytkowników i dojrzałość projektu, wybrano drugie rozwiązanie.

\section{Integracja urządzeń na platformie UAV} %TODO średni tytuł %ODP OK

Zbudowanie drona w oparciu o gotowe komponenty nie jest zadaniem skomplikowanym. 
Jednak z uwagi na wymagania projektu (opisane w \ref{sec:koncepcja}), należało dokładnie przemyśleć jego przebudowę. Rysunek \ref{fig:drone_photo} przedstawia zdjęcie zmodyfikowanej, gotowej do lotu platformy. %TODO przerobić zdanie, że z uwagi na wymagania projektu - które powinny być opisane wcześniej. %ODP OK
Z kolei schemat \ref{fig:architecture} opisuje połączenia pomiędzy urządzeniami na platformie UAV.  %TODO arch. sygnałową - dziwne określenie %ODP OK 
\begin{figure}[h]
	\centering
	\captionsetup{justification=centering,margin=1cm}
	\hspace*{0cm}
	\includegraphics[width=14cm]{5_drone_photo.jpg}
	\caption{Dron po niezbędnych modyfikacjach. Pod gąbką ochronną znajduje się układ PYNQ}
	\label{fig:drone_photo}
\end{figure}
\begin{figure}[]
	\centering
	\includegraphics[width=15cm]{5_drone_architecture.png}
	\caption{Relacje pomiędzy urządzeniami na platformie UAV}
	\label{fig:architecture}
\end{figure}
Linie czerwone określają dystrybucję zasilania z zamontowanej baterii (pominięto konwertery napięcia), natomiast linie czarne opisują propagację sygnałów pomiędzy urządzeniami. Dodatkowo, liniami przerywanymi zaznaczono sygnały opcjonalne wykorzystywane w trakcie naziemnej analizy systemu. Porty szeregowe komputera służą do komunikacji z konsolą zaimplementowaną na układzie Zynq, oraz do konfiguracji autopilota poprzez aplikację MISSION Planner.


%TODO brakuje omówienia rysunku. Trzeba go po prostu opisać w tekście. %ODP OK

%TODO Brak referencji w txt. "skrywa" się potoczne. %ODP OK

%TODO co to znaczy komputer w trakcie implementacji ? %ODP zamieniono linię na przerywaną (komputer jest opcjonalnie podłączany na ziemi w celu komunikacji z układami)


%Generalnie nie mam pomysłu na to, gdzie wrzucić powyższe informacje związane z HW.
\chapter{Zastosowanie układów rekonfigurowalnych w bezzałogowych statkach powietrznych}

Drony są urządzeniami, które zazwyczaj pracują w pewnej odległości od osoby nadzorującej, a często nawet poza jej zasięgiem widzenia. 
O ile może rodzić to pewne trudności, jest to chociażby jeden z głównych powodów udziału dronów w zastosowaniach militarnych -- wówczas operator drona wojskowego nie tylko znajduje się poza obszarem zagrożenia, ale często zdalnie podejmowane przez niego decyzje wiążą się z niższym poziomem stresu. 
Taka platforma powietrzna musi być jednak wyposażona w elementy elektroniczne pozwalające przesłać użytkownikowi zestaw danych o statusie drona. 
Podstawową grupę pełnią informacje telemetryczne z czujników (akcelerometr, żyroskop, GPS, stan baterii lub zapas paliwa w zbiorniku). 
Ponadto, już nawet komercyjne drony ze średniej półki cenowej umożliwiają przesłanie informacji wizyjnej, a niektóre nawet ją przetwarzają. 
Przykładem może być DJI Spark, miniaturowa konstrukcja, która na podstawie informacji z wbudowanej kamery może wykonywać komendy wydawane gestami \cite{SPARK}.

Poprawne działanie platformy latającej wymaga przetworzenia sporej ilości informacji, gwałtownie rosnącej wraz z dodawaniem nowych funkcjonalności. 
Na etapie projektowania ważne jest zatem dobranie odpowiedniej jednostki obliczeniowej. 
Schemat \ref{fig:autopilot_architecture} opisuje podstawowe zależności w systemie, które należy uwzględnić definiując architekturę.
\begin{figure}[h]
	\centering
	\includegraphics[width=12cm]{7_drone_platform_overview.jpg}
	\caption{Architektura programowo-sprzętowa związana z pracą autopilota \cite{Bouhali2017} }
	\label{fig:autopilot_architecture}
\end{figure} 

Rynek urządzeń typu UAV obfituje w rozwiązania oparte mikrokontrolery. 
Popularnymi autopilotami stosowanymi w komercyjnych produktach -- i przede wszystkim w ręcznie tworzonych konstrukcjach -- są: NAZA-M V2 firmy DJI oraz Pixhawk firmy 3DR. 
Oba te moduły, bazując na mikrokontrolerach ARM i dużej liczbie peryferiów, w zupełności wystarczają do zastosowań niewymagających przetwarzania dużej ilości danych, zapewniając łączność z aparaturą radiową i realizację misji opartych na predefiniowanej sekwencji ruchów. 

Nieco bardziej rozbudowane są rozwiązania wykorzystujące dwa mikrokontrolery -- jeden z~nich jest skonfigurowany w tzw. trybie „baremetal”, który oznacza uruchomienie aplikacji bezpośrednio na procesorze. 
Umożliwia to wykonywanie niskopoziomowych, krytycznych zadań, jak na przykład stabilizacja lotu uwzględniająca sterowanie pracą silników i pomiar wartości z czujników. 
Na drugim mikrokontrolerze uruchomiony jest system operacyjny, który stanowi platformę dla aplikacji wysokopoziomowych -- takich jak algorytmy planowania trasy, stereowizja czy śledzenie celu. 
Jedną z takich komercyjnych platform jest „MikroKopter” stworzony przez firmę HiSystems GmbH \cite{MikroKopter}.

Układy rekonfigurowalne, w miarę postępu technologicznego, stają się godną uwagi alternatywą -- pomimo często wyższej ceny oferują konkurencyjną wartość poboru mocy i niezrównanie szybszą prędkość działania algorytmów. 
Szczególnie, jeśli podczas ich implementacji uwzględni się możliwość zrównoleglania obliczeń. 

\section{Układy rekonfigurowalne w roli autopilota platformy UAV}

Przewaga układów rekonfigurowalnych okazuje się być widoczna już w przypadku zadania stabilizacji, a przykładem może być projekt \cite{Eizad}, w którym częstotliwość pracy regulatora PI dla osi obrotu w układzie FPGA osiągnęła wartość $4.3$MHz, w porównaniu z $0.71$MHz dla rozwiązania programowego na mikrokontrolerze ARM7. 
Prawdziwym przełomem okazały się być układy SoC, które zintegrowały część procesorową i konfigurowalną, i w efekcie dały sporą swobodę w sposobie realizacji projektu autopilota.
Zespół badawczy odpowiedzialny za opisany wyżej projekt wykorzystał układ Zynq w projekcie kolejnego drona, gdzie w części konfigurowalnej (PL -- ang. programmable logic) zaimplementowano regulację PID wymaganej do stabilizacji urządzenia, a niezależnie podejście programowo-sprzętowe pozwoliło zrealizować implementację algorytmu planowania ruchu \cite{Eizad2}. %TODO 2 objaśnić PL %ODP OK
Z kolei dla konstrukcji opisanej w publikacji \cite{Schlender}, najważniejsze zadania rozdzielono pomiędzy 3 procesory w układzie Zynq: 2 z nich, softprocesory Microblaze, odpowiadały za stabilizację, a wyższa warstwa zadań związanych ze zdefiniowaną misją realizowana była w części procesorowej (PS -- ang. processing system) opartej o architekturę ARM. %TODO 2 objaśnić PS %ODP OK

Istotnym aspektem pracy autopilota powinna być możliwość zapewnienia odpowiedniego interfejsu komunikacji z szeregiem wykorzystywanych czujników. 
Układy rekonfigurowalne nie tylko udostępniają ogromną liczbę wejść i wyjść, ale są też często wyposażone w sprzętowe kontrolery dla popularnych interfejsów. 
Ponadto, w przypadku ich niewystarczającej liczby istnieje możliwość sprzętowej implementacji własnego kontrolera. 
Przede wszystkim jednak, układ rekonfigurowalny pozwala przetworzyć otrzymane wartości w sposób bardziej złożony, niż pozwalałaby na to moc obliczeniowa mikrokontrolera. 
W publikacji \cite{MEMS} opisano implementację sprzętową cyfrowego kontrolera dla żyroskopów MEMS, który poprzez kwadraturową demodulację sygnału wejściowego i wykorzystanie równolegle pętli synchronizacji fazy (PLL) i automatycznej regulacji wzmocnienia (AGC) poprawia dokładność działania czujnika.

Kolejnym, bardzo ważnym zadaniem autopilota jest estymacja stanu, polegająca na kompensowaniu wszelkich zakłóceń pochodzących z pomiarów. 
Jest ona najczęściej realizowana w formie filtru Kalmana. 
Jedna z publikacji \cite{SohKalman} opisuje sprzętowo-programową implementację bezśladowego filtru Kalmana (UKF -- \textit{Unscented Kalman Filter}), której osiągi i zużycie zasobów są konfigurowalne poprzez zdefiniowanie liczby tzw. Bloków Przetwarzania (w liczbach: 1,2,5,10). 
Algorytm uruchomiony na urządzeniu Zynq XC7Z045 osiągał ponad dwukrotnie większą prędkość działania w porównaniu z rozwiązaniami programowymi i zużywał mniej energii (131mW dla konfiguracji z pojedynczym Blokiem). 

Ostatnim aspektem pracy autopilota jest generacja sygnałów sterujących silnikami. 
W~dronach najchętniej montowane są bezszczotkowe silniki prądu stałego, których poprawne działanie wymaga kontrolera sterującego odpowiednim przepływem prądu w uzwojeniach.
Zastosowanie układu FPGA pozwala nie tylko generować sygnały PWM wysyłane do kontrolerów prędkości (ESC), ale realizować ich funkcję z pomocą niezależnego obwodu dostarczającego zasilanie. 
Drugą formę rozwiązania zaprezentowano w pracy \cite{ESC}. 
Dzięki niemu osiągnięto wyższą częstotliwości pracy ($12.5$MHz) niż dla tradycyjnego urządzenia ESC ($50$Hz), w efekcie tworząc bardziej responsywną maszynę.

Powyższe rozważania przedstawiają potencjał, jaki osiągnąć mogą autopiloty bazujące na układach rekonfigurowalnych -- co więcej, jednostki tego typu są już dostępne w sprzedaży. %TODO 2 - to "jednak" dziwnie tutaj brzmi
Jedna z nich steruje quadrotorem „Phenox” \cite{Konomura}, w którym część konfigurowalna układu z rodziny Zynq jest odpowiedzialna za generację sygnałów PWM sterujących silnikami, odbiór informacji z czujników oraz przetwarzanie obrazu i dźwięku.
Kolejny, ważny przełom został osiągnięty przez firmę Aerotenna, która w 2016 roku rozpoczęła produkcję autopilotów kompatybilnych z niezwykle popularnym oprogramowaniem Ardupilot. 
Pierwszym z urządzeń był „OcPoc”, z układem Zynq-7000 firmy Xilinx. 
Kilka miesięcy później firma rozszerzyła portfolio o „OcPoC-Cyclone”, którego sercem został układ Intel FPGA Cyclone V \cite{Aerotenna}.
%TODO 2 a podali skąd taka zmiana Xilinx -> Intel ? %ODP Nie ma zmiany, to jest niezależny produkt (pewnie dla fanów pracy z określonym środowiskiem)

\section{Układy rekonfigurowalne w systemach wizyjnych dla platformy UAV}

Inną grupę rozwiązań stanowią układy realizujące kontrolę wysokiego poziomu, czyli wykorzystanie dodatkowych informacji w celu zapewnienia określonego poziomu autonomiczności.
 
\subsection{Zadanie detekcji i śledzenia}
Jednym z podstawowych sposobów przetwarzania materiału wideo jest ekstrakcja jego cech w celu detekcji, klasyfikacji i śledzenia obiektów oraz utrzymywania orientacji kamery. 
W publikacji \cite{RHOG} porównano dwie grupy algorytmów:
\begin{itemize}
	\item algorytmy detekcji cech: SIFT, FAST, STAR, SURF, ORB, HCD, D-HCD
	\item algorytmy opisu cech: SIFT, FEAK, BRIEF, SURF, ORB, HOG, R-HOG.
\end{itemize} 

Następnie w oparciu o algorytmy D-HCD oraz R-HOG stworzono zintegrowany system w układzie Zynq, który na podstawie analizy cech obrazów pochodzących z dwóch kamer ($1080\times 1920$ @ $30$fps) został wykorzystany w zadaniu trójwymiarowego śledzenia scen -- ale może sprawdzić się również w zadaniu śledzenia obiektów, multimodalnej rejestracji obrazów lub generowaniu struktur 3D na podstawie ruchu. 
%TODO 2 a konktrestnie to co on robił ? %ODP doprecyzowano
System z powodzeniem poddano testom na materiałach zarejestrowanych w trakcie lotu, a przy zapotrzebowaniu na moc na poziomie $4$W, częstotliwości odświeżania $30$Hz i opóźnieniu wynoszącym mniej niż $3$ klatki obrazu, deklasuje rozwiązanie uruchomione na laptopie z 8-rdzeniowym procesorem Intel i7 2.8GHz, na którym częstotliwość pracy uruchomionego algorytmu to zaledwie ok. $2$Hz.

Inna praca \cite{FIRE} opisuje system detekcji ognia i ludzi na podstawie obrazów rejestrowanych na dużej wysokości (około 2km). 
W obu przypadkach przetwarzanie dotyczy obrazów wizyjnych oraz termowizyjnych. 
Na takich materiałach rzeczywista odległość pomiędzy środkami dwóch sąsiednich pikseli wynosi $12$cm--$25$cm, zatem osoby znajdujące się na ziemi będą przedstawione za pomocą kilku pikseli.

Detekcja ludzi wymaga rozszerzenia analizy o kształt i wielkość cieni oraz charakter ruchu. 
Wykorzystywane są tu dwa rodzaje obszarów: jeden z nich związany jest z wykrywaną osobą; jest odpowiedzialny za odrzucenie obiektów, których rozmiar nie odpowiada oczekiwanej uśrednionej wielkości człowieka na obrazie. 
Dodatkową referencję stanowi obraz termowizyjny, gdzie informacja o wydzielanym cieple w określonym miejscu może zwiększyć prawdopodobieństwo obecności człowieka. 
Drugi obszar reprezentuje cień osoby, którego kierunek powinien być związany z pozycją słońca w danej lokalizacji i określonym momencie dnia. 
Informacje te można uzyskać poprzez komunikację z urządzeniami działającymi na dronie: GPS i kompasem. 
Parowanie obszarów obu typów pozwala wykryć osoby na obrazie.

Detekcja ognia polega na wykryciu dużej różnicy w temperaturach pomiędzy obszarem ogarniętym pożarem a jego tłem. 
Obszar taki jest następnie klasyfikowany pozytywnie lub negatywnie (wykorzystywany jest SVM) w oparciu o wektor cech związanych ze zmianami chromatyczności badanego obszaru. %TODO 2 klasyfikowany, klasyfikacje... %ODP OK
Autorzy pracy opisują układy rekonfigurowalne jako najlepszą platformę do realizacji systemu detekcji  -- tak ze względu na pobór mocy, wymiary, jak i moc obliczeniową przewyższającą wydajność procesorów wielordzeniowych.
\begin{figure}[h]
	\centering
	\includegraphics[width=12cm]{fire.png}
	\caption{Przykładowa detekcja ludzi (po lewej) i ognia \cite{FIRE}}
	\label{fig:fire}
\end{figure}

\subsection{Zadanie omijania przeszkód}

Systemy ostrzeżenia przed kolizją lub omijania przeszkód stają się powoli standardowym elementem wyposażenia dronów. Są one niezastąpione w sytuacji, gdy użytkownik traci maszynę z pola widzenia. 

Jedną z najczęściej rozważanych realizacji takiego systemu na platformach latających jest wykorzystanie stereowizji. 
Jej działanie polega na wyznaczeniu współrzędnych punktów sceny trójwymiarowej na podstawie obrazów uzyskiwanych za pomocą co najmniej dwóch kamer, co w efekcie umożliwia określenie odległości od przeszkód lub celów. 
Opisana w pracy \cite{STEREOVISION} implementacja algorytmu SGM (Semi-Global Matching) w układzie FPGA (XC7A100T) pozwoliła utworzyć mapę dysparycji w oparciu o obraz o parametrach $480\times 752$ @ $60$fps. 
Przetworzone dane są przechwytywane przez procesor wykorzystywany zazwyczaj w smartfonach (Samsung Exynos 4412 SoC) i zapisywane w pamięci DDR2. 
Praca ta została później wykorzystana na małej platformie UAV \cite{STEREOVISION2}, gdzie spełniała rolę systemu omijania przeszkód o niskiej latencji (poniżej $2$ms).
\begin{figure}[h]
	\centering
	\captionsetup{justification=centering,margin=1cm}
	\includegraphics[width=13cm]{stereovision.png}
	\caption{System omijania przeszkód \cite{STEREOVISION2}. Po lewej wejściowy obraz w skali szarości, po prawej mapa dysparycji}
	\label{fig:stereovision}
\end{figure}

Kolejnym rozwiązaniem jest system detekcji linii zasilania \cite{STEREOVISION3}. 
W przypadku małych statków powietrznych, latających zazwyczaj na wysokości kilku, kilkunastu metrów, istnieje zwiększone ryzyko kolizji platformy UAV z instalacją elektryczną. 
Opracowany system wykorzystuje obraz stereowizyjny, poddając go równolegle transformacie Censusa oraz transformacji Top-Hat i określa stopień (koszt) dopasowania pomiędzy obrazami. %TODO 2 ale to chodzi o stereo -> tak wnoskuje.... %ODP Tak
Agregacja kosztów jest obliczana z uwzględnieniem obszaru wsparcia o krzyżowej postaci (eng. cross-based support region). %cross-based support region, nie mogłem znaleźć dobrego tłumaczenia
%TODO 2 - dac eng. %ODP OK
Po nieznacznej korekcie informacje są zapisywane w dwuportowej pamięci RAM i przekazywane do niezależnego procesora sygnałowego, który odpowiada za wyższą warstwę logiczną i realizuje funkcje decyzyjne.  
%TODO 2 - a doczytał Pan na jakies zasadzie to się dzieje ? %ODP w dokumencie nie opisano nic poza tym
Przetworzone informacje są ponownie przekazywane poprzez RAM do kontrolera wideo, który generuje obraz wyjściowy z wysegmentowanymi obszarami.
\begin{figure}[h]
	\centering
	\captionsetup{justification=centering,margin=1cm}
	\includegraphics[width=12cm]{line_detection.png}
	\caption{Przykład systemu detekcji linii zasilania zrealizowanego w \cite{STEREOVISION3}. Po lewej wejściowy obraz, po prawej mapa dysparycji}
	\label{fig:line_detection}
\end{figure}

\subsection{Zadanie lokalizacji drona}
Inna praca \cite{Chenini} opisuje metodę estymacji pozycji drona w oparciu o obraz z kamery zamontowanej na statku powietrznym i skierowanej pionowo w dół. 
Pozwala ona wspomóc system nawigacyjny podczas lotu w utrudnionych warunkach (obniżających skuteczność np. GPS). 
Pierwszym krokiem jest określenie punktów zainteresowań -- wykorzystuje się w tym celu metodę Harrisa do wykrycia narożników, a następnie ZNCC (ang. Zero-mean Normalized Cross-Correlation) do określenia podobieństwa pomiędzy kolejnymi obrazami. %TODO 2 rozwinąć skrót ZNCC
Otrzymany zestaw informacji zawiera sporo nieprawidłowo skorelowanych par. 
Korekcja jest dokonywana w trakcie działania metody RANSAC (ang. Random Sample Consensus) opartej o algorytm Levenberga-Marquardta. %TODO 2 rozwinąć RANSAC i na tym bym skończył, bo co tu ma epipolarna do rzeczy ? %ODP OK
Implementacja systemu uwzględniała przeniesienie stworzonego wcześniej modelu programowego do części PS i stworzenie sprzętowego akceleratora dla detektora Harrisa. 
Wymiana informacji jest realizowana przez port ACP, który zapewnia bezpośredni dostęp do pamięci cache procesora. 
Ostatecznie, porównano sposoby implementacji detekcji narożników metodą Harrisa. 
Okazuje się, że dla podejścia programowego etap ten trwa ponad $750$ms, podczas gdy sprzętowa implementacja skraca ten czas do zaledwie $173$ms.

%TODO 2 - przegląd OK. Jakby się Panu udało w artykułach odszukać np. reprezentatywane zdjęcia i je tu zmieścic to już byłoby bardzo ekstra
% !TeX spellcheck = pl_PL

\chapter{Śledzenie obiektów}
\label{cha:sledzenieObiektow}

Zadanie śledzenia obiektów jest zagadnieniem z dziedziny przetwarzania obrazów. Polega na zachowaniu ciągłości analizy ruchu obiektów i zapewnieniu ich obecności w polu widzenia kamery. W praktyce wyznacza się punkt na rejestrowanym obrazie (wartość zadaną) i steruje pozycją oraz orientacją kamery w taki sposób, by jego odległość od śledzonego obiektu (również reprezentowanego przez punkt) była jak najmniejsza. Systemy wizyjne tego typu znajdują zastosowanie m.in. w rozpoznawaniu zachowań ludzi, wykrywaniu kolizji z pieszymi (systemy ADAS), weryfikacji pracy urządzeń przemysłowych czy nawet śledzeniu pojazdów na polu walki.
%TODO rozpoznawania zachowania twarzy dziwnie brzmi. Ogólnie lepsze przykłady. Z ADAS kolizja z pieszym, nawet wojskowe na polu walki śledzenie pojazdów. $ODP OK

%---------------------------------------------------------------------------

\section{Rodzaje algorytmów śledzących}
\label{sec:algorytmySledzace}

Algorytmy śledzenia można umieścić w 4 głównych kategoriach utworzonych ze względu na metodę identyfikacji śledzonego obiektu na obrazie. 
Wyróżnia się metody:
\begin{itemize}
	\item różnicowe,
	\item częstotliwościowe,
	\item gradientowe,
	\item korelacyjne.
\end{itemize}

Ponadto, istnieje również podstawowy podział na sceny stacjonarne i ruchome. 
Wykorzystanie kamery stacjonarnej jest stosunkowo proste w realizacji i nie wymaga dużego nakładu obliczeniowego (do głównych czynności należy separacja tła i segmentacja obiektów ruchomych). %TODO uprościć. sceny: stancjonarne i ruchome. W pierwszym przyapdku.... (bo to co Pan napisał jest ciut zbyt zamotone) $ODP OK
W przypadku kamery posiadającej stopnie swobody, pomiędzy obecną i kolejną klatką obrazu zmianie ulec mogą wartości wszystkich pikseli. %TODO wartości wszystkich pikseli $ODP OK
Wyklucza to stosowanie algorytmów opartych o wyodrębnianie tła. 
Dodatkowo, jakikolwiek ruch kamery może zmieniać położenie obiektu i jego wielkość na obrazie - wymusza to wybór zdecydowanie bardziej złożonych obliczeniowo metod. %TODO może zmieniać (bo też może np. śledzić kamera obiekt i wtedy ruch będzie wolniejszy) $ODP racja, OK

\subsection{Algorytmy różnicowe}

Jest to grupa metod, które działają w oparciu o obraz różnicowy, powstały w wyniku odjęcia bieżącego obrazu od poprzedniej klatki lub wcześniej wygenerowanego modelu tła. %TODO modelu tła $ODP OK
Pozwala to na wyodrębnienie obszarów, gdzie nastąpiła zmiana, która zwykle związana jest z ruchem. %TODO, która zwykle związana jest z ruchem. $ODP OK
W celu wyeliminowania szumu (minimalnych zmian w wartościach pikseli) stosuje się progowanie \cite{Rosin}. 
Uzyskuje się maskę binarną - piksele nieruchome (0) i ruchome (1). 
Analizując taką informację, możliwe jest wyznaczenie nowego położenia obiektu i zdefiniowanie przesunięcia, które nastąpiło pomiędzy klatkami. 

Największą zaletą metod różnicowych jest ich prostota w zrozumieniu oraz implementacji, mająca również efekt w niskiej złożoności obliczeniowej. 
Z drugiej jednak strony, tak proste metody bywają zawodne w przypadku większych zakłóceń i ruchu w sąsiedztwie obiektu, błędnie interpretowanego jako właściwe przesunięcie. 
Ponadto, zastosowania tych metod ograniczają się jedynie do obrazów rejestrowanych za pomocą kamery stacjonarnej, co eliminuje je z dalszych rozważań w tej pracy. 

%TODO No dobra. A co jak będą dwa obiekty ? %ODP Pytanie o tę sekcję, czy ogólnie? Jeśli w moim projekciebędą dwie osoby, to algorytm wybierze "bardziej widoczną" i się jej będzie trzymał do czasu zasłonięcia jednej osoby drugą, wtedy się pewnie wysypie.

\subsection{Algorytmy częstotliwościowe}

Metody te opierają się na interpretacji obrazu w dziedzinie częstotliwości. 
Istnieją różne filtry częstotliwościowe, które pozwalają na wykrycie krawędzi - więc, odpowiednio zaimplementowane, umożliwiają również identyfikację obiektu na kolejnych klatkach wideo. %TODO może na kolejnych %ODP OK
 %TODO ale akurat Gabor raczej w dziedzinie przestrzennej %ODP OK, wykasowano
Praca w dziedzinie częstotliwości daje możliwość wykrycia przesunięć obiektu, które mogłyby zostać niezauważone w przypadku stosowania pozostałych typów metod. 
Swoją wysoką dokładność algorytmy te każą jednak opłacić nieporównywalnie większą złożonością obliczeniową, związaną z obliczeniem transformaty Fouriera i stosowaniem filtrów częstotliwościowych. %TODO każą jednak opłacić (styl !) %ODP OK

\subsection{Algorytmy gradientowe}

Działanie wspomnianych metod śledzenia obiektów opiera się na założeniu niewielkich zmian w luminancji (jasności, oświetleniu) oraz niewielkim przesunięciu obiektu pomiędzy kolejnymi klatkami obrazu. %TODO trzeba dodać niewielkim przesunięciu %ODP OK, gdzieś to musiałem zgubić
Istotą tych algorytmów jest znalezienie odpowiedniego przesunięcia obiektu pomiędzy kolejnymi klatkami, tak, aby wskaźnik jakości wynikający z podobieństwa obszarów był zminimalizowany. 
Poszukiwania obszaru można realizować globalnie, bądź na fragmencie będącym otoczeniem śledzonego obiektu. 
Algorytm realizujący obliczenia lokalnie może zawodzić w sytuacjach, gdy przesunięcia są zbyt duże i wykraczają poza obszar poszukiwań, jednak taka metodyka obniża złożoność obliczeniową całego algorytmu i znajduje swoje zastosowania w określonych sytuacjach. %TODO algorytm prowadzący... - nie personifikować...  %ODP OK
Spośród wielu algorytmów gradientowych stosowanych do śledzenia obiektów, najpopularniejszymi są MeanShift, Camshift oraz KLT. %TODO do HOG się nie zgodzę. A dodałbym jeszcze KLT. %ODP OK 

\subsection{Algorytmy korelacyjne}

Działanie tej grupy algorytmów polega na maksymalizacji funkcji korelacyjnej obliczanej na blokach pikseli. %TODO trochę powt. Działanie tej grupy algorytmów polega na maksymalizacji korealcji....%ODP OK
Podstawowym założeniem jest jeden wspólny kierunek przemieszczenia wszystkich punktów obiektu - nie jest to zatem najlepsze rozwiązanie dla innych typów ruchu. %TODO styl. nie działają.%ODP OK
Istotnym dla złożoności i wydajności parametrem przy implementacji tej grupy metod jest określenie rozmiaru obszaru poszukiwań -- w przypadku mniejszych, problemem bywa przemieszczenie obiektu poza obszar; w przypadku tych większych istnieje ryzyko znalezienia maksimum funkcji korelacyjnej, które nie odpowiada obiektowi. %TODO no i złożóność...%ODP poprawione
Zaletą metod korelacyjnych jest jednak mniejsza wrażliwość na zachodzące na siebie obiekty, co umożliwia śledzenie bardziej złożonych ruchów %TODO to jest niejasne %ODP OK poprawione, sam teraz nie wiem o co mogło mi chodzić, a to przeoczyłem przy sprawdzaniu.
Stosuje się je często, gdy obliczanie pochodnych (dla metod gradientowych) byłoby utrudnione -- przykładowo w przypadku gwałtownego ruchu obiektu lub niewystarczającej częstotliwości próbkowania sygnału wideo. %TODO ten klatkaż to potworek. czestotliwosci próbkowania sygnłaku wideo %ODP niby jest w używany przez filmowców, więc to już bardziej słownictwo zawodowe niż potoczne;) https://sjp.pwn.pl/poradnia/haslo/klatkaz;13832.html; ale zmieniłem
Jedną z najbardziej popularnych metod korelacyjnych jest BMA, często używana w kompresji wideo \cite{Aroh}.

\subsection{Algorytmy śledzenia przez detekcję}
Działanie tej grupy algorytmów polega na niezależnym wykrywaniu określonych obiektów w kolejnych klatkach obrazu. Detekcja jest oparta na ekstrakcji kilku charakterystycznych cech obiektu, na przykład kształtu. W kolejnym etapie tak stworzony deskryptor jest klasyfikowany jako obiekt lub odrzucany. Istotną wadą tych metod śledzenia jest duża wrażliwość na zmianę odległości od obiektu -- rozwiązuje się ten problem poprzez analizę kilku przeskalowanych obrazów, jednak podnosi to ogólną złożoność obliczeniową ekstrakcji cech. Odrębną kwestią są parametry klasyfikatora, uzyskiwane w procesie uczenia i mocno wpływające na efekty działania tych metod.
W skład tej grupy wchodzą m.in. algorytmy HOG, SIFT oraz SURF.
%TODO Jeszcze brakuje Panu śledzenia przez detekcję - i do tego ten HOG wchodzi. Z popluarnych metod to jeszcze filtry cząsteczkowe, choć one chyba też umykają Pana klasyfikacji.
%TODO kolejna kwestia to jednak wypadałoby powołać się na jakąś literaturę na podstawie, której Pan to opisał.

\subsection{Algorytmy śledzenia wykorzystujące filtry cząsteczkowe}
Idea działania tej grupy metod polega na przedstawieniu pozycji obiektu za pomocą zbioru losowych próbek (cząsteczek). Każda z nich związana jest z prawdopodobieństwem obecności obiektu w pozycji reprezentowanej przez tę cząsteczkę. Po otrzymaniu nowej klatki obrazu, cząsteczkom są przypisywane wagi na podstawie zgodności predykcji. Następnie odpowiednio cząsteczki z dobrym wynikiem są pomnażane, natomiast te o złych parametrach są usuwane z obliczeń.
Wydajność tej grupy algorytmów w dużej mierze zależy od liczby zdefiniowanych cząsteczek, co może mocno wpłynąć na złożoność obliczeniową.
%Nie wiem, czy mój opis jest do końca poprawny.
\subsection{Podsumowanie}

Ostatecznie w realizacji projektu postanowiono zaimplementować algorytm gradientowy: MeanShift i metodę śledzenia przez detekcję: HOG+SVM. %TODO HOG śledznie przez detekcję. Tzn. wiem, że tam korzysta się z gradientów, ale...%ODP OK
O takim wyborze zadecydowała możliwość wykorzystania obu rozwiązań dla obrazu pochodzącego z kamery ruchomej, co w przypadku pracy drona było decyzją naturalną. %TODO powt. algorytmów.%ODP OK
Mimo, że MeanShift jest algorytmem iteracyjnym (kryterium zakończenia obliczeń stanowi limit iteracji bądź uzyskanie określonej dokładności), to odpowiednie zaprojektowanie modułu obliczeniowego w FPGA pozwoli przetwarzać obraz w czasie rzeczywistym, tj. ukończyć przetwarzanie aktualnej klatki przed rozpoczęciem przetwarzania kolejnej - tak szybko, jak tylko uzyska się na jej temat komplet informacji. %TODO dodać, że przy odpowiednim zaprojektowaniu modułu obliczeniowego%ODP OK
Metoda ta charakteryzuje się dobrymi wynikami w przypadku obiektów zmieniających orientację -- jest to szczególnie przydatne w sytuacji, gdy miejscem umiejscowienia kamery jest dron -- maszyna podatna na podmuchy wiatru zmieniające chwilowo jej orientację względem obiektu. 

Z kolei zastosowanie algorytmu HOG/SVM będzie miało szczególne znaczenie w pierwotnej detekcji śledzonej postaci. Ponadto, jednoczesna analiza kilku przeskalowanych obrazów na danym obszarze pozwoli wybrać najlepszy wynik i określić na tej podstawie odległość kamery od celu  %TODO kompensowanie głębi- usunąć. Natomiast tą rolę trzeba opisać lepiej. Ale to i tak będzie miało sens tylko, jak wcześniej (zgodnie z sugestią) przecyzjie zostanie opisana koncepcja. Rola algorytmu HOG jest dwojaka. Odległość, ale też rozumiem inicjalizacja/re-inicjalizacja mean-shift (metody się uzupełniają) - no przynajmniej w teorii. %ODP Opisano
%Poprawność działania systemu wbudowanego będącego połączeniem obu algorytmów została potwierdzona testami, które zostały przeprowadzone na modelu programowym oraz na symulacjach kodu System Verilog. %TODO to zdanie to do innej cześci pracy. Tu na razie koncepcja. %ODP OK

Oba algorytmy były wcześniej implementowane w układach rekonfigurowalnych w niezależnych projektach \cite{Mazur},\cite{Patel}, \cite{Drozdz}.
%TODO Można też dodać, że oba algorytmy były wcześniej implementowane w FPGA (literatura). %ODP OK

\section{Algorytm MeanShift}

MeanShift jest algorytmem zaprezentowanym po raz pierwszy w 1975 roku przez K. Fukunagę i L. Hostetlera \cite{Fukunaga}. Jest to metoda znajdowania maksimum rozkładu gęstości na pewnym ograniczonym obszarze. Iteracyjny charakter algorytmu pozwala aktualizować położenie obszaru, przemieszczając go w kierunku maksimum i rozpoczynając ponownie analizę.

Z czasem stwierdzono, że MeanShift może znaleźć zastosowanie w systemach wizyjnych. Dla uprzednio zlokalizowanego na obrazie obszaru z obiektem należy zapisać go w postaci histogramu barw. W celu zapewnienia zbieżności histogram ten oblicza się w oparciu o postać jądra, którego wagi będą faworyzowały wartości pikseli w centrum obszaru. Pierwszy histogram pełni rolę wzorca, z którym porównywane są histogramy obszarów z kolejnych klatek obrazu (kandydaci). Dzięki odpowiedno zastosowanym wagom algorytm znajduje maksimum rozkładu gęstości (podobieństwa) i przemieszcza obszar w jego kierunku, by na nowym zestawie pikseli rozpatrzyć kolejnego kandydata.
%TODO TU by się przydało jakieś wprowadzanie i podanie źródła z literatury. %ODP OK

\subsection{Konwersja przestrzeni barw RGB->HSV}
\label{sec:rgb2hsv} %TODO chyba zły label %ODP OK

%TODO Opis mean-shift jest ~ poprawny, ale taki dość trudny do zrozumienia. Na początku trzeba w kilku zadaniach opisać o co chodzi - tak jak jest to na tym schamcie blokwoym i potem uszczegółowić. Proszę też jeszcze raz sprawdzić wzory. %ODP OK

Algorytm MeanShift użyty w procesie śledzenia wykorzystuje rozkład prawdopodobieństwa wybranej cechy na obrazie -- w tym wypadku barwy. 
Jest to składowa H z przestrzeni barw HSV (ang. \textit{Hue, Saturation, Value}). 
Domyślną przestrzenią do rejestracji i zapisu obrazów jest jednak RGB, zatem wymagana jest jej konwersja. 
O ile przestrzeń RGB reprezentowana jest przez sześcian określony parametrami 3 kolorów: czerwonego, zielonego i niebieskiego, to przestrzeń barw HSV opisywana jest przez stożek. 
Jego podstawą jest koło, którego kąt opisuje barwę (H -- \textit{Hue}). %TODO to jest barwa, kat opisuje barwę %ODP OK
Kolor czerwony jest reprezentowany przez kąty $0$\si{\degree} (lub $360$\si{\degree}), zielony przez kąt $120$\si{\degree}, a kolor niebieski przez $240$\si{\degree}. 
Promień koła barw opisuje nasycenie koloru (S -- \textit{Saturation}), zaś za moc światła białego, czyli jasność (V -- \textit{Value}) odpowiada wysokość stożka. 
Model HSV jest lepiej powiązany ze sposobem, w jaki postrzega ludzki narząd wzroku, dla którego wszystkie barwy są światłem odbitym od obiektów.
Rysunek \ref{fig:HSV_cone} przedstawia interpretację graficzną składowych przestrzeni HSV na stożku.

\begin{figure}[h]
	\centering
	\includegraphics[width=6cm]{2_HSV.jpg}
	\caption{Stożkowa przestrzeń barw modelu HSV \cite{HSV}} %TODO  źródło obrazu (na oko wiki) %ODP OK
	\label{fig:HSV_cone}
\end{figure}

Poniższe wzory opisują sposób wyznaczenia poszczególnych składowych przestrzeni HSV w oparciu o RGB \cite{Kryjak}. Wymagają one kilku względnie złożonych operacji dzielenia.

\begin{equation}
\label{HSV_first}
V=max(R,G,B)
\end{equation}

\begin{equation}
S=\begin{cases}
\frac{V-min(R,G,B)}{V}, & V\neq0 \\
0, & V=0
\end{cases}
\end{equation}

\begin{equation}
\label{HSV_last}
H=\begin{cases}
	0, & \text{jeśli } V-min(R,G,B)==0 \\
	\frac{60(G-B)}{V-min(R,G,B)}, & \text{jeśli } V==R \\
	\frac{60(B-R)}{V-min(R,G,B)}+120, & \text{jeśli } V==G \\
	\frac{60(R-G)}{V-min(R,G,B)}+240, & \text{jeśli } V==B 
\end{cases}
\end{equation}


%TODO Można skomentować, że wymajają one operacji dzielena, która jest wzlędnie złożona. %ODP OK

\subsection{Wektor MeanShift}
\label{ssec:MS}

Istotą działania algorytmu MeanShift jest wyznaczanie maksimum funkcji gęstości w kolejnych iteracjach. 
Dla określanego obszaru obliczany jest środek ciężkości punktów, których nagromadzenie wskazuje na większy stopień podobieństwa rozpatrywanego obszaru z oryginałem. Do nowego środka ciężkości zostaje przesunięte aktualne położenie środka po zakończeniu algorytmu. %TODO trochę nie wiadomo skąd te punkty %ODP OK

To przesunięcie nosi nazwę \textit{wektora MeanShift}, który dla zbioru \textit{n} punktów $\{x_{i}\}_{i=1..n}$ oraz środka obszaru w punkcie $y_0$ może zostać zapisany jako:

\begin{equation}
\label{eq:ms1}
M(y)=\bigg[\frac{1}{n}\mathlarger{\sum\limits_{i=1}^{n}}x_i\bigg]-y_0
\end{equation}

Prostota wzoru \eqref{eq:ms1} wynika z ujednolicenia wagi dla wszystkich punktów. %TODO referencja do wzorów \eqref - dodaje nawias %ODP OK
Wprowadzając wagę punktu zależną od jego odległości od środka obszaru -- takiej, która maleje wraz z oddalaniem się od centrum), wzór \eqref{eq:ms1} można przepisać jako:

\begin{equation}
\label{eq:ms2}
M(y)=\frac{\sum_{i=1}^{n}w_i(y_0)x_i}{\sum_{i=1}^{n}w_i(y_0)}-y_0
\end{equation}
W równaniu \eqref{eq:ms2}, $w_i(y_0)$ są wagami dla poszczególnych punktów.

Rysunek  \ref{fig:ms_vector} ilustruje przykładowe wygenerowanie wektora MeanShift, gdzie niebieskim okręgiem zaznaczono obszar podlegający działaniu algorytmu. 
Wszystkie punkty mają tu jednak identyczną wagę.
\begin{figure}[h]
	\centering
	\includegraphics[width=10cm]{2_meanshift.jpg}
	\caption{Graficzna interpretacja wektora MeanShift \cite{Egorov}}
	\label{fig:ms_vector}
\end{figure}
%TODO przy tak prostym rysunku można w paint podmienić napisy na PL %ODP OK

\subsection{Jądro obszaru}

Celem wyznaczenia jądra (ang. \textit{kernel}) dla obszaru o stałej wielkości jest przypisanie najwyższej wagi punktom, które znajdują się najbliżej środka obszaru. 
Konsekwencją tego jest nadawanie znikomych wag punktom na jego brzegach. %TODO rubieże to złe słowo (brzegi) %ODP OK
Postać jądra zwykle przyjmuje klasyczne postacie gęstości rozkładów probabilistycznych: między innymi rozkładu jednorodnego (wagi jednakowe), prostokątnego, Gaussa, Cauchy'ego, Epanechnikova. 
W niniejszej pracy zdecydowano się wykorzystać jądro trójkątne ze względu na jego dobre wyniki i jednoczesną prostotę implementacji, nie bez znaczenia podczas późniejszej realizacji sprzętowej w układzie FPGA. 
Jądro takie można opisać poniższym wzorem:

\begin{equation}
\label{eq:ms3}
K(u)=\begin{cases}
1-\frac{u}{b}, & \text{jeśli }\frac{u}{b}\leq 1 \\
0, & \text{jeśli }\frac{u}{b} > 1
\end{cases}
\end{equation}
gdzie: $u$ jest odległością pomiędzy punktem a środkiem obszaru, a $b$ jest połową jego boku (zakładając, że obszarem jest kwadrat). Rysunek \ref{fig:kernel} przedstawia jądro o wymiarach 100x100 -- w punkcie (50,50) osiąga maksymalną wartość 1 (natomiast $u=0$).
\begin{figure}[h]
	\centering
	\captionsetup{justification=centering,margin=1cm}
	\includegraphics[width=12cm]{2_kernel.jpg}
	\caption{Jądro obszaru o wymiarach $100\times100$ (wygenerowane w programie MATLAB)}
	\label{fig:kernel}
\end{figure}
%TODO rysunek może być ciut mniejszy %ODP OK

\subsection{Gęstość jądra}
Mając zbiór $n\times n$ punktów: $\{x_{i},y_{j}\}_{i=1..n,i=j..n}$ z przestrzeni $\mathbb{R}^2$ oraz jądro $K(u)$ dla punktu centralnego $P=(x,y)$ można przedstawić funkcję oszacowania gęstości:

\begin{equation}
f(P)=\frac{1}{n^2}\sum_{i=1}^{n}\sum_{j=1}^{n}K(||P-P'(i,j)||).
\end{equation}
%TODO Nie za bardzo rozumiem dlaczego są dwa wzory... %ODP OK, usunięto nieużywany dalej wzór
Różniczkując powyższe otrzymuje się:
\begin{equation}
\label{eq:K}
\nabla f(P)=\frac{1}{n^2}\sum_{i=1}^{n}\sum_{j=1}^{n}(P-P'(i,j))K'(||P-P'(i,j)||),
\end{equation}
a po podstawieniu $g(x)$ za $-K'(x)$, wzór \eqref{eq:K} może zostać zapisany jako:
\begin{equation}
\nabla f(x)=\frac{1}{n^2}\sum_{i=1}^{n}\sum_{j=1}^{n}(P'(i,j)-P)g(||P-P'(i,j)||),
\end{equation}
%TODO tu chyba coś jest nie tak z minusem....też jedna suma uciekła... %ODP OK, walnąłem się przy przeglądaniu pracy Krzyśka Mazura
Po przekształceniu wzór może być przedstawiony jako:
\begin{equation}
\label{eq:kerms}
\begin{aligned}
\nabla f(x)= &\frac{1}{n^2}\bigg[\sum_{i=1}^{n}\sum_{j=1}^{n}g(||P-P'(i,j)||)\bigg] \cdot\\ \cdot&\bigg[\frac{\sum_{i=1}^{n}\sum_{j=1}^{n}P'(i,j)g(||P-P'(i,j)||)}{\sum_{i=1}^{n}\sum_{j=1}^{n}g(||P-P'(i,j)||)} -P\bigg],
\end{aligned}
\end{equation}
Ze wzoru \eqref{eq:kerms} można wyodrębnić człon będący gradientem jądra: $g(P)=-K'(P)$. 
Drugą jego część stanowi wektor MeanShift z wagami odpowiadającymi wartościom gradientu jądra o środku w punkcie $x$. 
Zakładając niezerowość wyrażenia $\sum_{i=1}^{n}\sum_{j=1}^{n}g(||P-P'(i,j)||)$, wektor ten może przyjąć ostateczną formę:
%TODO tu ma Pan ^2, a we wzorach nie... %ODP generalnie w źródłach podawana jest norma z ^2, co wydaje mi się dziwne - opisana przeze mnie funkcja K(||P-P'(i,j)||) zwraca zwraca wartość jądra dla danej odległości od środka [K(0)], w którym osiąga maksimum. Dziwi mnie trochę to zastosowanie kwadratu przy normie, stąd usunąłęm w swoich wzorach.
\begin{equation}
M_s(P)=\frac{\sum_{i=1}^{n}\sum_{j=1}^{n}P'(i,j)g(||P-P'(i,j)||)}{\sum_{i=1}^{n}\sum_{j=1}^{n}g(||P-P'(i,j)||)} -P.
\end{equation}

\subsection{Współczynnik Bhattacharyya}
 \label{ssec:Bhat}

Zastosowanie algorytmu MeanShift do śledzenia obiektów na materiale
wideo wymaga znalezienia zależności pomiędzy funkcjami gęstości wzorca oraz kandydata. 
W tym celu należy zestawić ze sobą oba rozkłady na przykład za pomocą współczynnika Bhattacharyya, który dla $m$-wymiarowych funkcji gęstości wzorca $q_h$ oraz obszaru-kandydata $p_h$ ze środkiem w punkcie $P$ wyraża się wzorem \cite{Comaniciu}:
\begin{equation}
\label{eq:Bhat}
\rho(P)=\sum_{h=1}^{m}\sqrt{p_h(P)q_h}
\end{equation}
Wyznaczenie przesunięcia obiektu na obrazie dla kolejnych klatek obrazu wymaga znalezienia maksymalnego współczynnika Bhattacharyya, co jest tożsame (według algorytmu) ze zidentyfikowaniem najbardziej podobnego fragmentu do oryginalnego obszaru śledzonego.

\subsection{Śledzenie}

Opisane wyżej jądro i jego gradient stanowią inicjalizację całego algorytmu i są obliczane jeszcze przed zdefiniowaniem wzorca. 
Cechą obrazu, dla której będzie liczona funkcja prawdopodobieństwa, jest kolor (H z przestrzeni barw HSV, jest to liczba z zakresu 0-359). 
Jeśli położenie piksela na obrazie wzorca oznaczone jest jako $\{x_{i},y_{j}\}_{i=1..n,i=j..n}$, niech zdefiniowana będzie funkcja $b:\mathbb{R}^2\rightarrow\{1..m\}$, która odwoływać się będzie do składowej \textit{H} danego piksela.

Barwa piksela stanowi argument funkcji prawdopodobieństwa utworzonego dla \textit{H} na danym obszarze detekcji. %TODO przede wszystkim ? %ODP OK, wykasowano
Współczynnik Bhattacharyya jest \textit{de facto} porównaniem dwóch funkcji prawdopodobieństwa (wzorca i kandydata), każdej będącej histogramem o 360 przedziałach. %TODO wymiar cech to średnio brzmi %ODP OK
Podczas obliczania prawdopodobieństwa wystąpienia określonego koloru, istotną rolę odgrywać musi wartość jądra, które zwiększa wagę pikseli znajdujących się w centrum obszaru, marginalizując znaczenie tych brzegowych -- które mogłyby być częścią zmiennego w czasie tła. 
O ile w tradycyjnym histogramie wartości przedziałów zwiększa się poprzez inkrementację, to w tym przypadku zdecydowano się na powiększenie o wartość jądra odpowiadającego położeniu piksela na obszarze 100x100.
Dla przykładowej barwy $h$, funkcja gęstości prawdopodobieństwa może być zdefiniowana jako:
\begin{equation}
q_h=C\sum_{i=1}^{n}\sum_{j=1}^{n}K(||P-P'(i,j)||)\delta[b(P'(i,j))-h],
\end{equation}
gdzie $P$ to nadal środek obszaru detekcji, a symbol $\delta$ jest deltą Kroneckera. Współczynnik $C$ odpowiada za normalizację $q_h$: $\sum_{h=1}^{m}q_h=1$.
Po przekształceniu okazuje się, że:

\begin{equation}
C=\frac{1}{\sum_{i=1}^{n}\sum_{j=1}^{n}K(||P-P'(i,j)||)} 
\end{equation}

W każdej iteracji dla kandydata wyznacza się funkcję gęstości prawdopodobieństwa. 
Uwzględniając obszar ze środkiem w punkcie $P$, wzór prezentuje się następująco:
\begin{equation}
\label{eq:density}
p_h(P)=C_k\sum_{i=1}^{n}\sum_{j=1}^{n}K(||P-P'(i,j)||)\delta[b(P'(i,j))-h] 
\end{equation}

Współczynnik $C_h$ wyznacza się podobnie, jak dla wzorca, jednak z uwgzlędnieniem położenia środka obszaru kandydata, $P$:
\begin{equation}
C_k=\frac{1}{\sum_{i=1}^{n}\sum_{j=1}^{n}K(||P-P'(i,j)||)} 
\end{equation}
Następnie należy zbadać podobieństwo obu rozkładów gęstości -- wykorzystywany jest w tym celu współczynnik Bhattacharyya, opisany w rozdziale \ref{ssec:Bhat}. Rozwijając wzór \eqref{eq:Bhat} w szereg Taylora ze środkiem obszaru wzorca równym $P_0$ można otrzymać:
\begin{equation}
\label{eq:approx}
\rho(p(P),q)\approx\frac{1}{2}\sum_{u=1}^{m}\sqrt{p_u(P_0)q_u} + \frac{1}{2}\sum_{u=1}^{m}p_u(P)\sqrt{\frac{q_u}{p_u(P_0)}}
\end{equation}
Podstawienie równania \eqref{eq:density} do powyższego wyrażenia daje:
\begin{equation}
\label{similarity_est}
\begin{aligned}
\rho(p(P),q)\approx & \frac{1}{2}\sum_{u=1}^{m}\sqrt{p_u(P_0)q_u} + \\ & \frac{C_k}{2}\sum_{i=1}^{n}\sum_{j=1}^{n}\sum_{u=1}^{m}\sqrt{\frac{q_u}{p_u(P_0)}}\delta[b(P'(i,j))-h] k(||P-P'(i,j)||)
\end{aligned}
\end{equation}
Stąd można wyodrębnić:
\begin{equation}
\label{eq:wi}
w_{i,j}=\sum_{u=1}^{m}\sqrt{\frac{q_u}{p_u(P_0)}}\delta[b(P'(i,j))-h]
\end{equation}


We wzorze \eqref{similarity_est} pierwsza składowa jest niezależna od położenia środka obszaru kandydata $P$. 
Maksymalizując funkcję podobieństwa, należy skupić się na poszukiwaniu największej wartości drugiego członu. 
Wykorzystując wektor MeanShift z rozdziału \ref{ssec:MS} z wagami równymi wartościom funkcji podobieństwa wzorca i kandydata oraz wzór \eqref{eq:wi}, dla każdej klatki obrazu i w każdej iteracji należy wyliczyć aktualne położenie obszaru w oparciu o względne przesunięcie dane poniższym wzorem.

\begin{equation}
\label{eq:position}
\Delta P=\frac{\sum_{i=1}^{n}\sum_{j=1}^{n}P(i,j)\cdot w_{i,j}\cdot g(P-P'(i,j))}{\sum_{i=1}^{n}\sum_{j=1}^{n}w_{i,j}\cdot g(||P-P'(i,j)||)}
\end{equation}

\subsection{Warunki zakończenia algorytmu}

W przypadku algorytmu iteracyjnego konieczne jest zdefiniowanie celów, które należy osiągnąć. 
Uzyskanie całkowitego podobieństwa pomiędzy wzorcem i kandydatem w zmiennej sekwencji obrazów jest w realnych warunkach nie do uzyskania. %TODO powt. osiągnąć. %ODP OK
Chcąc przetwarzać obraz w czasie rzeczywistym, należy zakończyć przetwarzanie klatki przed momentem, w którym dane z następnej będą gotowe. 
Stąd dwa podstawowe warunki zakończenia algorytmu to:
\begin{enumerate}
	\item numer iteracji == maksymalna liczba iteracji
	\item $||P_{n_m}-P_{n_{m-1}}||<\epsilon$, %TODO =0 czy mniejsze od Epislon %ODP OK
\end{enumerate} 
gdzie $n$ jest liczbą porządkową klatki, $m$ numerem iteracji algorytmu MeanShift dla konkretnej klatki, a $\epsilon$ stanowi akceptowalnie małą wartość.
W momencie spełnienia jednego z powyższych warunków, akceptuje się dotychczasowo uzyskaną zmianę położenia obszaru detekcji i rozpoczyna obliczenia następnej klatki. Rysunek \ref{fig:MS_diagram} przedstawia schemat blokowy algorytmu MeanShift dla sygnału wideo \ref{fig:MS_scheme}. %TODO schemat blokowy %ODP OK
\begin{figure}
	\centering
	\hspace*{-3cm}
	\includegraphics[width=14.5cm]{2_MS_visio.png}
	\caption{Schemat blokowy algorytmu MeanShift}
	\label{fig:MS_scheme}
\end{figure}



\section{Algorytm HOG+SVM}
\label{sec:HOG&SVM}

Jedną z metod śledzenia przez detekcję jest algorytm wykorzystujący zestaw cech obrazu -- konkretnie histogram zorientowanych gradientów (ang. \textit{Histogram of Oriented Gradients - HOG}); rolę decyzyjną pełni klasyfikator nazywany maszyną wektorów nośnych (ang. \textit{Support Vector Machine - SVM}). %TODO śledznie przez detekcję. ukierunk...->zorientowanych %ODP OK
Pierwotnie, rozwiązanie to zostało przedstawione w pracy \cite{Dalal} z zastosowaniem w detekcji pieszych; po nieznacznych zmianach powinno ono osiągnąć dobre wyniki w realizowanym projekcie. %TODO 1. cite, 2 hybrya ? po ,co średni styl. %DOP OK
Metodykę klasyfikacji w oparciu o zestaw cech można również zastosować do detekcji innych obiektów - w szczególności takich, które charakteryzują się dość unikalnym układem krawędzi.
%Oczywiście należy dodać, że dysponując odpowiednim zestawem informacji, odpowiednio trenowany mechanizm SVM byłby w stanie działać z innymi obiektami. %TODO to też takie sobie. Raczej, że podejście można też zastosować do detekcji innych obiektów, w szczególności takich, które charakteryzują się względnie unikalnym układem krawędzi. %ODP OK

Założeniem metody HOG+SVM jest przedstawienie okna detekcji w formie tzw. deskryptora (wektora cech), który pozwoli przeprowadzić proces uczenia klasyfikatora na podstawie dostępnych próbek w sposób umożliwiający skuteczną klasyfikację nowych obrazów. W tym celu okno detekcji dzielone jest na mniejsze, jednakowej wielkości obszary (komórki). Dla każdego z pikseli wewnątrz okna obliczana jest wartość gradientu, jego orientacja i moduł, po czym piksele przydzielone do jednej komórki tworzą histogram zorientowanych gradientów. Okno detekcji jest następnie reorganizowane w formę bloków składających się z współdzielonych pomiędzy sobą 4 komórek (histogramów). Każdy z bloków jest niezależnie normalizowany, po czym wszystkie są łączone w postać wektora cech.\newline
Uczenie maszyny wektorów nośnych polega na wyznaczeniu hiperpłaszczyzny, która rozdzieli deskryptory należące do dwóch klas z maksymalnym marginesem. Klasyfikacja jest etapem prostszym, podczas którego obliczana jest pozycja badanego wektora cech względem hiperpłaszczyzny (określana jest jego klasa).

%TODO Tutaj też jakieś kilka zdań ogólnych i później omówinie konkrentych etapów. %ODP OK

\subsection{Konwersja RGB->GRAY i obliczenie orientacji} %TODO Obliczenie orientacji %ODP OK
\label{sec:HOGgrad}
Pierwszym etapem jest dostosowanie pierwotnej palety barw RGB do 8-bitowej skali szarości. 
Dokonuje się tego, przekształcając każdy piksel według wzoru:
\begin{equation}
\label{eq:rgb2gray}
P_{GRAY}=0.299P_R + 0.587P_G + 0.114P_B 
\end{equation}
%Sumowanie się współczynników do jedności ma w zamierzeniu ograniczyć wynik tego przekształcenia do zakresu 0-255. %TODO zdanie niepotrzebne %ODP OK

%TODO swoją drogę, to w origninalej pracy inaczej jest taktowany obraz RGB, ale niech już będzie. %ODP Michał Drożdż oparł swój algorytm na skali odcieni szarości, wzorowałem się na nim

Następnie przeprowadzana jest operacja kontekstowa w celu obliczenia gradientów kierunkowych $g_x$ oraz $g_y$; używane maski to odpowiednio: $[-1,0,1]$ oraz $[-1,0,1]^T$. 
Mając gradienty, dla każdego piksela (o koordynatach [$i,j$]) obliczany jest moduł i kąt ze wzorów:

\begin{equation}
\label{eq:HOGangles}
\left.\begin{aligned}
m(i,j)=\sqrt{g_x(i,j)^2+g_y(i,j)^2} \\
\theta(i,j)=arctg\bigg(\frac{g_y(i,j)}{g_x(i,j)}\bigg)
\end{aligned}\right.
\end{equation}

\subsection{Histogram gradientów}

W kolejnym etapie poddawany przetwarzaniu obraz jest dzielony na kwadratowe obszary (dla jasności nazywane od teraz \textit{komórkami}) o wymiarach $4\times4$, $8\times8$ lub $16\times16$ pikseli. 
W każdej z komórek zostanie wyliczony niezależny histogram na podstawie orientacji gradientu. %TODO kątów -> orientacji %ODP OK
Oprócz rozmiaru komórki, kluczowym okazuje się również drugi parametr, czyli liczba przedziałów pojedynczego histogramu -- w tym wypadku zdecydowano, by rozpatrywany kąt przyporządkowywać jednemu z 9 przedziałów określających kierunek (z pominięciem zwrotu). Dzielą one równo zakres kątów $[0^{\circ},180^{\circ})$ -- i odpowiednio $[-180^{\circ},0^{\circ})$. %TODO przedziałów klasowych ??, komplementarnie - nie wynika z tego, że "zwrot" jest pomijane] %ODP OK
Wynika z tego, że każdy kolejny fragment o szerokości $a=180^{\circ}/9=20^{\circ}$ będzie stanowić osobny przedział histogramu. %TODO jw. %ODP OK

O ile typowy histogram tworzony jest poprzez inkrementację odpowiedniego licznika dla przedziału o 1, to tworzona struktura będzie wykorzystywać obliczony wcześniej moduł z gradientu. 
Dodatkowo, autorzy publikacji wspominają o możliwości interpolacji pomiędzy dwoma sąsiednimi przedziałami i inkrementacji poszczególnych wartości o proporcjonalne części modułu -- dowiedziono eksperymentalnie, że takie działanie pozytywnie wpływa na wyniki detekcji. %TODO nie ma tylko wypływa - są na dowody w tej pracy (eksperymentalne) %ODP OK
Jeśli zatem przyjąć, że $\theta_h$ i $\theta_l$ to wartości kątów będących środkami dwóch sąsiadujących ze sobą przedziałów, którym jednocześnie najbliżej do badanego kąta $\theta$, wówczas będzie można zdefiniować elementy $M_h$ i $M_l$ jako: %TODO inkrementy ??? %ODP OK

\begin{equation}
\label{eq:HOG_linear}
\left.\begin{aligned}
M_h(i,j)&=m(i,j)\bigg(\frac{\theta-\theta_l}{a}\bigg)\\
M_l(i,j)&=m(i,j)\bigg(\frac{\theta_h-\theta}{a}\bigg)
\end{aligned}\right.
\end{equation}

Ostatecznie, następuje powiększenie wartości dwóch odpowiednich przedziałów histogramu $H$ tworzonego w obrębie komórki zawierającej piksel o współrzędnych (i,j):

\begin{equation}
\label{eq:HOG_increment}
\left.\begin{aligned} 
H(\theta_h,i,j)&=H(\theta_h)+M_h(i,j) \\ 
H(\theta_l,i,j)&=H(\theta_l)+M_l(i,j)
\end{aligned}\right.
\end{equation}

Koncepcję zilustrowano na rysunku \ref{fig:HOG_interpolation}.  %TODO Koncepcję zilutrowano na rysunku XXX %ODP OK
Przedstawia on okrąg z ponumerowanymi przedziałami, każdy o szerokości $a$. Jeśli orientacja obliczonego gradientu jest reprezentowana przez kąt $\theta=57^{\circ}$, to przedziałami uwzględnionymi w interpolacji będą te najbliższe, o numerach $3$ oraz $4$. Bazując na wartości kątów w ich środkach, $\theta_h$ i $\theta_l$, obliczane zostaną $M_h$ i $M_l$, jako odpowiednio: $35\%$ i $65\%$ wartości modułu zgodnie z wzorami \eqref{eq:HOG_linear}.
%TODO Lepiej opisać ten rysunek ! %ODP OK
Wyjątkowym zdarzeniem jest takie, w którym kąt $\theta$ będzie znajdował w okolicy kąta $0^{\circ}=360^{\circ}$ (z przedziałami 1 oraz 9). 
Wówczas, dla wygody, w równaniach \eqref{eq:HOG_linear} należy użyć kątów $\theta_l=350^{\circ}$ oraz $\theta_h=370^{\circ}$.

\begin{figure}[h]
	\centering
	\hspace*{1cm}
	\includegraphics[width=12cm]{2_HOG_interpolation.jpg}
	\caption{Przykładowy problem liniowej interpolacji}
	\label{fig:HOG_interpolation}
\end{figure}

\subsection{Normalizacja}

Zazwyczaj, analizowany materiał wideo będzie rejestrowany w warunkach, w których ciężko zagwarantować równy poziom oświetlenia poszczególnych fragmentów obrazu. %TODO nie amatroskich... w warunkach, w których ciężko zagwarantować... %ODP OK
Zakłócenia te negatywnie wpływają na wyniki działania algorytmu.
%Zdegenerowany w ten sposób obraz nie dawałby dobrych wyników działania algorytmu.  
%TODO Zakłócenie te nagatywnie... %ODP OK
Z tego powodu proponuje się stosowanie blokowej normalizacji. 

Proponowane podejście zakłada utworzenie struktur nazywanych \textit{blokami}, gdzie każdy z nich obejmować ma $2\times 2$ sąsiednie komórki. 
Dla obrazu, na bazie którego utworzono $N \times M$ komórek, powinno powstać $N-1 \times M-1$ bloków -- przy ich generowaniu wykonuje się krok o jedną komórkę w poziomie i/lub w pionie względem poprzedniego obszaru. 
Rozmiar pojedynczego bloku sugeruje, że będzie się on składał z 4 histogramów, w formie wektora: 
\begin{equation}
v=[H_{i,j}, H_{i,j+1}, H_{i+1,j}, H_{i+1,j+1}],
\end{equation} 
gdzie: $i$  wiersz, $j$ - kolumna. Następnie należy dokonać normalizacji, wykorzystując jedną z sugerowanych zależności: %TODO raczej dokonać normalizacji.. %ODP OK

\begin{equation}
\label{eq:HOG_norm1}
\left.\begin{aligned} 
L1&=\frac{v}{\sum_{i}^{n}v_i+\epsilon}
\end{aligned}\right.
\end{equation}

\begin{equation}
\label{eq:HOG_norm2}
\left.\begin{aligned} 
L1_{sqrt}&=\frac{v}{\sqrt{\sum_{i}^{n}v_i+\epsilon}}
\end{aligned}\right.
\end{equation}

\begin{equation}
\label{eq:HOG_norm3}
\left.\begin{aligned} 
L2&=\frac{v}{\sqrt{\sum_{i}^{n}v_i^2+\epsilon^2}}
\end{aligned}\right.
\end{equation}

\begin{equation}
\label{eq:HOG_norm4}
\left.\begin{aligned} 
L2_{hys}&=\frac{v}{\sqrt{\sum_{i}^{n}v_i^2+\epsilon^2}}, v\leq 0.2
\end{aligned}\right.
\end{equation}
W powyższych równaniach $\epsilon$ to stała o małej wartości, a $n$ to liczba elementów w bloku (4 histogramy po 9 przedziałów $\rightarrow$ 36). Wektory $L$ zebrane z całego okna detekcji budują ostateczny wektor cech, na którym pracować będzie klasyfikator.

Niech podsumowaniem rozważania będzie przykład -- okno detekcji o rozmiarze $200\times 400$ i komórka o wielkości $8 \times 8$. 
Wynikiem opisywanej procedury będzie utworzenie aż $25\cdot50=1250$ komórek i $24\cdot 49=1176$ bloków. 
Po normalizacji, klasyfikator otrzymałby aż $1176\cdot4\cdot9=42336$ wartości do przetworzenia. 
Obsługa tak dużej ilości danych niesie za sobą konieczność pewnych optymalizacji, co będzie rozważone w rozdziale poświęconym implementacji algorytmu.
%TODO przykłąd nietafiony, bo zawsze się wykonuje detekcję w oknach %ODP wielkość zmieniono, chodziło mi raczej o przejście przez obliczenia na jakimkolwiek przykładzie.

\subsection{Klasyfikator SVM}

Maszyna wektorów nośnych to klasyfikator binarny, który dzięki swej prostocie i skuteczności jest powszechnie wykorzystywany w procesie odróżniania klas w różnych aplikacjach, w tym wizyjnch \cite{Gunn}. %TODO w różnych aplikacjach, w tym wizyjnch. %ODP OK
Jego działanie tradycyjnie można podzielić na dwie fazy: uczenie i klasyfikację. 
Etap uczenia to relatywnie dłuższy czasowo proces wyznaczenia hiperpłaszczyzny, która z jak najlepszym marginesem rozdzieli wejściowe wektory cech obiektów klasyfikowanych jako \textit{1} od tych określonych jako \textit{0} (umownie). 
Stosowane podeście nie powinno odbiegać od klasycznej metodologii uczenia maszynowego -- dysponując zbiorem danych wejściowych, należy dokonać podziału na próbki uczące oraz testowe (w stosunku około 70:30). %TODO trenujące -> uczące %ODP OK
Każdy z tych podzbiorów (a przede wszystkim uczący) powinien posiadać zarówno obiekty zdefiniowane jako pozytywne, jak i negatywne. 
W niniejszej pracy, która ma na celu rozpoznanie osoby w postawie stojącej, wymagane będzie użycie obrazów o rozmiarch $128\times64$, przedstawiających taką postać w jak największej liczbie kombinacji (ale w podobnej skali), lecz także obrazów ze zróżnicowaną scenerią, na których żadnych osób jednak nie ma. 
%TODO z tym wyraźnie odmiennym to byłbym ostrożny  ... %ODP OK
Dla poprawienia ostatecznych wyników należy zadbać również o zróżnicowanie obrazów pod kątem poziomu oświetlenia, a nawet tła otaczającego postać.
Etapem klasyfikacji nazwać można wykorzystanie otrzymanych w procesie uczenia parametrów do wyznaczenia położenia badanego wektora cech względem hiperpłaszczyzny. 
Na tej podstawie do obiektu zostanie przydzielona odpowiednia klasa.

%TODO przydałoby się równianie SVM z opisem. %ODP niżej

\subsubsection{Uczenie}

Aby detektor mógł działać poprawnie, klasyfikator musi dysponować odpowiednimi parametrami. 
Uzyskuje się je na etapie uczenia, używając odpowiedniego zestawu danych.
Wzorując się na pracy Dalala oraz Triggsa, postanowiono skorzystać ze zbioru ponad 5000 obrazów udostępnionych przez twórców. 
Specyfikacja tego zestawu pozwala nauczyć klasyfikator rozpoznawania osoby na obrazie o wielkości $128\times 64$ pikseli. 
Wysokość przedstawionej osoby, wyrażona w pikselach, powinna oscylować wokół 95 ($\pm$ 5). 
Zgodnie z przyjętymi metodologiami, zestaw ten zawiera gotowy podział na próbki pozytywne i negatywne, dalej pogrupowane na zestaw uczący oraz testowy w stosunku zbliżonym do 70:30. %TODO uczący/testowy %ODP OK
Przykłady przedstawiono na ilustracji:

\begin{figure}[h]
	\centering
	\captionsetup{justification=centering,margin=1cm}
	\includegraphics[width=14.5cm]{2_HOG_image_examples.jpg}
	\caption{Przykłady obrazów ze zbioru wykorzystanego przy uczeniu -- na górze próbki pozytywne, na dole negatywne}
	\label{fig:HOG_image_examples}
\end{figure}
%TODO no ale dlaczego te megatywne mają inne rozmiary ? powinny mieć przecież takie same. I może wiecej przykładów. %ODP rozmiary były inne w bazie INRIA - ale do uczenia/testów wykorzystywało się i tak wycinki 128x64. Dodano więcej przykładów
Wśród próbek pozytywnych połowa plików to duplikaty odwrócone horyzontalnie. Ma to zapewnić poprawne uwzględnienie cech osób niezależnie od orientacji względem kamery.

\subsubsection{Klasyfikacja}
\label{sec:klasyfikacja}
Etap ten jest stosunkowo prosty i stanowi główną zaletę klasyfikatora SVM. 
Wymaga bowiem wcześniejszego przeprowadzenia etapu uczenia oraz dostarczenia wektora cech klasyfikowanego obiektu. 
Realizowane jest wówczas równanie:
\begin{equation}
\label{eq:HOG_classification}
\left.\begin{aligned} 
r=\sum_{i=1}^{N}(a_i\cdot l_i)+b,
\end{aligned}\right.
\end{equation}
gdzie $l_i$ to elementy wektora cech, natomiast $a_i$ i $b$ to parametry wyliczone na etapie uczenia. 

W projekcie, proces uczenia i klasyfikacji oparto o funkcje dostępne w środowisku MATLAB, gdzie to równanie jest nieco zmodyfikowane:
\begin{equation}
\label{eq:HOG_classificationMATLAB}
\left.\begin{aligned} 
r=-\sum_{i=1}^{N}a_i(l_i+b_i)-c,
\end{aligned}\right.
\end{equation}
gdzie $a$, $b$ oraz $c$ to parametry hiperpłaszczyzny wyliczone na etapie uczenia ($a$ i $b$ - wektory, $c$ - skalar).

Dla obu przypadków, detekcja wskazuje obiekt jako przynależący klasy jeśli $r>0$, w przeciwnym wypadku go odrzuca.
%TODO nie podoba mi się a postać równania. Zykle jest inna. suma wagi x wektor cech +b > 0 %ODP opisano nieco więcej na ten temat

\subsection{Detekcja na ramce obrazu}
%TODO Wcześniej trzeba zaznaczyć, że rozważania dotyczą klasyfikacji próbek z sylwektami/lub bez o rozmiarch np to 128 x 64. %ODP OK
%TODO A ten rozdział to detekcja na ramce obrazu. %ODP OK

Klasyfikator jest w stanie rozpoznać postać znajdującą się w centrum próbki o rozmiarze $128\times 64$. 
%TODO druga część zdania niejasne %ODP OK, określono w pierwszej części zdania

Osiągnięcie dobrych wyników detekcji wymaga zastosowania mechanizmu okna przesuwnego. Dysponując określonym obszarem zainteresowań, algorytm powinien sprawdzić każdą możliwą konfigurację położenia okna $128\times 64$ wewnątrz tego obszaru. Oznaczać to przesuwanie okna o 1 piksel w bok, a po dotarciu do prawej linii o 1 piksel w dół. Takie podejście skutkuje bardzo dużą złożonością obliczeniową - dla obszaru zainteresowań o wymiarach $144\times 96$ i rozpatrywanego okna istnieje 512 możliwości (512 wektorów cech). W praktyce zwiększa się wykonywany krok o kilka lub kilkanaście pikseli.

Metoda HOG+SVM jest mało odporna na zmianę wielkości wykrywanego obiektu na oryginalnym obrazie, a porównując wymiary ramki ($1280\times 720$) z wielkością okna detekcji, może wydawać się to poważnym ograniczeniem. Skutecznym rozwiązaniem okazuje się być zastosowanie algorytmu HOG+SVM na kilku przeskalowanych w różny sposób kopiach obrazu. Z każdej skali należy wydobyć próbki obrazów o wymiarach $128\times 64$ i znaleźć najlepszą pozytywną detekcję. Odpowiadająca wybranemu wynikowi skala da również pośrednią informację o odległości. 

Metodyka detekcji w wielu skala stanowi dobrą odpowiedź na jedno z założeń systemu, mówiące o konieczności określenia odległości kamery od rozpoznanej osoby, by poprzez odpowiednie sterowanie utrzymać zadany dystans.  %TODO z tym kierunkiem ? -> niejasne %ODP OK

Opisują to rysunki \ref{fig:HOG_scaling}. %TODO Ilustrują i nie poniżej tylko \ref %ODP OK
\begin{figure}[h]
	\centering
	\captionsetup{justification=centering,margin=1cm}
	\hspace*{0cm}
	\includegraphics[width=15.5cm]{2_scaling.jpg}
	\caption{Skalowanie 1:1, 1:2 oraz 1:4 z naniesionym przykładowym oknem detekcji $128 \times 64$}
	\label{fig:HOG_scaling}
\end{figure}
\newline

Stała wielkość okna detekcji każe wnioskować, że osoba wykryta na oryginalnym obrazie byłaby znacząco oddalona od kamery. W przypadku przedstawionym na rysunku, klasyfikator prawdopodobnie rozpozna postać na oknie detekcji z obrazu przeskalowanego w stosunku 1:4. %TODO styl. to nie okno rozpoznaje... %ODP OK
Znając przybliżoną odległość pomiędzy kamerą oraz osobą wykrywaną przez takie okno na oryginalnym obrazie i stosując proste twierdzenie Talesa, można oszacować jej rzeczywistą odległość dla pozytywnej detekcji w jednej ze skal. Opisuje to poniższy wzór:
\begin{equation}
\label{eq:scaling}
d_r=\frac{d_o}{s_c},
\end{equation}
gdzie $d_r$ to aktualna odległość pomiędzy osobą i kamerą , $d_o$ to odległość do wykrytej osoby na obrazie oryginalnym (zmierzona), a $s_c$ to skala obrazu (w postaci ułamkowej), w którym okno detekcji uzyskało najlepszy wynik.

%TODO co jest rozumiene przez odległość ? %ODP wyjaśniono



%TODO tu komplemetnie brakuje  opisu jak działa tzw. okno przesuwne.  Same skale też można lepiej opisać. %ODP Dodano opis, w rozdziale o implementacji jest o tym więcej napisane
%TODO Kolejna sprawa to tzw. grupowienie detekcji z wielu skal. %ODP to mam opisane w implementacji 

%TODO rozdziały o klasyfikacji i uczeniu połączyć z SVM, a to przetwarzanie dla cłąej ramk na końcu. %ODP OK


\chapter{Implementacja modelu programowego}


Pierwszym etapem weryfikującym poprawność przedstawionej koncepcji było stworzenie modeli programowych algorytmów MeanShift oraz Hog+SVM, bez warstwy nadzorującej/sterującej dronem. 
Wybrano środowisko MATLAB ze względu na jego powszechne zastosowanie w branży naukowej/inżynierskiej, co skutkuje olbrzymią bazą bibliotek i materiałów pomocniczych. 
W tym przypadku MATLAB ułatwia pracę na obrazach lub sekwencjach wideo poprzez: wczytywanie materiału, wyświetlanie, dostęp do poszczególnych klatek oraz zapewnia wiele wbudowanych funkcji, m.in. do konwersji określonych przestrzeni barw oraz klasyfikacji z wykorzystaniem maszyny wektorów nośnych. 

\section{Model algorytmu  MeanShift}

Skrypt rozpoczyna działanie od wyznaczenia jądra obszaru \eqref{eq:ms3} o wymiarach $100 \times 100$ (zgodnie ze schematem \ref{fig:kernel_build}) oraz jego gradientów. %TODO tu by się przydały odnośniki do stosowanych wzorów. %TODO 2 nie dodał Pan %ODP OK
Po tym następuje właściwa część algorytmu, która pracuje na wczytanym materiale wideo, przekonwertowanym do przestrzeni HSV. 
Obszarem śledzonym jest kwadrat $100 \times 100$, który dla pierwszej klatki obrazu jest zlokalizowany w miejscu obecności śledzonego obiektu. 
Dla tego fragmentu obliczany jest histogram barw wzorca. 
Następnie, dla kolejnych klatek, obliczany jest histogram kandydata. Na podstawie dwóch takich zestawów danych obliczany jest wektor MeanShift, czyli przemieszczenie maksymalizujące wartość funkcji podobieństwa obu obszarów. %TODO 2 dalej średni styl. %ODP OK
Dla każdej z klatek operacja ta jest przeprowadzana 10 razy, poprawiając precyzję ostatecznego przesunięcia. 
Przed rozpoczęciem przetwarzania kolejnej klatki, następuje aktualizacja pozycji śledzonego obszaru. 
Po wykonaniu algorytmu dla zdefiniowanej liczby klatek, skrypt wyświetla film w oryginalnych barwach RGB z dorysowaną czerwoną ramką otaczającą śledzony obszar, co pozwala wizualnie sprawdzić poprawność działania całego kodu. 
Niestety, czas wykonania tej symulacji jest nieporównywalnie dłuższy od rzeczywistego trwania sekwencji i, przykładowo, na komputerze wyposażonym w~procesor klasy i7 materiał o długości 685 klatek (ok. 11 sekund filmu) jest przetwarzany w~czasie ok. 100 sekund.

Rysunek \ref{fig:meanshift_prog} przedstawia śledzenie obiektu w postaci czerwonej koszulki z ciemnym poziomym paskiem. Kolejne zrzuty są realizowane z odstępem 100 klatek. Zauważyć można poprawne działanie algorytmu nawet pomimo zmiany odległości obiektu od kamery.
 
\begin{figure}[]
	\centering
	\includegraphics[width=16cm]{3_meanshift.jpg}
	\caption{Śledzenie MeanShift}
	\label{fig:meanshift_prog}
\end{figure}


\section{Model algorytmu HoG+SVM}

Funkcję modelu programowego można podzielić na dwa zadania. Pierwszym jest proces uczenia klasyfikatora SVM, którego współczynniki zostaną użyte nie tylko w drugiej części, ale również w trakcie implementacji sprzętowej. 
Drugim zadaniem jest realizacja detekcji osoby na wielu skalach obrazu wejściowego, w oparciu o wskazany przez użytkownika punkt środkowy obszaru zainteresowań.
 
%TODO 2 te dwa zdanie dalej nie jasne - proszę to jakoś przeredagować. Też niejasna gezena rozmiaru tego okna... %ODP OK poprawiono, wcześniej model wykonywał detekcję na jednej wybranej skali, ostatnio to zintegrowano

\subsection{Uczenie}
Uczenie jest dość złożonym obliczeniowo procesem przetworzenia obrazów i ekstrakcji wektorów cech, na podstawie których powstanie płaszczyzna rozdzielająca próbki pozytywne od negatywnych. 
Przejście przez tak wielką liczbę obrazów ma charakter jednorazowy, po którym będzie dysponować się parametrami umożliwiającymi klasyfikację. 
Z tego względu nie zdecydowano się na implementację modułu uczącego w układzie FPGA -- tak ze względu na złożoność tego etapu, czas trwania procedury, jak i prawdopodobnie duży udział modułu w~zużyciu zasobów logiki. 

Wybór padł na środowisko MATLAB dysponujące funkcjami wspierającymi metodę SVM. 
Skrypt korzysta z plików tekstowych zawierających listy próbek do wczytania, które ostatecznie muszą być w rozmiarze $128\times 64$. 
Wymiary próbek negatywnych mocno odbiegają od założonego wymiaru klasyfikowanego obrazu ($128\times 64$). 
Zdecydowano, iż w ich przypadku uczeniu poddany będzie obszar o wymiarach $128\times 64$ wycięty ze środka oryginałów. 
Proces tworzenia wektorów cech przeprowadzono zgodnie z informacjami zawartymi w rozdziale \ref{sec:HOG&SVM}; za normę blokową obrano L2 \eqref{eq:HOG_norm3}.
Na bazie tak przygotowanych próbek powstaje tablica deskryptorów oraz druga, która przechowuje odpowiadające im wartości klas (1 lub 0). 
Z obu tablic korzysta dostępna w pakiecie MATLAB funkcja \textit{svmtrain}, która zwraca strukturę \textit{svm\char`_struct} ze współczynnikami wykorzystywanymi w klasyfikacji. 
Z kolei funkcja \textit{svmclassify} służy do klasyfikacji obrazu (a właściwie jego deskryptora). 
Funkcję tę użyto w procesie testowania niezależnego zestawu danych złożonego z manualnie sklasyfikowanych obrazów, by sprawdzić skuteczność klasyfikacji wyuczonego SVM. %TODO 2 poprawnie, czy manulanie ? tzn. nie rozmiem do czego się to poprawnie odnosi %ODP racja, manualnie to lepsze słowo
Etap testowania jest dodatkowo przydatny -- pozwala eksperymentalnie zdefiniować optymalny rozmiar komórki (kwadratu pikseli), na bazie których liczony jest pojedynczy histogram. 
Tabela \ref{tab:HOG_cell_size} prezentuje wyniki, wśród których uwzględniony jest błąd, liczony jako stosunek liczby niepoprawnych klasyfikacji (0 zamiast i 1 zamiast 0) do liczby testowanych obrazów.
Eksperyment przeprowadzono na komputerze wyposażonym w procesor Intel klasy i7, trzeciej generacji.
\newcolumntype{P}[1]{>{\centering\arraybackslash}p{#1}}
\begin{table}[h]
	\centering
	\captionsetup{justification=centering,margin=1cm}
	
	\begin{tabular}{|P{3cm} |P{3cm} |P{4.5cm}| P{3.5cm}|}	
		\hline
		\rowcolor{lightgray} Rozmiar komórki [px] & Liczba elementów wektora cech & Błąd klasyfikacji [\%]  & Czas trwania obliczeń [s] \\ 
		\lbrack$2\times $2\rbrack			& 70308		& 3.4069		& 52min 42s		\\ 
		\hline		
		\lbrack$4\times 4$\rbrack			& 16740 	& 2.3344		& 14min 08s		\\ 
		\hline
		\lbrack$8\times 8$\rbrack			& 3780		& 3.3438		& 05min 01s		\\ 
		\hline
		\lbrack$16\times 16$\rbrack			& 756		& 5.1735		& 02min 50s		\\ 
		\hline
	\end{tabular}
	
	\caption{Skuteczność algorytmu na zbiorze testowym w zależności od wielkości pojedynczej komórki}
	\label{tab:HOG_cell_size}
\end{table}

Okazuje się, że algorytm wykazuje największą skuteczność dla komórek o rozmiarze $4\times~4$. 
Zdecydowano się zatem na implementację algorytmu HOG w układzie FPGA właśnie dla tej wartości parametru. 
Szybkie obliczenia pokazują, że dla obrazu o wymiarach $128\times 64$ powstanie $32\times 16$ komórek, a idąc dalej, $31\times 15$ bloków po 4 histogramy po 9 wartości -- długość wektora cech będzie wynosić 16740. 
Oznacza to również, że po zakończonym procesie uczenia powstanie wektor współczynników o takiej długości.


%Działanie skryptu rozpoczyna się od przeprowadzenia uczenia klasyfikatora, która bazuje na dwóch tablicach: jedna agreguje wektory cech, a druga przechowuje skojarzone z nimi klasy (0 lub 1). 
%Wykorzystywana baza obrazów wyposażona jest w pliki z ich listami (osobno dla pozytywnych i negatywnych przykładów), co umożliwia proste wczytywanie kolejnych próbek. 
%Oczekuje się tu rozmiaru $128\times 64$ pikseli, jednak niektóre (zwłaszcza negatywne) obrazy są za duże -- przeprowadza się tutaj wycięcie fragmentu o podanym wyżej rozmiarze ze środka każdej próbki. Kolejnymi etapami dla takiego obszaru są kolejno: konwersja do odcieni szarości oraz obliczenie wektora cech. 


\subsection{Rozpoznawanie}

Po uzyskaniu struktury ze współczynnikami możliwa jest klasyfikacja obszarów z obrazów wczytanych przez użytkownika.
Napisany w tym celu skrypt rozpoczyna działanie od wczytania obrazu wejściowego. 
Po zdefiniowaniu punktu będącego środkiem obszaru zainteresowań następuje realizacja algorytmu HOG+SVM dla każdej z 5 zdefiniowanych wcześniej skal. Wielkość obszaru zainteresowań określono na podstawie potrzeb i możliwości związanych z~implementacją sprzętową - dla każdej ze skal ma on rozmiar $144 \times 96$.
Pierwszym krokiem jest obliczenie $5\times 9$ wektorów cech na fragmencie $144 \times 96$ -- celem jest znalezienie jak najlepszego wyniku detekcji.  %TODO 2 - jak Pan wczesniej napisze dlaczego akturat taki rozmiar tego ROI to będzie OK, bo tak to jest niejasne. %ODP OK, zdanie wyżej
Wykorzystując strukturę ze współczynnikami i wzór \ref{eq:HOG_classificationMATLAB}, następuje klasyfikacja każdego obliczonego wektora cech. %TODO 2 dlaczego "idąc za przykładem" ? %ODP poprawiono
Proces ten jest powtarzany dla każdej z wcześniej określonych skal. 
Końcowym etapem jest wyświetlenie oryginalnego obrazu w kolorze, z naniesionymi konturami: obszarów zainteresowań $144\times 96$ oraz pozytywnych detekcji o~wymiarach $128\times 64$ -- przeskalowanych do rzeczywistego rozmiaru. 
Finalna detekcja jest obliczana na podstawie najlepszego (najniższego) wyniku klasyfikacji z zebranych wyników analiz wszystkich skal.

Przykłady detekcji w pojedynczych skalach zaprezentowano na rysunkach \ref{fig:SVMmodel}, \ref{fig:SVMmodel_2_5} oraz \ref{fig:SVMmodel_3}.
Różowym konturem oznaczono okno detekcji $144 \times 96$, natomiast na czerwono zakreślono ostateczne obszary $128\times 64$ pozytywnie wybrane przez klasyfikator. Przedstawiono również tabele z wynikami klasyfikacji poszczególnych deskryptorów. Pogrubione zostały pozytywne wyniki detekcji.
\begin{figure}[h]
	\centering
	\includegraphics[width=12cm]{3_SVM_model.jpg}
	\caption{Detekcja na oknie wewnątrz obrazu (skala: 5)}
	\label{fig:SVMmodel}
\end{figure}

\newcolumntype{P}[1]{>{\centering\arraybackslash}p{#1}}
\begin{table}[h]
	\centering
	\caption{Tabela z wynikami dla $5 \times 9$ detekcji z rysunku \ref{fig:SVMmodel}}
	\begin{tabular}{|P{1.2cm} |P{1.2cm} |P{1.2cm} |P{1.2cm} |P{1.2cm} |P{1.2cm} |P{1.2cm} |P{1.2cm} |P{1.2cm} |}
		
		\hline
$1.4357$ &   $1.0343$  &  $1.4887$  &  $1.1960$  &  $1.1270$  &  $1.2742$ &   $0.6366$  &  $0.9465$  &  $0.8936$ \\ \hline
$0.9876$ &   $1.1790$  &  $1.7361$  &  $1.4707$  &  $1.0850$  &  $1.2012$ &   $1.2482$  &  $1.5096$  &  $1.2834$ \\ \hline
$1.4188$ &   $1.3449$  &  $2.1487$  &  $1.8528$  &  $0.9389$  &  $0.7458$ &   $1.0194$  &  $1.5121$  &  $1.7702$ \\ \hline
$1.5967$ &   $2.0461$  &  $1.7974$  &  $1.1585$  &  $0.6553$  &  $0.6955$ &   $0.5781$  &  $1.6717$  &  $1.3235$ \\ \hline
$1.2971$ &   $2.0113$  &  $1.2289$  & $\textbf{-0.0191}$  & $\textbf{-0.5477}$  & $\textbf{-1.0460}$ &   $0.2423$  &  $1.0235$  &  $1.7798$ \\ \hline		
	\end{tabular}
	\label{tab:scale_window_cover_5}
\end{table}

\begin{figure}[!]
	\centering
	\includegraphics[width=12cm]{3_SVM_model_2_5.jpg}
	\caption{Detekcja na oknie wewnątrz obrazu (skala: 2.5)}
	\label{fig:SVMmodel_2_5}
\end{figure}

\newcolumntype{P}[1]{>{\centering\arraybackslash}p{#1}}
\begin{table}[!]
	\centering
	\caption{Tabela z wynikami dla $5 \times 9$ detekcji z rysunku \ref{fig:SVMmodel_2_5}}
	\begin{tabular}{|P{1.2cm} |P{1.2cm} |P{1.2cm} |P{1.2cm} |P{1.2cm} |P{1.2cm} |P{1.2cm} |P{1.2cm} |P{1.2cm} |}
		
		\hline
    $2.1223$  &  $1.9234$  &  $1.8072$ & $1.8854$  &  $1.6932$ &  $1.5320$  & $ 0.5443$ &  $ 0.2648$  & $\textbf{-0.0366}$ \\ \hline
$1.9547$  &  $2.5649$  &  $1.6400$ & $1.7075$  &  $1.1940$ &  $1.5821$  & $ 0.6492$ &  $ 0.2804$  & $\textbf{-0.0880}$ \\ \hline
$2.5165$  &  $2.7328$  &  $2.3592$ & $2.2650$  &  $2.1196$ &  $1.1509$  & $ 0.8788$ &  $ 0.6314$  & $ 0.1222$ \\ \hline
$2.4744$  &  $2.8618$  &  $2.7310$ & $2.6338$  &  $2.1627$ &  $1.4763$  & $ 0.4520$ &  $\textbf{-0.7166}$  & $\textbf{-1.1757}$ \\ \hline
$2.6806$  &  $2.4200$  &  $2.2032$ & $1.6812$  & $1.9216$ &  $0.9355$  & $\textbf{-0.0505}$ &  $\textbf{-1.3537}$  & $\textbf{-1.3712}$ \\ \hline	
	\end{tabular}
	\label{tab:scale_window_cover_2_5}
\end{table}

\begin{figure}[!]
	\centering
	\includegraphics[width=12cm]{3_SVM_model_3.jpg}
	\caption{Detekcja na oknie wewnątrz obrazu (skala: 3)}
	\label{fig:SVMmodel_3}
\end{figure}


\newcolumntype{P}[1]{>{\centering\arraybackslash}p{#1}}
\begin{table}[!]
	\centering
	\caption{Tabela z wynikami dla $5 \times 9$ detekcji z rysunku \ref{fig:SVMmodel_3}}
	\begin{tabular}{|P{1.2cm} |P{1.2cm} |P{1.2cm} |P{1.2cm} |P{1.2cm} |P{1.2cm} |P{1.2cm} |P{1.2cm} |P{1.2cm} |}
	\hline	
    $1.8159$  &  $1.8308$  &  $1.7164$  &  $1.4868$  &  $1.1955$  &  $1.1542$  &  $0.8346$  & $ 0.4625$   & $0.6144$ \\ \hline
$2.3357$  &  $2.6091$  &  $2.0017$  &  $2.0776$  &  $2.2638$  &  $1.8270$  &  $0.6915$  & $\textbf{-0.0323}$  &  $0.6554$ \\ \hline
$2.3233$  &  $2.4182$  &  $1.9292$  &  $2.1127$  &  $1.8924$  &  $1.7307$  &  $0.8044$  & $ 0.3019$  &  $0.7611$ \\ \hline
$1.5705$  &  $1.6276$  &  $1.5120$  &  $1.7848$  &  $1.6619$  &  $1.7323$  &  $0.4511$  & $ 0.8625$  &  $1.1717$ \\ \hline
$1.4874$  &  $1.8153$  &  $1.0768$  &  $1.2884$  &  $1.0186$  &  $1.8310$  &  $0.8986$  & $ 0.6349$  &  $1.1386$ \\ \hline
	\end{tabular}
	\label{tab:scale_window_cover_3}
\end{table}

Następnie wykonano eksperyment pozwalający potwierdzić działanie detekcji dla implementowanych sprzętowo skal: $2/2.5/3/3.5/4$. 
W tym celu nagrano sekwencję wideo, która przedstawia postać zbliżającą się w stronę kamery z odległości ok. $10$m. 
Skrypt z wczytanym materiałem wideo realizuje detekcję na wielu skalach i zaznacza na ramkach obrazu wszystkie obszary zainteresowań oraz najlepsze okno detekcji. %TODO 2 to "w oparciu" średnie... %ODP OK
Wyniki przedstawia rysunek \ref{fig:SVMmodel_dist}. 
Dla obu detekcji z pierwszego rzędu, najlepsze wyniki osiągnięto na obszarze w skali równej $2$. 
Należy zwrócić uwagę, że ze względu na brak uwzględnienia mniejszych skal w teście, na obrazie przeskalowanym ze współczynnikiem $2$ może zostać wykryta osoba o małych rozmiarach względem okna detekcji (przypadek w lewym górnym rogu rysunku). 
Rzędy drugi oraz trzeci przedstawiają poprawnie wykrytą postać w pozostałych skalach: odpowiednio $2.5/3/3.5/4$. 

Eksperyment posłużył również do oszacowania odległości postaci od kamery, które należałoby przyporządkować detekcjom w poszczególnych skalach. 
Na podstawie minimalnej i maksymalnej odległości osoby od kamery w sekwencji wideo (ok. $2m-10$m) można oszacować, że detekcja w skali $4$ odpowiada odległości ok. $2$m, natomiast każda kolejna skala powiększa tę wartość o ok. $1-1.5$m, aż do odległości ok. $7m$ dla detekcji w skali $2$. 
Tak jak wspomniano wcześniej, w skali tej wykrywana jest również dalej stojąca osoba (nawet $10$m), choć nie jest to regułą.
\begin{figure}[h]
	\centering
	\includegraphics[width=15cm]{HOG_scale_all.jpg}
	\caption{Detekcja dla wielu skal}
	\label{fig:SVMmodel_dist}
\end{figure}


\chapter{Implementacja sprzętowa systemu wizyjnego}

%TODO krótki wstępn. M. in. po co implementacja sprzętowa.
Celem implementacji sprzętowej jest realizacja określonej funkcjonalności przez dany układ scalony. W przeciwieństwie do programu komputerowego, urządzenie realizujące sprzętowo odpowiednio zaimplementowany algorytm cechuje się większą szybkością działania. Wadą jest z kolei koszt i czas produkcji takiego rozwiązania.

\section{Układy Zynq}
%TODO No właśnie FPGA czy Zynq ? Bo chyba jednak Zynq%ODP OK

Najważniejszym i najbardziej czasochłonnym elementem pracy była sprzętowa implementacja algorytmów detekcji i śledzenia. %TODO raczej sprzętowa implementacja algorytmów detekcji i śledzenia.%ODP OK
W dzisiejszych czasach istnieje możliwość skorzystania z przeróżnych platform obliczeniowych, które dzieli się, jak przedstawiono poniżej:
\begin{itemize}
	\item brak możliwości zmiany funkcjonalności po wyprodukowaniu układu:
	\begin{itemize}
		\item procesory ASIP (ang. \textit{Application Specific Instruction Set Processor}) -- zaprojektowane z dedykowanym zestawem instrukcji		
		\item układy ASIC (ang. \textit{Application Specific Integrated Circuit}) -- zaprojektowane do realizacji określonej z góry zadań
		\item procesory ogólnego przeznaczenia GPP (ang. \textit{General Purpose Processor}) %TODO raczej GPP general purpose processor %ODP OK
		\item procesory graficzne GPU (ang. \textit{Graphics Processing Unit})
		\item procesory sygnałowe DSP (ang. \textit{Digital Signal Processor})
	\end{itemize}
	\item z możliwością konfiguracji po procesie produkcyjnym:
	\begin{itemize} 
		\item układy rekonfigurowalne z architekturą gruboziarnistą,  czyli softprocesory (Nios, uBlaze)%TODO czyli co to takiego ? %ODP OK, ale nie jestem pewien
		\item układy rekonfigurowalne z architekturą drobnoziarnistą - FPGA (ang. \textit{Field-Programmable Gate Array}) oraz CPLD (ang. \textit{Complex Programmable Logic Device})
	\end{itemize}
\end{itemize}

Specjalizowane układy ASIC są niestety drogie w prototypowaniu. Proces ten oznacza stworzenie układu scalonego od podstaw, i biorąc pod uwagę brak możliwości rekonfiguracji, musi uwzględniać szereg działań związanych z zapewnieniem poprawności działania w momencie przekazania do produkcji.  %TODO może zdanie dlaczego %ODP OK

Procesory CPU nie pozwalają na przetworzenie tak wielkiej ilości danych w czasie rzeczywistym. %TODO pozwalają %ODP nie jestem pewien czy to uwaga dotycząca błędu styl. czy tego, że procesory są zdolne ogarnąć przetworzenie obrazu real-time?

Z kolei druga z wymienionych grup urządzeń ma atut polegający na możliwości zmiany architektury układu oraz zrównoleglenia obliczeń tak bardzo, jak tylko pozwala na to liczba dostępnych zasobów. %TODO Druga z wymienionych grup ilość->liczba %ODP OK
Dodatkowo -- i paradoksalnie zarazem -- urządzenia te charakteryzują się stosunkowo niskim zapotrzebowaniem na energię. 
Ma to istotne znaczenie, mając na uwadze światowy trend poszukiwań energooszczędnych rozwiązań w każdej branży -  a zwłaszcza na rynku dronów, których czas lotu jest przeważnie ograniczony pojemnością baterii do kilkunastu minut. %TODO napisać, że dla drona tym bardziej. %ODP OK
Obszary, w których FPGA znajduje zastosowanie to między innymi:
\begin{itemize}
	\item studia dźwiękowe,
	\item telekomunikacja,
	\item przemysł motoryzacyjny/zbrojeniowy/lotniczy/kosmiczny,
	\item szyfrowanie i przetwarzanie danych,
	\item aparatura medyczna,
	\item systemy wizyjne.
\end{itemize}
To wszystko nie oznacza jednak, że układy FPGA nie są pozbawione wad. 
Przykładowo, realizacja niektórych rodzajów algorytmów (głównie rekurencyjnych) jest utrudniona i pozbawiona sensu. %TODO nieprawda, można ale jest to utrudnione i nieracjonalne. Zawsze przecież może Pan "zrobić" procesor... %ODP OK
Ponadto, sam proces rozwoju konkretnej architektury nie należy do najprostszych i może wymagać potwierdzenia funkcjonalności na kilka sposobów:
\begin{itemize}
	\item porównanie wyników z modelem stworzonym w innym, konwencjonalnym środowisku (MATLAB, OpenCV). Przykładowo, pozwala to zweryfikować poprawność logiki w układzie rekonfigurowalnym, która jest tworzona w oparciu o zapis stałoprzecinkowy. Dla wspólnego scenariusza testowego, przebiegi odpowiednich sygnałów nie powinny przekraczać akceptowalnego poziomu błędu.
	\item symulacja HDL na komputerze - umożliwia weryfikację projektu lub jednego z modułów poprzez określenie wymuszeń sygnałów z testowym module zewnętrznym, tzw. \textit{testbenchu}. Proces ten aktualizuje stan logiczny wszystkich obiektów w określonym odstępie czasu, przykładowo 1ps.
	\item weryfikacja wybranych sygnałów w układzie przy użyciu zintegrowanego analizatora logicznego -  SignalTap dla urządzeń firmy Intel (dawniej Altera), ILA dla układów firmy Xilinx. Oznaczając w budowanym projekcie odpowiednie sygnały i wykorzystując protokół JTAG, możliwe jest skomunikowanie się z układem i akwizycja zdefiniowanej ilości próbek każdego z elementów. Na szczególną uwagę zasługuje opcja wyzwolenia procesu akwizycji spełnieniem określonego warunku (powiązanego ze stanem dostępnych sygnałów).  %TODO Altera to już Intel.  firmy Intel (Altera), firmy Xilinx. (takie Xilina to potoczne jest) %ODP OK, co prawda nazwa 'Altera' nadal figuruje jako spółka córka, ale produkty mają brand Intel FPGA.
\end{itemize}
%TODO Natomiast te punkty to są zbyt skórtowo opisane. 1-2 zdania więcej. %ODP OK

Część problemów bywa z czasem rozwiązywana przez producentów. 
Stopień skomplikowania projektu można znacznie zredukować, wykorzystując dostępną z oprogramowaniem własność intelektualną -- odpowiednio skonfigurowane tzw. IP Core'y -- i łącząc sygnały pomiędzy nimi na diagramach graficznych. 
Co więcej, narzędzie Vivado High-Level Synthesis (HLS) oferuje możliwość adaptacji języka C/C++ do technologii FPGA, upraszczając proces powstawania architektury. 

Postępująca zwłaszcza w ostatnich latach integracja i zmniejszanie procesu technologicznego pozwoliły na stworzenie układów łączących logikę programowalną (FPGA) oraz dwurdzeniowy procesor ARM. 
Układ, na którym oprócz jednostki obliczeniowej znajdują się różne peryferia, nosi nazwę układu heterogenicznego, w branży bardziej znanego jako \textit{System on a Chip} (SoC). %TODO Krzem (pot.) -> układ %ODP OK
Dzięki magistrali (przykładowo AXI) zapewniającej szybką wymianę danych pomiędzy blokami, taki układ pozwala łączyć możliwość zrównoleglania obliczeń, którą daje logika programowalna (\textit{ang. PL - Programmable Logic}) oraz prostotę rozwoju oprogramowania uruchamianego na procesorze (\textit{ang. PS - Processing System}). 
W tej ostatniej części istnieje możliwość pracy w systemie operacyjnym (jednym ze specjalnych dystrybucji Linuxowych) lub nawet stworzenia konfiguracji opartej o wieloprocesorowość asynchroniczną (ang. AMP - asynchronous multiprocessing), która pozwala wykorzystać oba dostępne rdzenie procesora do niezależnych celów. Poza tak prostym przypadkiem jak praca z dwiema równoległymi aplikacjami, możliwa jest nawet konfiguracja Linux+RTOS \cite{AMP}. %TODO powt. uruchomienia, rozwinąć AMP %ODP OK
Warto tu zauważyć, że narzędzia syntezy logiki od dawna oferowały możliwość stworzenia softprocesorów w układach FPGA (Altera - Nios, Xilinx - MicroBlaze), jednak te rozwiązania istotnie ograniczały dostępne zasoby, a częstotliwość taktowania takich komponentów rzadko przekraczała 200 MHz. 

Powyższa charakterystyka przekonuje, że układy FPGA, a zwłaszcza SoC mógłby być szczególnie doceniony na platformie latającej, gdyż byłby w stanie sprostać wymaganiom stawianym przez zadanie przetwarzania obrazu w czasie rzeczywistym, a dzięki niewielkim wymiarom i niskiemu zużyciu energii nie stanowiłby wielkiego obciążenia przy i tak już mocno ograniczonym czasie lotu na jednej baterii. 

W pracy zdecydowano się wykorzystać układ Zynq SoC firmy Xilinx. O wyborze zadecydował nie tylko charakter projektu wymagający zrównoleglenia wykonywanych obliczeń, ale też niski pobór mocy oraz obecność dwóch rdzeni procesora ARM do realizacji części zadań.%TODO Słabe uzasadnienie %ODP OK
Oczywiście, ze względu na poziom skomplikowania montażu tego urządzenia nie podjęto decyzji o prototypowaniu własnego obwodu drukowanego, lecz wykorzystano niewielki, lecz bogaty w peryferia zestaw PYNQ z układem Zynq SoC XC7Z020. Oprogramowaniem wspierającym rozwój architektury na to urządzenie jest pakiet Xilinx Vivado + SDK.

Część PS układu XC7Z020 jest wyposażona w:
\begin{itemize}
	\item dwurdzeniowy procesor Cortex-A9 taktowany zegarem 650MHz,
	\item kontroler pamięci DDR3 z 8 kanałami DMA oraz 4 wydajnymi portami AXI3,
	\item wydajne kontrolery 1Gb Ethernet, USB 2.0, SDIO,
	\item kontrolery SPI, UART, CAN, I2C.
\end{itemize}
Część PL stanowi logika z rodziny urządzeń Artix-7, z następującymi parametrami:
\begin{itemize}
	\item 13300 elementów slice, każdy wyposażony w 4 sześcioportowe tablice LUT oraz 8 przerzutników,
	\item 630KB szybkiej pamięci Block RAM,
	\item 4 obszary zarządzania zegarami, każdy z układem PLL i MMCM,
	\item 220 elementów DSP,
	\item konwerter analogowo-cyfrowy (XADC).
\end{itemize}

Poniżej przedstawiono pozostałe najważniejsze parametry platformy PYNQ:
\begin{itemize}
	\item 512 MB pamięci DDR3 z 16-bitową magistralą o przepustowości 1050Mbps,
	\item 16MB pamięci flash,
	\item slot na kartę MicroSD,
	\item złącza: MicroUSB z interfejsem JTAG, USB, Ethernet, 2x PMOD, 3.5mm audio, we/wy HDMI,
	\item 4 przyciski, 2 przełączniki, 4 jednokolorowe diody LED oraz 2 RGB LED.
	\item możliwość zasilania bateryjnego, poprzez USB lub dołączony zasilacz 12V.
\end{itemize}

\begin{figure}[h]
	\centering
	\includegraphics[width=10cm]{4_PYNQ.jpg}
	\caption{Platforma PYNQ z układem Zynq SoC XC7Z020 w centrum}
	\label{fig:PYNQ}
\end{figure}
%TODO Po takim wstępnie powinien nastąpić krótki opis układu Zynq, schemat, omówienie części PS i PL oraz omówinie platfomry PYNQ - zdjęcie... %ODP OK

\section{Tor wizyjny w części PL} %TODO w czści PL ? %ODP OK
\label{sec:counter}


%TODO ???  trzeba zacząć prosto. Podstawowym komponentem... %ODP OK
Podstawowym komponentem toru wizyjnego jest moduł odbierający sygnał wideo z zewnątrz -- w tym wypadku poprzez port HDMI - i poprawne zdekodowanie go do podstawowej, użytecznej przestrzeni barw RGB. 
Taki zestaw sygnałów może być w prosty sposób wykorzystany w tworzonych algorytmach. %TODO może zostać wykorzystane %ODP OK
Opcjonalne i zalecane jest również wyprowadzenie przetworzonego sygnału RGB na monitor, co pozwoli zweryfikować poprawność tworzonej architektury. 
Wyżej opisany szkielet został stworzony w tzw. \textit{Block Designie}, a poza modułami firmy Xilinx wykorzystano konwertery DVI$\rightarrow$RGB oraz RGB$\rightarrow$DVI dostarczone przez firmę Digilent -- producenta zestawu PYNQ. 
Ponadto, podstawowym zegarem systemu jest dostarczany z karty PYNQ zegar o częstotliwości $125$MHz, na bazie którego powstaje szereg pozostałych zegarów. %TODO a ten radiowy to co i skąd ? Też w tej początkowej koncepcji to powinno być opisane... %ODP pozbyłem się tu informacji o czymkolwiek niezwiązanym z torem wizyjnym
 
W zadaniu śledzenia i detekcji zdecydowanie najważniejszym parametrem jest częstotliwość próbkowania sygnału wideo ze względu na potrzebę analizy jak najmniejszych ruchów obiektu. Kolejnym parametrem jest rozdzielczość - obiekt na obrazie o wyższej rozdzielczości będzie opisywany większą ilością pikseli, jednak niekorzystnie wpływa to na zużycie zasobów układu - wymagane jest osiągnięcie pewnego kompromisu. Z tego względu założono, że transmisja z kamery i przetwarzanie wideo będzie się odbywać w oparciu o standard \textit{720p/60fps}.%TODO napisać dlaczego %ODP OK

Sygnały wideo, które można wyodrębnić po zdekodowaniu, to:
\begin{itemize}
	\item zegar taktujący pikseli ($74.25$MHz),
	\item kolor czerwony R (8 bitów),
	\item kolor zielony G (8 bitów),
	\item kolor niebieski B (8 bitów),
	\item synchronizacja pozioma -- poziom wysoki sygnalizuje koniec poziomej linii,
	\item synchronizacja pionowa -- poziom wysoki sygnalizuje koniec klatki obrazu,
	\item sygnał aktywny -- poziom wysoki sygnalizuje obecność poprawnego piksela.
\end{itemize}

Uwzględniając narzut dany przez sygnały sterujące, ostateczna częstotliwość taktowania piksela wynosi w tym przypadku $74.25$MHz. 
To właśnie ten zegar, nazywany dalej w pracy \textit{\boldmath pixel\char`_clk}, steruje pracą modułów odpowiadających za potokowe przetwarzanie obrazów w części PL. %TODO raczej steruje pracą modułów odpowiadających za %ODP OK
Poniższa ilustracja przedstawia zapis klatki z sygnałami sterującymi, gdzie jednostką jest cykl zegara taktującego piksel. 

\begin{figure}[h]
	\centering
	\includegraphics[width=17cm]{4_720p.png}
	\caption{Schemat zapisu klatki w rozdzielczości $720p$}
	\label{fig:720_frame}
\end{figure}

Algorytmy opisane w dalszej części pracy będą często posiłkować się informacją o aktualnym położeniu piksela na właściwym obrazie. 
Stworzono w tym celu licznik obliczający tę pozycję dla osi pionowej i poziomej w oparciu o długości trwania sygnałów kontrolnych z rysunku \ref{fig:720_frame}. 

 
\section{Implementacja algorytmu MeanShift} %TODO Implementacja %ODP OK

Działanie algorytmu MeanShift rozpoczyna się po otrzymaniu sygnału \textit{meanshift\_en} i polega na zdefiniowaniu wzorca, na bazie którego obliczana i zapisywana jest funkcja gęstości prawdopodobieństwa. Następnie porównuje się ją z funkcjami gęstości prawdopodobieństwa kandydatów uzyskanych w kolejnych ramkach obrazu. Na tej podstawie obliczany jest wektor MeanShift, który określa przesunięcie kandydata i jednocześnie obszar następnych poszukiwań.

Pracując nad implementacją, istotne stało się określenie wielkości obszaru śledzonego, który ma być zapisywany z każdej klatki obrazu na podstawie materiału wideo w przestrzeni barw HSV.
Uwzględniając parametry optyczne kamery, jak i rozdzielczość wejściową materiału wideo, podjęto decyzję o śledzeniu obszaru o wymiarach $100 \times 100$ pikseli i ograniczeniu jego otoczenia (czyli możliwości ruchu obiektu pomiędzy klatkami) do 15 pikseli w każdym kierunku.
Po zakończeniu zapisu fragmentu o wymiarach $130 \times 130$, obliczana jest funkcja gęstości prawdopodobieństwa. Tworzona jest ona w oparciu o jądro i jego gradient, które wygenerowano w procesie inicjalizacji. Ostatnim etapem jest obliczenie podobieństwa i określenie przesunięcia obszaru. Algorytm wykonuje ostatnie etapy w sposób iteracyjny, tj. wykorzystuje zapisane sąsiedztwo obszaru w celu obliczenia zaaktualizowanej funkcji gęstości prawdopodobieństwa i zwiększenia precyzji przesunięcia. Po określonej ilości iteracji podejmowana jest akwizycja obszaru kandydata z kolejnej klatki.
Analizując możliwości pracy układu okazało się, że algorytm jest w stanie pracować z częstotliwością \textit{60Hz}. Może bowiem przetworzyć rozpatrywany fragment z bieżącej klatki obrazu w czasie ok. XXX sekund, co przy 16.(6)ms czasu trwania ramki zdecydowanie pozwala zakończyć analizę przed zapisem obszaru z kolejnej ramki.  %TODO przedstawić tą analizę... %ODP OK


%TODO  Tu opisać koncepcję ralizacji... 

\subsection{Konwersja przestrzeni barw RGB->HSV}
%TODO To dać w subsection %ODP OK

Wstępnym etapem toru wizyjnego wewnątrz algorytmu MeanShift jest konwersja przestrzeni barw, realizowana zgodnie ze wzorami \eqref{HSV_first}--\eqref{HSV_last} w module \textit{rgb2hsv}. %TODO na drodze danych -> potoczne %ODP OK
Moduł działa w trybie potokowym pracując na zegarze \textit{pixel\char`_clk} i odpowiednio opóźnia wszystkie sygnały sterujące, co jest istotnym warunkiem poprawnego rozpoznawania odpowiednich pikseli w dalszych etapach przetwarzania.
Dane te są na bieżąco dostarczane do modułu \textit{Meanshift}, realizującego główną część zadań. 

\subsection{Inicjalizacja}

Mimo, że właściwe działanie algorytmu ma miejsce po otrzymaniu sygnału zewnętrznego (\textit{algorithm\char`_en}), wymaga on wcześniejszej inicjalizacji -- odbywa się ona tuż po zaprogramowaniu części PL układu Zynq. %TODO części PL układu Zynq ? %ODP OK
W głównej mierze jest ona związana z obliczeniem jądra i jego gradientów. 
Dane te, wykorzystywane potem w charakterze informacji tylko do odczytu, muszą tu być zapisane w dość uporządkowany sposób. 
Najlepiej nadaje się do tego konfigurowalna pamięć BRAM. 
Powinna ona przechowywać informacje dla wszystkich elementów obszaru, to jest 10000 pól. 
Jej ostateczną organizację przedstawia tabela \ref{tab:kerBRAM}. W kolumnie „Format” litera \textit{U} oznacza liczbę bez znaku, natomiast \textit{S} uwzględnia znak. Kropka oddziela dwie liczby, które określają liczbę bitów przyporządkowanych odpowiednio części całkowitej i ułamkowej.
\newcolumntype{P}[1]{>{\centering\arraybackslash}p{#1}}
\begin{table}[h]
\centering
\begin{tabular}{|P{5cm} |P{3cm} |P{2.5cm}|}

\hline
\rowcolor{lightgray} Informacja & Adres rejestru & Format \\ 
Jądro: $K(||P-P'(x,y)||)$				& 0:9999		& U3.15\\ 
\hline
Gradienty: $g_x$, $g_y$		& 10000:19999	& S0.11, S0.11\\ 
\hline
Norma: $\sqrt{g_x^2+g_y^2}$	& 20000:29999	& U0.11\\ \hline
\end{tabular}
\caption{Organizacja pamięci BRAM \textit{kernel\char`_ram}}
\label{tab:kerBRAM}
\end{table}
%TODO wypda objaśnić kolumnę format %ODP OK
\newline
Warto nadmienić, że zestaw informacji związanych z konkretnym pikselem w obszarze 100x100 opisują rejestry pod adresami:
\begin{equation}
\{100y+x, 10000+100y+x, 20000+100y+x\}, x,y=0..99,
\end{equation}
Ułatwia to zaprogramowanie dostępu do tych danych. %TODO raczej Ułatwia zaprogramowanie dostępu... %ODP OK
By umożliwić odczyt obydwu gradientów w jednym cyklu zegara, zagregowano je w wektor o długości rejestru, tj. 18 bitów. 
Obserwacje poczynione na zapisie gradientów w symulacji dowiodły jednak, że ich wartości są na tyle małe, że 3 najstarsze bity ułamkowe są zawsze równe bitowi znaku. Przykładowo, wartość $0.01$ w formacie S0.11 ma postać $12'b111111011000$. Dysponując $18/2=9$ bitami na gradient można było rozszerzyć precyzję informacji pomijając te 3 nieistotne bity i zapisując ją w rejestrze w notacji S0.11, a nie S0.8 -- co pozytywnie wpływa do dokładność obliczeń. Należy pamiętać jednak o tym, by odczyt z rejestru przechowującego gradienty dołączył te 3 bity do wartości przed rozpoczęciem jakichkolwiek obliczeń. %TODO trochę to niejasne -> opisać jak uzyskano ostateczną wartość %ODP OK

Pierwszym etapem inicjalizacji jest obliczenie jądra o wymiarach $100\times 100$. Proces ten przebiega iteracyjnie, budując kolejne warstwy jądra \ref{fig:kernel} w 49 powtórzeniach (bok obszaru -1). Wartością dodawaną do odpowiednich elementóW jądra jest $1/100/2=0.02$, zapisywana w formacie U3.15 jako $0.019989013671875$. Algorytm przechodzi przez 3 zagnieżdżone pętle, przemieszczając się po komórkach jądra \textit{\{x,y\}} i kolejnych warstwach \textit{z}. Schemat \ref{fig:kernel_build} przedstawia przebieg algorytmu. Główny warunek może wydawać się wyzwaniem, jednak wyrażenia typu $W/2$, $H/2$ wymagają jedynie przesunięcia wektora o 1 bit w prawo. Nie zmienia to faktu, że implementacja obliczeń wymaga uwzględnienia odpowiednich opóźnień, a następnie synchronizacji z odczytem i zapisem danego rejestru jądra.
\begin{figure}[h]
	\centering
	\includegraphics[width=15cm]{4_kernel.jpg}
	\caption{Proces tworzenia jądra obrazu}
	\label{fig:kernel_build}
\end{figure}

Kolejnym etapem jest obliczenie gradientów i ich normy na podstawie gotowego jądra. Dla każdego elementu odczytywane są wartości 4 sąsiadów i kalkulowane gradienty przy użyciu masek: $[1,0,-1]$ oraz $[1,0,-1]^T$. W przypadku sytuacji brzegowych niedostępny sąsiad zostaje zastąpiony rozpatrywanym elementem, a wartość gradientu jest przesuwana o jeden bit w lewo (dwukrotnie zmniejszona odległość pomiędzy obiektami skutkuje dwukrotnym powiększeniem wyniku ilorazu różnicowego).  Na tej podstawie obliczana jest norma gradientu, która wymaga zastosowania dwóch mnożarek i pierwiastkującego bloku CORDIC. Jest to proces na tyle długi, że po zapisie gradientów w pamięci zdecydowano się przejść do kolejnych elementów jądra, jedynie zapamiętując adres rejestru, w którym gotowa wartość normy powinna być zapisana.

Po zebraniu kompletu danych ustawiana jest flaga \textit{kernel\char`_rom\char`_ready}, której obecność sygnalizuje gotowość uruchomienia właściwej części algorytmu. 
Jak stwierdzono w oparciu o symulacje, w rzeczywistości inicjalizacja pamięci \textit{kernel\char`_rom} trwa około 5ms, jest to zatem pomijalnie krótki czas, niewpływający na funkcjonalność (<1 klatka obrazu).


%TODO no dobra, ale to jest jakoś liczone nie ? przecież nie jest to hard-coded. Czyli proszę opisać też elementy to obliczające. %ODP OK, opisano i wrzucono schemat

\subsection{Zapis obszaru wideo}
\label{ssec:savideo}
%TODO żeby to było jasne, to gdzieś wcześniej - na początku, trzeba opisać koncepcję realizacji tego modułu.

Na szerszy opis zasługuje sposób zapisu informacji z obrazu, należy bowiem tymczasowo zapamiętać wartości pikseli $H$, na których obliczenia będą kilkukrotnie wykonywane. 
Dużym obciążeniem dla zasobów układu byłaby próba zapisania całych klatek -- pojedyncza wymagałaby: $1280\cdot720\cdot9\text{b} = 1.037$MB dostępnego miejsca. 
Z tego względu postanowiono zapisywać jedynie obszar w aktualnym położeniu obiektu, z dodatkowym sąsiedztwem 15 pikseli z każdej strony. Szerokość sąsiedztwa dobrano metodą doświadczalną, uwzględniając maksymalne przemieszczenia obiektu pomiędzy kolejnymi klatkami i ograniczenia związane z zasobami.
W procesie zapisu uwzględniono zabezpieczenie na wypadek próby wyjścia poza przestrzeń obrazu. Opiera się na sprawdzeniu pozycji środka obszaru względem ramki obrazu - minimalna odległość od każdej krawędzi to $65$. Zabezpieczenie to funkcjonuje również w samym algorytmie, utrzymując ostateczną pozycję obszaru wewnątrz dozwolonego wycinka ramki. %TODO za długie zdanie. to zabezpieczenie to osobnego zdania i opisać lepiej %ODP OK
%Sąsiedztwo to pozwoli algorytmowi wskazać przesunięcie obszaru najbardziej zbliżone do faktycznego ruchu obiektu. 
%TODO styl. algorytmowi i w sumie niejanse. %ODP zakomentowano, parę linijek wyżej jest lepsze wytłumaczenie


Proces śledzenia musi poprawnie określić położenie aktualnego, \enquote{użytecznego} piksela, a także rozpoznać początek kolejnej klatki. 
Jest to możliwe poprzez stworzenie logiki opierającej się na licznikach: horyzontalnym i wertykalnym, która wykrywa zbocza sygnałów sterujących. %TODO analizującą -- źle brzmi i też coś nie pasuje %ODP OK
Liczniki wykorzystują informacje przedstawione na rysunku \ref{fig:720_frame}; zliczają odpowiednio do wartości 1280 oraz 720.

Otrzymanie sygnału \textit{algorithm\char`_en} inicjuje realizację algorytmu. 
Istotne jest, aby w momencie pojawienia się tego sygnału, w obszarze zdefiniowanym jako startowy (domyślnie środek obrazu) znajdował się śledzony obiekt. 
Obszar ten, będący od teraz wzorcem, jest zatrzaskiwany na najbliższej pełnej klatce obrazu. 
Następnie obliczana jest funkcja gęstości prawdopodobieństwa, opisana w podrozdziale \ref{ssec:fgp}. %TODO sekcja -> (pod)rozdzał (to jest złe słowo w PL) %ODP OK

Wartości pikseli -- linia po linii --  są przechowywane w module BRAM działającym w trybie True Dual Port, \textit{image\char`_data\char`_ram}.
Dzięki temu możliwy jest jednoczesny zapis/odczyt pod warunkiem, że porty nie pracują jednocześnie na tym samym adresie. 
Ze względu na konieczność posiadania kompletnego fragmentu obszaru i możliwość zapisu kolejnego, pamięć zdolna jest pomieścić 2 obszary z ostatnich klatek, każdy o wymiarach $130 \times 130$: łącznie 33800 adresów. 
Zamiennie, co klatkę, wykonywany jest zapis najnowszych danych do jednej połowy przestrzeni, i przetwarzanie (odczyt) drugiej \ref{tab:imageBRAM}. 
%Warto jednak zwrócić uwagę na rozbieżność w szybkości obu tych procesów --  o ile zapis działa synchronicznie z zegarem piksela, \textit{pixel\char`_clk}, to algorytm będzie odczytywał dane z prędkością \textit{\boldmath calc\char`_clk}, celowo taktowanego innym zegarem (100MHz).
%TODO pozwala to na ?? %ODP True dual port pozwala na dostęp do pamięci na dwóch różnych zegarach, ale rzeczywiście jakiś czas temu cofnąłem prędkość działania meanshift w całości do PixelClk, bo w zupełności wystarcza. Zakomentowałem to zdanie.

\newcolumntype{P}[1]{>{\centering\arraybackslash}p{#1}}
\begin{table}[h]
	\centering
	\begin{tabular}{|P{4cm} |P{3cm} |P{2cm}|}
		
		\hline
		\rowcolor{lightgray} Informacja & Adres rejestru & Format \\ 
		Klatka $t\%2==0$			& 0:16899		& U9\\ 
		\hline
		Klatka $t\%2==1$		& 16900:33799	& U9\\ 
		\hline
	\end{tabular}
	\caption{Organizacja pamięci BRAM \textit{image\char`_data\char`_ram}}
	\label{tab:imageBRAM}
\end{table}
%TODO brak referencji do tabeli w tekście %ODP OK
%\newline

%TODO Jednego nie rozumiem. Zatrzaskiwany jest też wzorzec, czy tylko są dla niego obliczany histogram.
%ODP Zatrzaskiwany jest histogram wzorca, do bieżącego śledzenia wymagane są jedynie wartości pikseli kandydata.
\subsection{Funkcja gęstości prawdopodobieństwa}
\label{ssec:fgp}

Wartość funkcji gęstości prawdopodobieństwa jest inaczej wartością histogramu dla danej barwy piksela ($H$ z zakresu 0-359). 
W tym celu utworzono pamięć BRAM, która przechowuje dwa histogramy dla wzorca oraz kandydata -- w sumie 720 rejestrów \ref{tab:histBRAM}. 
\newcolumntype{P}[1]{>{\centering\arraybackslash}p{#1}}
\begin{table}[h]
	\centering
	\begin{tabular}{|P{4cm} |P{3cm} |P{2cm}|}
		
		\hline
		\rowcolor{lightgray} Informacja & Adres rejestru & Format \\ 
		Histogram wzorca			& 0:359		& U10.15\\ 
		\hline
		Histogram kandydata		& 360:719	& U10.15\\ 
		\hline
	\end{tabular}
	\caption{Organizacja pamięci BRAM \textit{histogram\char`_ram}}
	\label{tab:histBRAM}
\end{table}
%TODO brak ref do tab %ODP OK
\newline
Histogram wzorca jest tworzony raz, w oparciu o pierwszy obraz; kolejne histogramy są związane z kandydatem i zostają zapisane w górnej połowie adresowej ($H+360$). 
Wartość piksela, umożliwiając dostęp do odpowiedniego rejestru pamięci, pozwala na odczyt dotychczasowej wartości przedziału histogramu i przygotowanie go do aktualizacji. Z kolei współrzędne piksela w odniesieniu do obszaru $100 \times 100$ są wykorzystywane do odczytu elementu jądra. Przedział histogramu jest aktualizowany zapisem sumy obu wartości.  %TODO styl. piksel nie dodaje. + lepiej to opisać ? moduł DSP użyty , %ODP OK
Format rejestrów pamięci \textit{histogram\char`_ram} pozwala na zapis wartości w formacie U10.15, czyli w zakresie [0:1023.99997]. %TODO też precyzję opisać. %ODP OK
Zdarzają się jednak przypadki, gdy większość pikseli na obszarze (zwłaszcza w centrum -- miejscu największych wartości jądra) jest jednokolorowa, co może skutkować przekroczeniem dopuszczalnych wartości dla rejestru. %TODO bez próbą %ODP OK
Chroni przed tym dodatkowy fragment logiki, który wykrywając takie zdarzenie, pozostawia rejestr z maksymalną wartością.

Wartość pikseli jest odczytywana bezpośrednio z pamięci przechowującej analizowany obszar, opisanej w podrozdziale \ref{ssec:savideo}. 
Do obliczenia histogramu wymaganych jest jedynie $100\times 100$ pikseli, pozostałe (sąsiedztwo) należy zignorować. %TODO to jest trochę niejasne %ODP OK
Wartości \textit{offset\char`_X} oraz \textit{offset\char`_Y} określają pozycję pierwszego analizowanego piksela (w lewym górnym rogu). Startując z tego miejsca, praca liczników \textit{H\char`_count} oraz \textit{W\char`_count} pomaga w zebraniu jedynie 100 pikseli ze 100 linii. %TODO styl. rejestry obliczają. %ODP OK
Przykładowo, obliczając po raz pierwszy histogram dla ostatnio zapisanej klatki zmienne \textit{offset\char`_X} i \textit{offset\char`_Y} mają wartości domyślne $(15,15)$. 
Oznacza to, że z zapisanego obszaru o wymiarach $130\times 130$ należy odczytać jego środek, a koordynaty pierwszego piksela będą wynosić $(15,15)$. W kolejnych iteracjach algorytmu dla tej samej klatki obrazu zmienne mogą przyjąć inne wartości, tworząc w efekcie funkcję gęstości prawdopodobieństwa odpowiadającą nowemu fragmentowi obszaru $100\times 100$. %TODO czyli w środku rozważanego obszaru ? %ODP tak, poprawiono
Zakres dopuszczalnych wartości dla każdej z tych zmiennych to $<0,29>$. 

Po przejściu przez wszystkie piksele obszaru następuje ustawienie flagi \textit{histogram\char`_ready}, gdzie w przypadku przetwarzania klatki-wzorca kolejnym zadaniem jest akwizycja pierwszego kandydata, natomiast dla każdej kolejnej rozpocznie obliczanie funkcji podobieństwa.
Podobnie jak \textit{image\char`_data\char`_ram}, \textit{histogram\char`_ram} działa w trybie True Dual Port. %TODO algroytm przechodzi %ODP OK
Po obliczeniu funkcji gęstości dla aktualnego obszaru, z pamięci jednocześnie są odczytywane wartości funkcji dla wzorca (port A) i kandydata (port B pamięci)- umożliwiając wykonywanie części dalszych obliczeń w sposób równoległy. %TODO też "jest on w stanie". %ODP OK

%TODO to proszę konktrertnie co jest przez jakie porty... %ODP OK,wskazano porty w zdaniu w zdaniu wyżej

\subsection{Funkcja podobieństwa}

Moduł implementujący kalkulację funkcji podobieństwa (i ostatecznie przesunięcia) obszaru, jest najbardziej złożony w tej części systemu. 
Funkcja podobieństwa, opisana wzorem \eqref{eq:position}, jest obliczana dla każdego piksela w pamięci, należącym do właściwego obszaru (bez sąsiedztwa). %TODO raczej dla każdego piksela %ODP OK

Dla każdego piksela z obszaru $100 \times 100$, będącego adresem dla pamięci $histogram\char`_ram$, wczytywana jest wartość funkcji prawdopodobieństwa wzorca i kandydata. 
Moduł oblicza ich iloraz (wzorzec/kandydat) używając bloku Divider Generator (w wersji 5.1) dostarczonego przez firmę Xilinx. %TODO konkretnie %ODP OK
Ze względów optymalizacyjnych obcięto 9 najmłodszych bitów obu wartości, gdyż nie wpływały w większym stopniu na dokładność, a ich pominięcie pozwoliło zmniejszyć latencję modułu. %TODO pozbycie -> pominięcie %ODP OK
Podczas takiego dzielenia logika musi ponadto sprawdzić obecność zera w mianowniku -- wówczas iloraz powinien być zerowany. 
Brak obsługi takiego zdarzenia doprowadziłby do otrzymania ilorazu o trudnej do przewidzenia wartości -- mogłoby to powodować poważne błędy w działaniu algorytmu. 

Gotowy iloraz należy poddać pierwiastkowaniu. 
Moduł obliczający pierwiastek wymaga, by wejście z danymi było liczbą całkowitą lub  też ułamkową -- ale z jednym bitem dla części całkowitej: U1.X. 
W tym celu wynik dzielenia -- liczba w formacie: U16.16 -- została obcięta do U13.11., a następnie ,,wirtualnie'' przesunięta o 12 bitów w prawo, do postaci: U1.23. 
Wynik pierwiastkowania wymaga przesunięcia w lewo już tylko o 6 bitów ($\sqrt{2^{12}}=2^6$), po czym jest obcinany do 16 najbardziej znaczących bitów: U7.9.
Operacje te, mimo że zawiłe, nie wpływają na wynik, lecz gwarantują poprawne działanie modułu \textit{CORDIC} realizującego pierwiastkowanie.

Wartości pierwiastka współtworzyć będą zarówno licznik, jak i mianownik ilorazu wektora MeanShift -- oddzielnie dla przesunięcia pionowego i poziomego. %TODO rozdzielnie -> oddzielnie %ODP OK
Obliczenie iloczynów wymaga pobrania z pamięci \textit{kernel\char`_ram} obu gradientów oraz ich normy. 
Odczyt jest zoptymalizowany pod zminimalizowanie liczby cykli zegara potrzebnych na odczytanie obu wartości i został oparty o maszynę stanu, będącą zresztą szkieletem całej tej części algorytmu (OPISAĆ). %TODO no właśnie opisać... %nie przyuważyłem tego braku.


%TODO Brakuje schematu blokowego modułów wchodząych w skład algorymtu. Te pamięci, konwersja itp...


\section{Realizacja algorytmu HOG+SVM}

%TODO Też trzeba zacząć od jakieś koncepcji implementacji. %ODP to mam przed właśnie poniżej

Wymagania stawiane przez system mówią o konieczności rozpoznania osoby wraz z dodatkowym określeniem jej wielkości na ekranie (skala ta pozwoli dość ogólnie wyznaczyć odległość drona od obiektu, co wpłynie na sterowanie maszyny). Głównym celem jest utrzymanie stałej odległości od postaci. %TODO w sumie można napisać/powtórzyć, że celem jest utrzymanie stałej odłegłości od pieszego. %ODP OK
Z tego względu implementacja musi wykorzystać tzw. piramidę skal, czyli równoległą detekcję HOG+SVM na serii obrazów o różnych wymiarach. %TODO serii obrazów o różnych wymiarach %ODP OK
Dysponując obrazem o rozdzielczości $1280 \times720$, należy przeskalować go do kilku mniejszych, a następnie przeprowadzić na nich opisane wcześniej operacje wyliczenia wektora cech i sklasyfikować go przy użyciu SVM. 
Odpowiednia lokalizacja postaci (i jej odległość od kamery) zostanie wybrana na podstawie najlepszego \textbf{pozytywnego} wyniku klasyfikacji, w oparciu o wartość $r$ \eqref{eq:HOG_classification}. %TODO \eqref do stosownego równania %ODP OK

Algorytm ten może rozpocząć działanie jedynie na pełnej klatce obrazu, zatem stworzono mechanizm, który niezależnie od momentu pojawienia się sygnału wyzwalającego pracę algorytmu będzie oczekiwał na zbocze opadające sygnału synchronizacji pionowej, czyli nową, pełną ramkę. Tylko w takim przypadku gwarantowane jest poprawne działanie algorytmu. %TODO trochę niejasne %ODP OK
Drugim warunkiem jest to, by w momencie nadejścia nowej klatki algorytm nie był w trakcie przetwarzania poprzedniej -- w przeciwnym wypadku histogramy mogłyby być nadpisywane nowymi wartościami. 
Oznacza to jednak, że co druga klatka obrazu będzie pomijana w przetwarzaniu, ograniczając częstotliwość pracy algorytmu do \textit{30Hz}.
%TODO rozumiem, że to nie jest w pełni potokowa implementacja %ODPOstatecznie nie, wymagane byłoby stworzenie drugiego zestawu pamięci dla wektorów cech nieparzystych klatek, a na to nie ma BRAMów

W pierwszej kolejności obraz wejściowy poddawany jest konwersji do skali odcieni szarości. 
Następna w kolei, operacja skalowania do 5 mniejszych obrazów metodą najbliższego sąsiedztwa, jest realizowana potokowo, dzięki czemu można zachować sygnały kontrolne VGA (synchronizację poziomą oraz pionową). %TODO jaka metoda skalowanie, jakie te skale, co Pan rozumie poprzez "na bieżąco". %ODP OK
Na pomniejszonych obrazach obliczane są wektory cech, bazując na komórkach o wielkości $4\times 4$ i blokach o rozmiarze $2\times2$. 
Dysponując określoną liczbą zasobów układu XC7Z020, nie jest możliwa analiza całego obrazu, a jedynie otoczenia aktualnie śledzonego fragmentu.
%TODO no to nie jest prawda, bo da się to zrobić i dla całego obrazu i dla wielu skal - są takie artykułu. Przy czym jest to niewątpliwe dość trudne i wymaga sporo zasobów %O tym wiem, odnosiłem się oczywiście do urządzenia które mam pod ręką - uściśliłem
Musi ono być jednak wystarczająco duże, by klasyfikator miał szansę wybrać jak najlepszą detekcję osoby w jego wnętrzu. 
Jak wspomniano wcześniej, wektor cech tworzony jest na wycinku o rozmiarze $128\times 64$ pikseli. 
Przedstawia to obraz \ref{fig:HOG_mesh}, gdzie taki wycinek zaznaczono zieloną linią. %TODO rzut-> obrazek oraz nie poniższy tylko ref

\begin{figure}[h]
	\centering
	\includegraphics[width=15cm]{4_scaled_hog_example.jpg}
	\caption{Analiza obrazu o rozmiarach $144\times 256$}
	\label{fig:HOG_mesh}
\end{figure}
%\newline

Jeśli algorytm będzie pracował na obszarze $144\times 96$, to zakładając stałe położenie komórek na obrazie (są to kwadraty $4\times4$ wydzielone czerwonymi liniami), powstanie łącznie $5\cdot9=45$ wektorów cech. 
Reszta obrazu zostaje zignorowana. 
Współrzędne centrum obszaru detekcji są określane na początku działania pojedynczej iteracji algorytmu i są natychmiastowo konwertowane do odpowiednich skal obrazu. Po zakończeniu klasyfikacji współrzędne odpowiadające najlepszej detekcji są konwertowane do oryginalej skali i wykorzystywane podczas analizy kolejnej dostępnej klatki obrazu.

\subsection{Konwersja RGB do skali odcieni szarości}

Obraz wejściowy jest poddawany konwersji zgodnie ze wzorem \eqref{eq:rgb2gray}. 
Używane są tu trzy równoległe mnożarki, a suma iloczynów jest zaokrąglana do 8 bitów (do postaci liczby całkowitej z zakresu 0-255).

\subsection{Skalowanie}

Kolejny etap to przeskalowanie obrazu wejściowego. 
Sama idea okazuje się być tym bardziej na miejscu, jeśli wziąć pod uwagę parametry kamery zamontowanej na dronie -- w przypadku tego projektu jest to Xiaomi Yi, urządzenie do zastosowań sportowych i charakteryzujące się dużym kątem widzenia -- 155$^{\circ}$. %TODO myśl dobra, ale styl do zmiany.. %ODP OK
W efekcie osoba oddalająca się od kamery bardzo szybko zmniejszy swoje wymiary na obrazie. 
Materiał $720\times 1280$ pikseli przeskalowano do 5 obrazów przy użyciu następujących skal:
\NumTabs{15}
\begin{itemize}
	\item \textbf{1:  2}\tab{:}\tab{$720\times 1280\rightarrow360\times 640$} pikseli
	\item \textbf{1:2.5}\tab{:}\tab{$720\times 1280\rightarrow288\times 512$} pikseli	
	\item \textbf{1:  3}\tab{:}\tab{$720\times 1280\rightarrow240\times 426$} pikseli
	\item \textbf{1:3.5}\tab{:}\tab{$720\times 1280\rightarrow205\times 365$} pikseli
	\item \textbf{1:  4}\tab{:}\tab{$720\times 1280\rightarrow180\times 320$} pikseli
\end{itemize}

Powyższe wartości pozwalają jednocześnie zachować prostotę implementacji (skale są reprezentowane w formacie U3.1) i z akceptowalnym marginesem błędu określić odległość drona od postaci. %TODO wyraźnie to złe słowo. %ODP OK

Skalowanie przebiega w dość prosty sposób i polega na pomijaniu odpowiednich wierszy lub/i kolumn oryginalnego obrazu. Jeśli założyć, że: %TODO chyba pomijanie odpowiednich/wybranych %ODP OK
\begin{itemize}
	\item $x_i$, $y_i$ -- współrzędne obrazu wejściowego,
	\item $x_o$, $y_o$ -- współrzędne obrazu wyjściowego (przeskalowanego),
	\item $s_c$ -- skala do zastosowania w pionie oraz w poziomie,
\end{itemize}
to przypisanie wartości obrazu wejściowego nastąpi przy jednoczesnym spełnieniu obu poniższych warunków:
\begin{equation}
\label{eq:scaling}
\left.\begin{aligned} 
x_i&==\lfloor s_cx_o\rfloor \\ 
y_i&==\lfloor s_cy_o \rfloor
\end{aligned}\right.
\end{equation}
Po natrafieniu na odpowiedni piksel, oprócz przypisania jego wartości do wyjścia, wystawiony zostanie sygnał sterujący \textit{valid}, bardzo ważny dla dalszej części algorytmu.
Przykład dla kilku pierwszych wartości \textit{x\char`_o} jest widoczny w tabeli \ref{tab:scaling}.
\begin{table}[h]
	\centering
	\captionsetup{justification=centering,margin=1cm}
	\begin{tabular}{|P{2cm} |P{3cm} |P{2cm}|}	
		\hline
		\rowcolor{lightgray} $x_o$ & $x_is_c$ & $x_i$ \\ 
		1		& 2.5	& 2\\ 
		\hline
		2		& 5		& 5\\ 
		\hline
		3		& 7.5	& 7\\ 
		\hline
		4		& 10	& 10\\ 
		\hline		
	\end{tabular}
	\caption{Przykładowy przebieg skalowania dla $s_c=2.5$ wraz z przypisywanymi pikselami wejściowymi}
	\label{tab:scaling}
\end{table}

Działanie modułu opiera się na stworzeniu dwóch zestawów liczników dla każdej skali. 
Pierwszy zestaw pozwala na określenie indeksów piksela wejściowego ($x_i$, $y_i$) i został opisany w sekcji \ref{sec:counter}. %TODO zdeterminowanie... dziwne słowo w tym kontkeście %ODP spolszczenie angielskiego słowa, poprawione
Drugi zestaw liczników działa zgodnie ze schematem \ref{fig:scaling_sch}. %TODO nie poniższym tylko przedstawionym na rys... %ODP OK 
Oznaczeniem \textit{@PixelClk} opisano proces oczekiwania na kolejne zbocze narastające zegara pikselowego. W momencie pojawienia się sygnału synchronizacji pionowej następuje inicjalizacja liczników koordynatami piksela zlokalizowanego w lewym górnym rogu docelowej klatki skalowanego obrazu. W trakcie otrzymywania kolejnych pikseli na wejściu porównywane są wartości obu liczników - spełniony warunek oznacza wystawienie wartości piksela na wyjście modułu wraz z sygnałem aktywnym. W przeciwnym wypadku sygnał aktywny jest zdejmowany. Obecna na schemacie zmienna \textit{flag} służy do jednorazowych inkrementacji licznika $y_0$ z uwagi na dwie kwestie:
\begin{itemize}
	\item sygnał synchronizacji poziomej jest ustawiany na 40 cykli zegara pikselowego
	\item sygnał synchronizacji poziomej jest obecny po sygnale synchronizacji pionowej a przed nadejściem pierwszego piksela
\end{itemize}   
\begin{figure}[!h]
	\centering
	\includegraphics[width=11cm]{4_scaling.jpg}
	\caption{Schemat działania licznika skalującego}
	\label{fig:scaling_sch}
\end{figure}
%TODO jednak opis tego schematu. %ODP dodano nad schematem

Gwarancją poprawnie przeprowadzonego procesu skalowania jest obecność sygnałów sterujących VGA -- tylko wtedy następuje poprawny przyrost wartości liczników. 
Z kolei inicjalizacja (po lewej stronie diagramu) ma miejsce po otrzymaniu sygnału synchronizacji pionowej, zatem podłączony do układu sygnał wideo będzie skalowany już od pierwszej pełnej klatki.

%TODO albo tu albo wcześniej jest potrzebny schemat jak to wygląda - może ogólny. Że jest ramka wejściowa. 5 wynikowych potem HOG i klasyfikacja.

\subsection{Obliczanie gradientów}

Nawet odpowiednio przeskalowane obrazy są zbyt duże, by przechowywać informację o ich gradientach w wewnętrznych zasobach układu XC7Z020. %TODO to zdanie jest niejasne... chyba, żeby je zapamiętać bez wykorzystania zewnętrznej pamięci RAM. %ODP OK
Nie ma jednak takiej potrzeby, jeśli zastosuje się implementację algorytmu z maksymalizacją przetwarzania potokowego, opartego o sygnał aktywnego piksela \textit{valid} z modułu skalowania. %TODO postawi się -> złe sformułwoanie (zastosuje,wykorzysta) %ODP OK

Implementacja gradientu pionowego jest nieco złożona, gdyż wymagane w pojedynczej operacji piksele leżą w kilku kolejnych liniach obrazu. 
Konieczne jest zapamiętanie dwóch ostatnich linii -- zrealizowano to przy użyciu dwóch kolejek FIFO \ref{fig:fifo_gradient}.
Do jednej z nich (oznaczonej numerem \#1) wpisywane są wartości bieżących pikseli.
Przejście do każdej kolejnej linii obrazu powoduje systematyczną wymianę pikseli na najnowsze - wówczas do kolejki \#2 są zapisywane wartości bezpośrednio z \#1. %TODO trochę potoczny styl tego opisu. Najlepiej rysunek (takie jak jest) %ODP nieco poprawiono
Logika została zaprojektowana w sposób pozwalający uzyskać jednoczesny dostęp do 3 kolejnych pikseli leżących w linii pionowej. 
Umożliwia to specjalny tryb modułu FIFO -- First Word Fall Through (FWFT), dzięki któremu pierwsze dostępne słowo jest natychmiastowo wystawiane na wyjście, i tylko zdejmowane (zastępowane kolejnym) w odpowiedzi na wysoki stan sygnału odczytu. %TODO nie duża załuga, tylko umożliwia to %ODP OK
Ostatecznie logika, będąc w linii $i$ ($i>1$), obliczy gradient pionowy dla piksela z linii $i-1$. %TODO nie poniżej tylko \ref. no i nie algorytm %ODP OK, \ref umieszczono wyżej
\begin{figure}[h]
	\centering
	\includegraphics[width=11cm]{4_fifo_gradient.jpg}
	\caption{Schemat działania kolejek FIFO w procesie obliczania gradientu pionowego}
	\label{fig:fifo_gradient}
\end{figure}

Obliczanie gradientu poziomego nie nastręcza już tak wielu trudności -- sąsiadujące ze sobą piksele pojawiają się tuż po sobie, jednak w tym wypadku zamiast aktualnych wykorzystywane są piksele wychodzące z FIFO \#1 i zapamiętywane w rejestrze przesuwnym. 
Oznacza to, że w chwili pojawienia się na wejściu do modułu nowego piksela ($i,j$), obliczony zostanie gradient poziomy piksela ($i-1,j-1$). 
Przez tę latencję potrzebne jest również nieznaczne opóźnienie gradientu pionowego, by obie wartości były zsynchronizowane i ustawione na wyjściu w tym samym momencie.

Sytuacje opisane powyżej dotyczą gradientów dla pikseli wewnątrz obrazu. 
Dla piksela znajdującego się na „początku” obrazu (lewa oraz górna krawędź), gradientem będzie dwukrotność różnicy pomiędzy nim a jedynym jego sąsiadem -- w odpowiedniej osi. %TODO nie na początku tylko na brzegu. Swoja drogą to normalnie się to pomija. %ODP pomija się opis, czy obliczenia gradientów brzegowych?

Poprzednie obliczenia były przeprowadzane w oparciu o sygnał aktywnego piksela ($valid$), którego zbocza były wykorzystywane do określenia gradientów aż do przedostatniego wiersza i kolumny obrazu ($i-1, j-1$). %TODO powt. obliczenia
Piksele znajdujące się przy prawej i dolnej krawędzi ekranu wymagają innego podejścia. 
W tym przypadku, wymagane było stworzenie logiki kontynuującej obliczenia i generującej wyjściowe sygnały aktywne pomimo brak sygnału ($valid$). 
Opisywana sytuacja to również brak nowych wartości pikseli dla kolejek FIFO, co skutkuje ich spodziewanym opróżnieniem krótko po odebraniu pełnej ramki obrazu.
O ile poprzednio tempo obliczania kolejnych gradientów było podyktowane obecnością nowych pikseli, to w tym przypadku logika korzysta wyłącznie z opróżnianych kolejek FIFO, redukując odstęp pomiędzy wynikami do minimum (1 cykl zegara), co jest w zasadzie bez znaczenia dla dalszych obliczeń, które korzystają z poprawnie wygenerowanego sygnału aktywnego sygnalizującego obecność wyniku na wyjściu.


%TODO to jest trochę niejasne %ODP postarałem się nieco uporządkować ten paragraf

\subsection{Histogram gradientów}

Obliczone gradienty stanowią wejście modułu obliczającego wartość $arctg(\frac{g_y}{g_x})$. %TODO a moduł nie ? %ODP OK
Fragment logiki odpowiedzialny za obliczenie modułu został zrealizowany przy użyciu dwóch mnożarek dla obu gradientów, a sumę ich kwadratów następnie poddano pierwiastkowaniu, również w bloku CORDIC. %TODO w bloku - a to opisać wcześniej przy atan %ODP OK
Wykorzystano tutaj blok IP CORDIC \cite{CORDIC}, który na wejściu spodziewa się wektora złożonego z licznika oraz mianownika o tych samych długościach, przy czym jego całkowita długość jest zaokrąglana do wielokrotności liczby 16. 
Gradienty uzyskane z poprzedniego modułu są zapisane w notacji S9.1, zatem wektor wejściowy musi mieć długość 32 -- po 16 bitów na oba gradienty.
Bardziej znaczące bity tych połówek zostały wypełnione zerami i nie mają znaczenia dla obliczeń.
 
Moduł będzie rozpoczynać obliczenia dla danych wejściowych tylko w przypadku, gdy policzone zostały oba gradienty (dwa niezależne sygnały \textit{valid\char`_x/y}) oraz gdy przynajmniej jeden z nich jest różny od zera ($\frac{0}{0}$ jest elementem nieoznaczonym, z którego nie sposób policzyć implementowaną funkcję). 
Opisane warunki \textit{valid\char`_x/y}, połączone odpowiednimi operatorami logicznymi, wyprowadzono jako sygnał aktywny modułu. %TODO potoczne spięte %ODP OK

Otrzymane wartości należy następnie umieścić w dziewięciu 20-stopniowych przedziałach, co opisuje schemat \ref{fig:hog_gradient}. 
\begin{figure}[!ht]
	\centering
	\includegraphics[width=12cm]{4_HOG_gradients.jpg}
	\caption{Schemat działania obliczeń prowadzących do utworzenia histogramu gradientów}
	\label{fig:hog_gradient}
\end{figure}
Pierwszym krokiem jest operacja mnożenia kątów podanych w radianach przez $\frac{180}{20\pi}$, która konwertuje je do liczb ułamkowych stanowiących wstępny przydział (niech będzie to $bin_f$).

Następnie, w zależności od położenia względem środka danego przedziału, wybierane są przedziały: górny i dolny w postaci liczb całkowitych: $s\_bin_{up}$ oraz $s\_bin_{dn}$. 
Ostateczne będą one jednak wymagać normalizacji do postaci liczb z zakresu 0-8 i dopiero wówczas
dwa przedziały histogramu zostaną powiększone o interpolowane wartości modułu gradientów. %TODO coś źle na początku zdania, też ta personifikacja %ODP OK

W procesie obliczania modułu, suma mnożeń podnoszących gradienty do kwadratu jest wektorem U21.2, który poprzez dopisanie bitu '0' rozszerzono do U21.3. 
Zgodnie z dokumentacją bloku CORDIC, wektor o tej długości jest traktowany jako U1.23 -- zatem jest wirtualnie przemnożony przez $2^{20}$. 
Wartość wyjściową należy później interpretować jako U11.13 (wirtualnie podzieloną przez $\sqrt{2^{20}}=2^{10}$). 
Podstawową informacją wykorzystywaną w interpolacji jest odległość $abs(bin_f)$ od środków przedziałów $s\_bin_{up}$ oraz $s\_bin_{dn}$. 
Na tej podstawie obliczane są $module_{up}$ oraz $module_{dn}$ których suma jest równa pełnemu modułowi gradientów.
Ostateczne informacje -- to jest dane o przedziałach i odpowiadające im części modułu zostały przekazane dalej, wraz z wygenerowanymi sygnałami aktywnymi, które sygnalizują obecność wyniku. %TODO co to są wygenerowane sygnały aktywne %ODP OK

Cały powyższy fragment podrozdziału skupiał się na operacjach związanych z pojedynczym pikselem. 
Teraz należy spojrzeć jednak z innej perspektywy, mianowicie na grupowanie pikseli w komórki, bloki i tworzenie wektorów cech na podstawie histogramu. 

Jak opisano we wstępie do rozdziału, najlepszy rezultat detekcji osiągnie się, realizując klasyfikację w obszarze zainteresowań dla jak największej liczby wektorów cech. Te powinny być wygenerowane dla fragmentów obrazu, których przesunięcie względem siebie jest jak najmniejsze -- zilustrowano to na rysunku \ref{fig:HOG_mesh}. %TODO to jest niejasne %ODP OK
Informacje o wektorach cech są zapisywane w pamięci BRAM, %TODO ale jakie ? %ODP OK
jednak szybki przyrost zużycia zasobów układu ogranicza implementację do przetwarzania jedynie określonej liczby obszarów w sąsiedztwie miejsca podejrzewanego o obecność postaci - dane te muszą być przechowane jednocześnie do momentu zakończenia klasyfikacji. %TODO lepiej opisać, bo nie wiadomo dlaczego ten przyrost ma nastąpić. %ODP OK

By nie zużywać cennego miejsca w blokach BRAM, zdecydowano się zapisywać surowe histogramy, a nie gotowe wektory cech -- wiedząc, że dalsza logika dokonując odczytu z tej pamięci, w odpowiedni sposób przekaże te informacje do klasyfikatora. %TODO marnować - potoczne. %ODP OK
Ostatecznie, pojedyncze okno detekcji $128\times 64$ to $32\cdot16=512$ histogramów, czyli $4608$ wartości. 
Wektor cech to aż $31\cdot15\cdot4\cdot9=16740$ wartości. %TODO no właśnie... a dlaczego właściwie użył Pan 4x4 a nie 8x8 - proszę napisać (może przy modelu programowym) %ODP opisano w podrozdziale związanym z uczeniem
Oszczędność wynikająca z zapisu pojedynczych histogramów pozwoli utworzyć znacznie więcej wektorów cech. 
Przykładowo, dla okna o wielkości $144\times 96$ należy zapisać $7776$ wartości. 
Pozwala to jednak wygenerować $9\cdot5=45$ wektorów cech. 
Gdyby zaś wpisywać je do pamięci w gotowej formie, wymagałoby to aż $16740\cdot45=753300$ elementów. %TODO to pozycji to złe słowo %ODP OK

Pamięć RAM należy potraktować jako zbiór 9-elementowych histogramów ułożonych obok siebie. 
Przetwarzanie obrazu, rozumianego jako obiekt dwuwymiarowy, wymaga odpowiedniego mapowania tworzonych wartości do postaci jednowymiarowej, adresowej. 
Schemat \ref{fig:hog_histogram_scheme} przedstawia działanie logiki na ramce obrazu w kontekście zapisu histogramów do pamięci. %TODO styl. sposób pracy %ODP OK
Symbolem „$<=$” określa się przypisanie nieblokujące, które rzeczywisty efekt będzie miało dopiero na następnym zboczu narastającym zegara (i może być zastąpione kolejnym przypisaniem w obrębie jednego cyklu zegara).
 
\begin{figure}[]
	\centering
	\includegraphics[width=16cm]{4_HOG_Histograms.png}
	\caption{Procedura wyboru adresu pamięci RAM w oparciu o pozycję aktualnego piksela}
	\label{fig:hog_histogram_scheme}
\end{figure} 
%TODO Dlaczego x2 H_RES i V_RES %ODP Poprawiono
%TODO Obawiam się, że ten schemat bez opisu słownego (po tym wymienienu paramtrów) jest nieczytelny (tzn. trudny w odbiorze) %ODP opisano po liście parametrów
 
Określenie aktualnego położenia na obrazie jest możliwe dzięki zastosowaniu liczników, opierających swoje działanie na obecności sygnału aktywnego sygnalizującego gotowe dane wejściowe.
%TODO co to jest ten syg. aktywny %ODP OK
Wykorzystano następujące parametry:
\begin{itemize}
	\item \textit{H\_SIZE}, \textit{V\_SIZE} -- rozdzielczość obrazu przeskalowanego (indywidualnie dla każdej skali),
	\item \textit{H\_middle}, \textit{V\_middle} -- współrzędne piksela środkowego, będącego w centrum analizowanego obszaru (w odniesieniu do odpowiedniej skali, dostarczone przed rozpoczęciem analizy pełnej klatki),
	\item \textit{H\_RES}, \textit{V\_RES} -- wartości określające zasięg analizowanego obszaru -- w odległości od piksela środkowego - dla tego projektu są to odpowiednio: $96/2=48$ oraz $144/2=72$,
\end{itemize}
oraz zmienne:
\begin{itemize}
	\item \textit{Hcnt}, \textit{Vcnt} -- zmienne inkrementowane odpowiednio do wartości maksymalnych \textit{H\_SIZE}, \textit{V\_SIZE} -- pozwalają określić aktualne położenie na obrazie (i względem analizowanego obszaru),
	\item \textit{HV\_valid} -- sygnał aktywny, który wysokim stanem informuje o aktualnym położeniu wewnątrz analizowanego obszaru,
	\item \textit{Hmod}, \textit{Vmod} -- liczniki modulo służące do rozdzielenia pikseli wchodzących w skład różnych histogramów (kwadratów o boku $4\times 4$),
	\item \textit{Hmult}, \textit{Vmult} -- zmienne będące bazą adresową do zapisu aktualnego histogramu (horyzontalna zmienna powiększana o $9$, wertykalna o $9\cdot24$ -- liczba histogramów w linii poziomej),
	\item \textit{calculate\_en} -- sygnalizacja zakończonego procesu obliczania i zapisywania histogramów -- gotowość do rozpoczęcia normalizacji i klasyfikacji dla danej skali. %TODO a gdzie normalizacja ? %ODP normalizacja w module klasyfikacyjnym, więc odbywa się później - ale dla poprawności dodano tu ten termin 
\end{itemize}
Logika, działająca niezależnie dla każdej skali, jest reinicjalizowana po odebraniu sygnału nowej ramki (\textit{vsync}). W momencie otrzymania informacji o kolejnym zestawie przedziałów i modułu, liczniki \textit{Hcnt} oraz \textit{Vcnt} warunkują jego obecność w oczekiwanym obszarze detekcji. Jeśli wynik jest pozytywny, liczniki modulo \textit{Hmod} i \textit{Vmod} są inkrementowane -- odpowiednio co każdy piksel w obszarze, oraz co kolejną linię w obszarze. Podzielność któregokolwiek z nich przez 4 oznacza zmianę histogramu dla kolejnych danych -- wymaga to powiększenia rejestru \textit{Hmult} o 9, lub \textit{Vmult} o 216. 
Ostatecznie, dostępy do odpowiednich adresów pamięci są przedstawione równaniem:
\begin{equation}
\label{eq:adressing_hist}
\left.\begin{aligned} 
addr_{up}&=Hmult+Vmult+s\_bin_{up} \\ 
addr_{dn}&=Hmult+Vmult+s\_bin_{dn}
\end{aligned}\right.
\end{equation}
Pamięć histogramu pracuje w trybie True Dual Port, umożliwiając jednoczesny dostęp do dwóch interpolowanych przedziałów aktualnego histogramu, $s\_bin_{up}$ oraz $s\_bin_{dn}$. 
Zapis danych do pamięci histogramu jest realizowany po odczycie aktualnych wartości komórek i powiększeniu ich o odpowiednie części modułu: $module_{up}$ oraz $module_{dn}$. %TODO Styl. %ODP OK



\subsection{Uczenie}
Założeniem jest, by podczas pracy systemu wbudowanego nie ingerować we współczynniki, a opierać się na pierwotnych wynikach uczenia. %TODO raczej ? %ODP usunięto, niepotrzebny wtręt
Dane te muszą być przechowane w odpowiedni sposób, by możliwy był do nich prosty i szybki dostęp. 
Postanowiono zapisać wektor w pamięci ROM inicjalizowanej plikami \textit{*.mem}, utworzonymi podczas wykonywania skryptu uczenia w MATLABie.
Ręcznie dostosowany moduł pamięci posiada trzy niezależne sektory (w zakresie adresowania i długości danych), inicjalizowane następującymi informacjami: 

\begin{itemize}
	\item składniki skalujące \textit{shifts} -- 16740 elementów wymaganych do przesunięcia każdego elementu wektora cech. Wartości w przedziale: $<-0.2616, -0.0527>$; precyzja zapisu: S0.11.
	\item współczynniki maszyny wektorów nośnych \textit{vectors}. Wartości w przedziale: $<   -0.0076, 0.0063>$; precyzja zapisu: S0.27 (w formacie S0.23, lecz 4 najstarsze bity mają zawsze postać bitu znaku).
\end{itemize}

%TODO nie za bardzo rozumiem te składniki i czynniki %ODP to wsyzstko to współczynniki stanowiące wyjście z funkcji 'svmtrain' MATLABa. Przeanalizowałem w kodzie, jak MATLABowy klasyfikator 'svmclassify' wykorzystuje te dane, i na tym oparłem właściwie swoją logikę.

Dodatkowym współczynnikiem jest wartość przesunięcia gotowego wyniku o precyzji S0.40, jednak jest ona przechowywana w logice. 
Powyższa pamięć zajmuje aż 36 z wszystkich 140 bloków BRAM dostępnych w rozważanym układzie. %TODO dostęþnych w rozważanym układzie %ODP OK
Należy zauważyć, iż wymusza to współdzielenie pojedynczej instancji modułu we wszystkich procesach klasyfikacji. 
Z tego względu istotne jest stworzenie logiki synchronizującej początek przetwarzania wektorów cech ze wszystkich skal -- opisane jest to kolejnym podrozdziale.

\subsection{Klasyfikacja}


Po obliczeniu wektorów cech następuje proces klasyfikacji %TODO gdzie podziała się normalizacja w blokach ???%ODP dopisano, opis po schemacie
Poprzedza ją normalizacja w blokach, która jest częścią tego modułu ze względu na obecność każdego histogramu w kilku różnych blokach. 
Odczytywane z pamięci ROM wartości współczynników klasyfikatora muszą być współdzielone pomiędzy obliczeniami przeprowadzanymi dla każdej ze skal obrazu, jednak w każdym przypadku tempo generowania histogramów nie jest jednakowe -- proces ten przebiegnie szybciej dla większych obrazów (tam analizowany fragment obrazu pojawi się na wejściu wcześniej). 
O gotowości histogramów z odpowiedniej skali informuje indywidualny dla niej sygnał \textit{calculate\_en}. 
Dopiero w momencie otrzymania wszystkich sygnałów \textit{calculate\_en} (stan wysoki na wyjściu iloczynu logicznego) rozpoczynany jest właściwy proces klasyfikacji.

Moduł odpowiadający za sklasyfikowanie informacji pochodzących z pojedynczej skali zrealizowano w formie krótkiej maszyny stanu, na którą składają się następujące etapy:
\begin{itemize}
	\item inicjalizacja -- oczekiwanie na sygnał \textit{full\_frame}, informujący o rozpoczęciu algorytmu na pełnej klatce obrazu
	\item czyszczenie pamięci przechowującej wektory cech z poprzednich uruchomień algorytmu -- etap ten ma miejsce tuż po otrzymaniu sygnału \textit{full\_frame}, który pojawia się podczas stanu wysokiego synchronizacji pionowej; jest wykonywany na tyle szybko, by pamięć mogła być bezpiecznie zapisana wartościami histogramów z nowej ramki obrazu %TODO zdącyż - potoczne. Ale to chodzi o tą pamięć histogramów ??? %ODP tak, niezbyt dobre miejsce na akurat ten element logiki, mogło być gdzieś bardziej na topie... ale już szkoda ryzykować ew. błędy.
	\item oczekiwanie na iloczyn sygnałów \textit{calculate\_en}; inicjalizacja zmiennych algorytmu
	\item właściwa normalizacja i klasyfikacja
\end{itemize}

O ile zapisane w pamięci ROM współczynniki mają postać wektora cech, tak pamięć RAM przechowuje nieuporządkowane fragmenty histogramów. 
Wymagało to stworzenia logiki, która łączy ze sobą dane ze ściśle określonych adresów pamięci, interpretując je w postać deskryptora.  
Opisuje to diagram \ref{fig:hog_feature_histrogram_address}. %TODO nie poniższy, styl.: dane z odpowiendich... %ODP OK

\begin{figure}[!ht]
	\centering
	\captionsetup{justification=centering,margin=1cm}
	\includegraphics[width=16cm]{4_HOG_Features.png}
	\caption{Procedura wyboru adresu pamięci RAM w procesie odczytu kolejnych histogramów}
	\label{fig:hog_feature_histrogram_address}
\end{figure} 

Kolejnym etapem jest normalizacja w blokach, która ze względu na prostszą realizację została umieszczona wewnątrz procesu klasyfikacji (tym bardziej, że idea bloków właściwie nie funkcjonowała we wcześniejszych modułach). %TODO w blokach, nie bloków %ODP OK
Blokiem jest struktura 4 histogramów, czyli łącznie 36 wartości. 
Odczytane z pamięci RAM dane są podnoszone do kwadratu przez mnożarkę, a następnie sumowane z kolejnymi wynikami. %TODO a nie sumowane %ODP OK, inkrementacja następuje po sumie 36 wartości
Suma 36 takich wartości, dodatkowo zinkrementowana, jest poddawana pierwiastkowaniu i będzie stanowić mianownik w procesie normalizacji powiązanego bloku (zgodnie z równaniem \eqref{eq:HOG_norm3}). 
Moduł pierwiastkujący działa potokowo, więc zwraca również wartość pierwiastkową z niepełnych sum -- dlatego ważne jest wygenerowanie sygnału aktywnego w momencie zsumowania 36 elementów, a także zatrzaśnięcie poprawnej wartości pierwiastka na czas 36 dzieleń.

Normalizacja wymusza ponowne przejście przez odpowiednie dane z pamięci RAM. 
Zastosowanie dwuportowego modułu pozwala na uzyskanie dostępu do uprzednio przetworzonych danych na drugim porcie, podczas gdy pierwszy kontynuuje zwracanie informacji potrzebnych do obliczenia współczynników normalizacji dla kolejnych bloków. %TODO kanale-> porcie %ODP OK
Poprawną kolejność danych na drugim porcie osiągnięto przez odpowiednie opóźnienie sygnału adresowego z portu pierwszego. 

Napływające potokowo dane z portu drugiego, podzielone przez odpowiedni współczynnik normalizacji, mają ustawioną flagę \textit{normalized\_valid}.
Jej stan wysoki rozpoczyna inkrementację adresu pamięci ROM przechowującej współczynniki przesunięć (\textit{shifts}) pochodzące z procesu uczenia. %TODO nie rozumiem shifts. %ODP offset ze struktury klasyfikatora w matlabie
%TODO poza tym. nachodznie bloków na siebie powoduje, że danych jest więcej niż histogramów wejściowych %ODP zgodnie ze zdaniem nieco wyżej: Szybkie obliczenia pokazują, że dla obrazu o wymiarach $128\times 64$ powstanie $32\times 16$ komórek, a idąc dalej, $31\times 15$ bloków po 4 histogramy po 9 wartości, czyli długość wektora cech będzie wynosić 16740. 
Każda z 16740 par zostanie do siebie dodana. 
Znormalizowane elementy wektora cech mają postać U11.13, zatem należało uprzednio rozszerzyć wektor \textit{shifts} do tego formatu. 

Kolejnym krokiem jest wymnożenie każdej z sum przez właściwy współczynnik maszyny wektorów nośnych. %TODO skąd ten czynnik skalujacy ? połączyłem czynnik skalujący ze struktury matlabowskiej z dalszymi czynnikami, teraz mam dwa zestawy danych - 16740 offsetów i 16740 współczynników.
Aby to osiągnąć, należało ponownie i odpowiednio opóźnić linie adresowe uzyskujące dostęp do przestrzeni adresowej danych oraz \textit{vectors}. 
Ostateczna szerokość pojedycznego elementu wynosi S7.40.

Docelowa wartość definiująca detekcję jest sumą wszystkich 16740 przetworzonych elementów oraz jeszcze jednej stałej wyznaczonej na etapie uczenia, \textit{offset} równej $-0.6327$. 
\textit{Offset} stanowi wartość początkową sumy przy rozpoczęciu obliczeń dla kolejnego wektora cech. %TODO dla kolejnego %ODP OK
Ze względu na konieczność zapewnienia dobrej dokładności w procesie sumowania, wynik jest zapisywany w notacji S8.40, co wymaga rozszerzenia dodawanych elementów o 8 kopii najbardziej znaczącego bitu (będącego zresztą informacją o znaku).

\subsection{Przetwarzanie wyników}

Niezależnie od liczby przetwarzanych skal obrazu, etap klasyfikacji jest dla nich realizowany równolegle, przetwarzając po $5\times 9$ wektorów cech.  i zakończy się to w tym samym momencie, rozpoczynając etap analizy wyników. 
%TODO proszę to jaśniej opisać. Bo to jest tak, że "po kolei" dla każdej ze skal ? i z jakim zegarem ? To jest niejasne. %ODP
Jest on dość prosty, gdyż zakłada jedynie porównanie najlepszych rezultatów ze wszystkich skal -- w modułach klasyfikacji zaimplementowano logikę, która zapamiętuje swoje lokalne minima i parametry obszarów (w odpowiednich skalach), na podstawie których je osiągnięto. 
Skutkiem porównania jest wyłonienie skali z najlepszym wynikiem, a koordynaty tego obszaru zostają dostosowane do rozdzielczości $720 \times 1280$ i mogą być wykorzystane przez warstwę wyższą. %TODO natywnej (średnie słowo) %ODP OK

%TODO Ogólnie - nieco lepszy opis, szczególnie na ogólnym poziomie. I coś z tą normalizacją jest nie tak. %ODP normalizację wyjaśniłem wyżej, poprawiono


%TODO Kolejna sprawa to również jakiś schemat blokowy modułu sprzętowego.

%TODO Dalej. Zużycie zasobów przez moduł mean-shift i hog/svm - tabelki.%ODP OK

\section{Podsumowanie}
Proces implementacji sprzętowej szedł w parze z nadzorem wykorzystania zasobów w układzie. Nieodpowiednie zagospodarowanie dostępnymi w logice elementami mogłoby mieć wpływ na rozmieszczenie (routing) sygnałów i problemy związane z ich czasami ustalania lub podtrzymania (setup/hold time) . Tabele \ref{tab:utilizationMS}, \ref{tab:utilizationHOG} oraz \ref{tab:utilizationOverall} przedstawiają wykorzystanie zasobów w układzie Zynq 7Z020.
\begin{table}[h]
	\centering
	\captionsetup{justification=centering,margin=1cm}
	\begin{tabular}{|P{3cm} |P{3cm} |P{3cm}|P{4cm}|}	
		\hline
		\rowcolor{lightgray} Rodzaj zasobu & Wykorzystane & Dostępne & Wykorzystanie [\%]\\ 
		LUT		& 4898	& 53200 & 9.2\\ 
		\hline
		FF		& 7921	& 106400 & 7.4\\ 
		\hline
		BRAM	& 25.5	& 140 & 18.2\\ 
		\hline
		DSP		& 27	& 220 & 12.3\\ 
		\hline		
	\end{tabular}
	\caption{Wykorzystanie zasobów dla zaimplementowanego algorytmu MeanShift}
	\label{tab:utilizationMS}
\end{table}

\begin{table}[h]
	\centering
	\captionsetup{justification=centering,margin=1cm}
	\begin{tabular}{|P{3cm} |P{3cm} |P{3cm}|P{4cm}|}	
		\hline
		\rowcolor{lightgray} Rodzaj zasobu & Wykorzystane & Dostępne & Wykorzystanie [\%]\\ 
		LUT		& 23879	& 53200 & 44.8\\ 
		\hline
		FF		& 32668	& 106400 & 30.7\\ 
		\hline
		BRAM	& 66	& 140 & 47.1\\ 
		\hline
		DSP		& 55	& 220 & 25\\ 
		\hline		
	\end{tabular}
	\caption{Wykorzystanie zasobów dla zaimplementowanego algorytmu HoG+SVM}
	\label{tab:utilizationHOG}
\end{table}

\begin{table}[h]
	\centering
	\captionsetup{justification=centering,margin=1cm}
	\begin{tabular}{|P{3cm} |P{3cm} |P{3cm}|P{4cm}|}	
		\hline
		\rowcolor{lightgray} Rodzaj zasobu & Wykorzystane & Dostępne & Wykorzystanie [\%]\\ 
		LUT		& 33710	& 53200 & 63.3\\ 
		\hline
		FF		& 46386	& 106400 & 43.6\\ 
		\hline
		BRAM	& 92	& 140 & 65.7\\ 
		\hline
		DSP		& 83	& 220 & 37.7\\ 
		\hline		
	\end{tabular}
	\caption{Wykorzystanie zasobów - pełna architektura}
	\label{tab:utilizationOverall}
\end{table} 

Dodatkowo, postanowiono dokonać estymacji poboru mocy przez układ. W tym celu w środowisku Vivado wygenerowano raport w oparciu o określenie temperatury otoczenia równej $25^{\circ}$C i wybór najbardziej pesymistycznego poziomu estymacji. Z dokumentu wynika, że całkowity pobór mocy w układzie nie powinien przekraczać $3.5$W, z czego jednak aż $1.8$W to wartość przyporządkowana części PS. Fragment logiki realizujący metodę HoG+SVM potrzebuje $0.99$W, natomiast MeanShift zaledwie $0.1$W.
\chapter{Weryfikacja działania systemu wizyjnego}
%TODO bardziej inforamtywny tytuł. %ODP OK

Sprawdzenie poprawności działania obu algorytmów miało przebieg trzyetapowy:
\begin{itemize}
	\item model programowy,
	\item symulacja modelu behawioralnego,
	\item uruchomienie algorytmu z warstwą sterującą na dronie.
\end{itemize}
%TODO a model programowy ? %ODP OK

\section{Testy symulacyjne}

Skomplikowanie procesu implementacji sprzętowej wymagało równolegle przeprowadzanych testów symulacyjnych podczas prac nad każdym z algorytmów śledzących. %TODO usunac pierwszą część zdania, druga przeredagować. %ODP OK
Za referencję do symulacji obrano uprzednio stworzony model programowy w MATLABie, chociażby ze względu na wykorzystanie tych samych obrazów w celu porównania wyników. %TODO raczej za referencję. %ODP OK
Moduł symulacyjny stworzony w środowisku Vivado i języku SystemVerilog nie uwzględniał jedynie części logiki zawartej w \textit{Block Design} -- a więc instancji procesora oraz wejściowego i wyjściowego fragmentu toru wizyjnego. 
Testy procesora nie mają związku z testowanymi algorytmami, a mocno spowalniałyby pracę symulatora. %TODO inaczej to ująć. %ODP OK
Jako zamiennik wejścia informacji wizyjnej, wystarczył zasymulowany komplet sygnałów RGB z informacją o wartości piksela, która została wczytana z pliku tekstowego. 
Plik przechowujący jedną ramkę obrazu, wygenerowano w MATLABie umieszczając każdy kolejny piksel w nowej linii w formacie heksadecymalnym, po dwa znaki na każdą składową R, G i B.

Podstawowej zaletą symulacji jest możliwość podejrzenia propagowanych w układzie sygnałów w oknie wyświetlającym ich czasowy przebieg. %TODO zaletą + to przystosowane też inaczej. %ODP OK
Jednak ze względu na zwiększający się poziom skomplikowania projektu, z czasem postanowiono uprościć porównanie wybranych wyników symulacji z pracą modelu programowego. %TODO "rozrastający się". + niejasne %ODP OK
W kodzie architektury stworzono więc logikę zapisującą do plików określone informacje z pojedynczej iteracji algorytmu. 
By jednak zapis nie był realizowany na każdym zboczu narastającym zegara, potrzebne było określenie sygnałów wyzwalających - -zazwyczaj były to odpowiedniki sygnałów aktywnych powiązanych z określoną informacją. 
Do analizy stworzono w MATLABie dodatkowy skrypt, który parsował stworzone podczas symulacji pliki, uruchamiał pojedynczy przebieg modelu programowego i porównywał wyniki, określając liczbę błędów na danym etapie algorytmu dla całej ramki. 
Niektóre dane, z racji ograniczenia bitowej reprezentacji w architekturze, wymagały zdefiniowania akceptowalnego poziomu tolerancji błędu.


\section{Testy w układzie Zynq} %TODO możę konkretnie w Zynq %ODP OK

Symulacje są zbyt kosztownym czasowo narzędziem, dlatego w pewnym momencie należało przejść na testy na urządzeniu PYNQ. 
Proces budowy projektu sprowadza się do stworzenia konfiguracji sprzętowej w Vivado i skompilowania aplikacji uruchamianej na procesorze ARM. %TODO budowy do zbudowania %ODP OK, masakra.
Docelowo układ SoC może być uruchomiony z poziomu karty SD, jednak ze względu na kwestię praktyczności, pozostano przy połączeniu JTAG. 
Stworzona konfiguracja testowa jest przedstawiona na schemacie \ref{fig:testing_setup}. Zastosowaniem komputera (PC \#1) jest nie tylko zaprogramowanie układu Zynq poprzez interfejs JTAG, ale i diagnostyczna komunikacja z układem, realizowana poprzez UART. %TODO \ref + omówienie schematu %ODP OK
Źródłem obrazu wideo może być dowolne urządzenie z możliwością wysłania sygnału \textit{720p} poprzez kabel HDMI. W tym wypadku jest to kamera, albo inny komputer - służący to odtwarzania wcześniej zapisanych materiałów wideo. Najważniejszym elementem podczas testów jest wyświetlenie obrazu wyjściowego, z nakreślonymi obszarami detekcji.
\begin{figure}[h]
	\centering
	\includegraphics[width=14cm]{6_testing_setup.jpg}
	\caption{Schemat stanowiska testowego}
	\label{fig:testing_setup}
\end{figure}

Test polega na skanowaniu ruchomego obrazu w poszukiwaniu postaci, a następnie śledzeniu znalezionej osoby. Analiza jest rozpoczynana w lewym górnym rogu, po czym sukcesywnie przechodzi w linii poziomej do prawej krawędzi, co jest następnie powtarzane dla niższych pozycji \ref{fig:scan_scheme} %TODO 'ref %ODP OK
Na przykładowych zrzutach z materiału wideo \ref{fig:scan_screenshot} zaprezentowano proces skanowania obrazu. 
Niebieskimi prostokątami oznaczone są aktualne okna detekcji dla poszczególnych skal. 
Zielone okno to obszar śledzenia algorytmem MeanShift, natomiast czerwonym kolorem oznaczono aktualnie najlepsze okno $128 \times 64$ (przeskalowane ponownie do oryginalnej rozdzielczości).

\begin{figure}[h]
	\centering
	\includegraphics[width=10cm]{6_scan_1.jpg}
	\caption{Proces skanowania} %TODO Ilustracja procesu skanowania + pogrubić ramki w paint :) %ODP OK, prosty schemat wyżej
	\label{fig:scan_screenshot}
\end{figure}

\begin{figure}[h]
	\centering
	\includegraphics[width=10cm]{6_track_1.jpg}
	\caption{Śledzenie \#1}
	\label{fig:track_1}
\end{figure}

\begin{figure}[h]
	\centering
	\includegraphics[width=10cm]{6_track_2.jpg}
	\caption{Śledzenie \#2}
	\label{fig:track_1}
\end{figure}

\begin{figure}[h]
	\centering
	\includegraphics[width=10cm]{6_track_3.jpg}
	\caption{Śledzenie \#3}
	\label{fig:track_1}
\end{figure}

 Przebieg testu powinien być jak najbardziej zbliżony do pracy układu podczas lotu, z tego względu do procedury dodano komunikację UART z autopilotem. Nadzór nad nią jest sprawowany poprzez dodatkowe połączenie szeregowe łączące układ Zynq z komputerem, który wyświetla w oknie terminala wszystkie istotne komunikaty. Przykładowo, raport \ref{tab:log} przedstawia informacje uzyskane w ciągu pierwszych kilkunastu sekund jednego z testów. Po otrzymaniu sygnału startu (przełącznik na układzie PYNQ lub na aparaturze radiowej) do autopilota wysyłana jest komenda uzbrojenia (ARM) i startu (TAKEOFF), jednak ze względu na obecność drona w budynku niemożliwe jest określenie pozycji poprzez GPS - odebrana z autopilota wiadomość COMMAND ACK informuje, że wykonanie komendy TAKEOFF (ID: 22) się nie powiodło (status: 4). Podczas testu naziemnego ten komunikat jest jednak ignorowany, i rozpoczynany jest proces skanowania. Po dłuższej chwili układ wysyła informację o pozytywnym wyniku detekcji, po czym system rozpoczyna zadanie śledzenia. Od tego momentu większość komunikatów dotyczy informacji z PL o odległości śledzonego obiektu od wartości zadanej w pionie i poziomie. Trzecią wartością jest uśredniony wynik skali z 4 ostatnich detekcji -- wartość przemnożona przez 10 z powodu braku funkcji wyświetlającej liczby zmiennoprzecinkowe. Niezależnie od przebiegu pracy systemu wizyjnego mogą pojawiać się wiadomości z autopilota o statusie (INFO). Test jest przerywany przez użytkownika w momencie zmiany pozycji przełącznika sygnału startu - oprócz zakończenia pracy algorytmu wysyłana jest wówczas komenda lądowania (LAND).
 
%TODO \ref do tabeli. %ODP OK
%TODO jakieś jednak omówienie. %ODP OK, dodano wyżej


\begin{table}[h]
	\centering\scriptsize 
	
	\begin{tabular}{|p{8cm} |}
		
        \hline
prompt>>ARMING... \\
ARMED! \\
TAKING OFF... \\
Sent TAKEOFF CMD. \\
COMMAND ACK: 22 4 \\
STARTING SEARCH! \\
xDiff: 37        \tab yDiff: 33      \tab| scale: 27 || \\
DATA OK, STARTING FOLLOW! \\
xDiff: 20        \tab yDiff: 20      \tab| scale: 25 || \\
xDiff: 10        \tab yDiff: 20      \tab| scale: 25 || \\
xDiff: 52        \tab yDiff: 38      \tab| scale: 22 || \\
xDiff: 94        \tab yDiff: 8       \tab\tab| scale: 22 || \\
xDiff: 140       \tab yDiff: 12      \tab| scale: 20 || \\
xDiff: 165       \tab yDiff: 30      \tab| scale: 25 || \\
xDiff: 157       \tab yDiff: 22      \tab| scale: 22 || \\
xDiff: 125       \tab yDiff: 40      \tab| scale: 25 || \\
xDiff: 95        \tab yDiff: 35      \tab| scale: 22 || \\
xDiff: 62        \tab yDiff: 26      \tab| scale: 20 || \\
xDiff: 26        \tab yDiff: 28      \tab| scale: 20 || \\
xDiff: 10        \tab yDiff: 28      \tab| scale: 20 || \\
xDiff: -6         \tab\tab yDiff: 28      \tab| scale: 20 || \\
xDiff: -30        \tab yDiff: 36      \tab| scale: 20 || \\
xDiff: -70        \tab yDiff: 28      \tab| scale: 20 || \\
xDiff: -86        \tab yDiff: 20      \tab| scale: 20 || \\
xDiff: -118       \tab yDiff: 28      \tab| scale: 20 || \\
xDiff: -150       \tab yDiff: 12      \tab| scale: 20 || \\
xDiff: -150       \tab yDiff: 12      \tab| scale: 20 || \\
xDiff: -110       \tab yDiff: 20      \tab| scale: 20 || \\
xDiff: -86        \tab yDiff: 20      \tab| scale: 20 || \\
xDiff: -62        \tab yDiff: 28      \tab| scale: 20 || \\
INFO: PreArm: Throttle below Failsafe \\
xDiff: 22        \tab yDiff: 28      \tab| scale: 20 || \\
xDiff: 30        \tab yDiff: 4       \tab\tab| scale: 22 || \\
xDiff: 62        \tab yDiff: 42      \tab| scale: 27 || \\
xDiff: 95        \tab yDiff: 33      \tab| scale: 27 || \\
xDiff: 107       \tab yDiff: 33      \tab| scale: 30 || \\
xDiff: 107       \tab yDiff: 33      \tab| scale: 30 || \\
xDiff: 107       \tab yDiff: 33      \tab| scale: 30 || \\
xDiff: 172       \tab yDiff: 34      \tab| scale: 20 || \\
xDiff: 268       \tab yDiff: 14      \tab| scale: 20 || \\
xDiff: 368       \tab yDiff: 14      \tab| scale: 22 || \\
xDiff: 450       \tab yDiff: 20      \tab| scale: 25 || \\
xDiff: 560       \tab yDiff: 15      \tab| scale: 22 || \\
Sent LAND CMD. \\
COMMAND ACK: 21 0 \\    
\hline 	
	\end{tabular}
	\caption{Przykładowa zawartość terminala ze śledzenia na podstawie gotowego materiału wideo}
	\label{tab:log}
\end{table}
\chapter{Podsumowanie i możliwości rozwoju pracy}

W zrealizowanym projekcie przedstawiono sprzętową realizację detekcji i śledzenia osoby w heterogenicznym układzie Zynq SoC, na potrzeby kontroli bezzałogowego statku powietrznego. 
Osiągnięto przetwarzanie obrazu o rozdzielczości $1280\times 720$ dla 60 klatek na sekundę, z prędkością \textit{60Hz} i \textit{30Hz} odpowiednio dla algorytmów MeanShift oraz HoG+SVM.

Na uwagę zasługuje warstwa najwyższa, łącząca pracę obu algorytmów i komunikująca się z autopilotem Pixhawk. 
Wykorzystanie protokołu MAVLink pozwala stworzyć platformę, która jest w stanie wydawać polecenia ruchu bez udziału pilota.

%TODO jeszcze o budowie. Ogólnie nieco obszerniej by to można opisać.

Autor uważa jednak, że bazując na testowanej konfiguracji sprzętowo-programowej można dokonać szeregu usprawnień, podnoszących ogólną niezawodność rozwiązania. %TODO to jednak tu nie pasuje
Jednym z nich jest próba zredukowania liczby fałszywych detekcji (HoG+SVM) poprzez zmianę wielkości bloków lub komórek. %TODO no właśnie, jakoś Pan to testował ? Bo chyba tego tam nie ma.
Innym pomysłem mogłoby być proste zwiększenie liczby analizowanych skal. %TODO liczby ? %ODP OK
Kolejnym usprawnieniem, tym razem dla algorytmu MeanShift, byłoby zlikwidowanie rzadkich sytuacji utraty zbieżności ze śledzonym obszarem -- co skutkuje „wędrowaniem okna”. %TODO wie Pan dlaczego ?
Barierą na drodze większości zmian jest liczba dostępnych zasobów w układzie -- wymagana byłaby zmiana układu na dysponujący zwłaszcza większą liczbą bloków BRAM. 
Istnieją jednak zmiany niewymagające ingerencji w kod. 
Do podstawowych należałby lepszy dobór ustawień programowych kamery w celu poprawy rejestrowanych (wysyłanych kablem) kolorów. Kamery sportowe są wyposażone w wiele usprawnień (tryb nocny, rektyfikacja,itp.), które niekoniecznie muszą poprawiać działanie systemu wizyjnego %TODO to jest niejasne, a ważne %ODP OK
Kolejną mogłaby być lepsza stabilizacja obrazu - ze względu na podpięty do kamery kabel zmienia się jej środek ciężkości, co zaburza pracę gimbala - silnik na jednej z osi obrotu musiał nawet być z tego powodu wyłączony. %TODO szczegóły %ODP OK
Ostatnią, mającą największy wpływ na śledzenie, byłoby poprawienie właściwości lotnych drona. Mowa tu głównie o problemach z płynnością ruchu oraz utrzymywaniem drona w zadanej pozycji - co może mieć związek z ustawieniami akceleratorów lub przetwarzaniem sygnału GPS.%TODO j.w. %ODP OK

Ponadto, zbudowana platforma stanowi ogromny potencjał dla nowych pomysłów związanych z autonomizacją dronów. 
Podstawowym kierunkiem mogłaby być zdolność omijania przeszkód (nadal kosztowna opcja w dronach komercyjnych). Detekcja mogłaby być realizowana przez specjalistyczne czujniki odległościowe, stereowizję lub nawet lidar. %TODO odległościowe, sterowizja, lidar %ODP OK




\appendix
\input{praca_pliki/source_dvd_content}
\renewcommand{\thechapter}{\Alph{chapter}}
\chapter{Opis techniczny platformy}
\label{cha:opis_techniczny}

\section{Oprogramowanie ArduPilot}
Oprogramowanie ArduPilot jest jednym z najbardziej popularnych systemów instalowanych na kontrolerach lotu. Jest to projekt open source, którego największą zaletą jest bogata możliwość konfiguracji. W zależności od modelu, którym użytkownik zamierza sterować, istnieją 3 wersje oprogramowania: 
\begin{itemize}
	\item ArduPlane - zarządzający samolotami i tzw. platformami FPV
	\item ArduCopter - obsługujący platformy multirotorowe i helikoptery jednowirnikowe
	\item ArduRover - przeznaczony do obsługi pojazdów naziemnych i nawodnych
\end{itemize}
Oprogramowanie można instalować na wspierających je kontrolerach, m.in urządzeniach serii APM lub Pixhawk. Proces taki przebiega z poziomu specjalnej aplikacji na PC służącej głównie do zdalnego zarządzania ustawieniami autopilota: APM Planner lub MISSION Planner.

\subsection{Tryby lotu autopilota}
\label{flightmodes}
Oprogramowanie ArduCopter wyróżnia aż 14 trybów lotu, opisanych w dokumentacji \cite{FlightModes}. Najważniejsze z nich to:
\begin{itemize}
	\item \textit{Stabilize} -- podstawowy tryb umożliwiający manualne sterowanie dronem; autopilot stabilizuje osie obrotu Roll oraz Pitch. Wymaga nieustannego korygowania wychyleń drążków aparatury radiowej do utrzymania zadanej wysokości i pozycji.
	\item \textit{Alt Hold} -- wariant trybu \textit{Stabilize} z automatycznym utrzymaniem wysokości dla sygnału Throttle o wartości zakresu $40-50\%$. Dla wychyleń drążka Throttle spoza tego zakresu dron wykona odpowiedni ruch pionowy, z maksymalną prędkością wznoszenia/opadania konfigurowaną przez parametr PILOT\_VELZ\_MAX (domyślnie $2.5$m/s).
	\item \textit{Loiter} -- dalsze rozszerzenie trybu \textit{Stabilize}, pozwalające utrzymać zadaną pozycję w przestrzeni. Użytkownik może dowolnie sterować dronem, jednak w momenciu uwolnienia drążków platforma po krótkiej chwili zatrzyma się.
	\item \textit{RTL - Return-to-Launch} -- po zmianie trybu na RTL dron wzniesie się na wysokość zdefiniowaną przez parametr RTL\_ALT -- domyślnie $15$m (jeśli znajduje się wyżej, pomija ten proces). Następnie porusza się w linii prostej do lokalizacji, w której został uzbrojony - i opada na wysokość zdefiniowaną parametrem RTL\_ALT\_FINAL.
	\item \textit{Auto} -- tryb realizujący predefiniowaną misję, która jest zapisywana w autopilocie z poziomu oprogramowania MISSION Planner.
	\item \textit{Guided} -- tryb pozwalający akceptować komendy ruchu  wydawane dynamicznie przez niezależne urządzenie, pełniące rolę tzw. stacji naziemnej. Komunikacja jest realizowana zazwyczaj poprzez bezprzewodowe łącze telemetryczne korzystające z portu szeregowego autopilota, jednak możliwa jest też transmisja przewodowa, z urządzenia umieszczonego na platformie. Protokołem wymiany danych jest MAVLink.
	\item \textit{Land} - tryb rozpoczynający procedurę lądowania. Na wysokości większej niż $10$m opadanie następuje z prędkością zdefiniowaną przez parametr WPNAV\_SPEED\_DN (domyślnie $150$cm/s), dla niższych wysokości jest to prędkość związana z parametrem LAND\_SPEED (domyślnie $50$cm/s). Autopilot wykrywa kontakt z ziemią, jeśli różnica w zmierzonej przez barometr wysokości jest mniejsza niż $20$cm przez minimum jedną sekundę.
\end{itemize}

Aktualny tryb autopilota można zmienić, wysyłając odpowiednią komendę poprzez protokół MAVLink - jednak podstawowym sposobem jest wykorzystanie aparatury radiowej. ArduPilot rezerwuje bowiem 5 kanał transmisji radiowej do ustawienia jednego z 6 dostępnych trybów. Każdy z nich ma odgórnie przyporządkowany zakres długości pulsu PWM, dla którego może być aktywowany. Aplikacja MISSION Planner umożliwia konfigurację tej funkcjonalności z perspektywy autopilota - rysunek \ref{fig:flight_modes} przedstawia tryby lotu zapisane w urządzeniu na potrzeby projektu.

\begin{figure}[ht]
	\centering
	\includegraphics[width=14cm]{B_flight_modes.png}
	\caption{Tryby lotu w oknie konfiguracyjnym aplikacji MISSION Planner}
	\label{fig:flight_modes}
\end{figure}

Opis konfiguracji kanału 5 przedstawiony jest w podrozdziale poświęconym aparaturze radiowej.
\subsection{Protokół MAVLink}
\label{subsec:MAVLink}
Oprogramowanie ArduPilot obsługuje komunikację z urządzeniami pełniącymi rolę stacji naziemnych. Może to być komputer, tablet, ale również obecna na platformie UAV inna jednostka. Niezależnie od wyboru, komunikacja z autopilotem bazuje na wykorzystaniu jednego z jego portów szeregowych \cite{PixhawkSerial}.

Warstwą transportową takiej komunikacji jest protokół MAVLink, opracowany w 2009 roku na potrzeby małych pojazdów bezzałogowych. 
Dość duży podzbiór jego wiadomości i komend jest wspierany przez oprogramowanie ArduCopter (pełna lista jest dostępna pod adresem: \cite{ArduCopterCmds}, dość poręczna jest również strona \cite{MAVLinkMSG}). 
Ramka protokołu ma zmienną długość i jest opisana w tabeli \ref{tab:MAVlinkframe}.

\newcolumntype{P}[1]{>{\centering\arraybackslash}p{#1}}
\begin{table}[h]
	\centering
	\caption{Ramka protokołu MAVLink}
	
	\begin{tabular}{|P{1.5cm} |P{4cm}| p{9cm}|}
		
		\hline
		\rowcolor{lightgray} Bajt \# & Oznaczenie & Uwagi \\ 
		0	& Początek ramki &	Zawsze o wartości 254 \\ \hline
		1	& Długość danych & Wartość \textit{n} (w bajtach )	\\ \hline		
		2	& Sekwencja pakietu &	Wartość inkrementowana z każdą kolejną transmisją\\ \hline
		3	& ID systemu &	Dla komputera pokładowego (SoC) równa 255 \\ \hline
		4	& ID komponentu &	Dla komputera pokładowego (SoC) równa 190 \\ \hline
		5	& ID wiadomości &	\\ \hline
		6:n+6-1	& Dane & Struktura zależna od rodzaju wiadomości	\\ \hline
		n+6:n+7	& CRC &	Suma kontrolna całego pakietu bez bajtu \#0\\ \hline
	\end{tabular}
	\label{tab:MAVlinkframe}
\end{table}

%TODO std. przyjmuje się, że podpisa tabeli powinien być nad. Tu przeniosłem, w pozostałych proszę sprawdzić. %ODP OK

Protokół został zaprojektowany w postaci plików nagłówkowych -- taka forma umożliwia łatwe wykorzystanie w aplikacji tworzonej w języku C/C++.
\subsubsection{Implementacja komend MAVLink}
\label{commands}
Dokumentacja protokołu MAVLink jest dość obszerna, jednak doprowadzenie do pełnego zrozumienia działania poszczególnych komend i ich wpływu na zachowanie drona wymaga podjęcia testów praktycznych.
Każda z komend autopilota ma swój własny plik nagłówkowy, którego najważniejszą częścią jest definicja struktury z danymi, funkcji pakującej (do wysłania wiadomości) oraz dekodującej (do odebrania wiadomości). Mają one dość generyczne nazwy, przykładowo dla wiadomości HEARTBEAT są to odpowiednio:
\begin{itemize}
	\item \textit{mavlink\_heartbeat\_t}
	\item \textit{mavlink\_msg\_heartbeat\_pack(...)}
	\item \textit{mavlink\_msg\_heartbeat\_decode(...)}
\end{itemize}
Używanie tych funkcji ma dodatkową zaletę - uwzględniają liczenie sumy kontrolnej oraz sekwencji pakietu, a więc elementów, których błędne wartości powodowałaby nieprawidłowości w transmisji.
Do komunikacji pomiędzy urządzeniami wykorzystywany jest port szeregowy, zatem najmniejszą jednostką wymiany informacji jest bajt. Dla procesu odbierania wiadomości należy ten zestaw bajtów odpowiednio zinterpretować -- do formowania ich w ramki służy funkcja \textit{mavlink\_frame\_char(...)}, która powinna być wywoływana w trakcie odebrania każdego bajtu danych. Zwracany przez nią status określa, czy ramka jest już gotowa do dalszej analizy. Pełna wiadomość zostanie przepisana do struktury typu \textit{mavlink\_message\_t}. Jedno z z pól gotowej struktury, \textit{msgid}, określa typ wiadomości. Pozwala to zdefiniować użycie odpowiedniej funkcji dekodującej w zależności od ID - gotowe informacje będą się znajdować w odpowiedniej strukturze danych.\newline
Przykładowo, ID o numerze 253 oznacza wiadomość typu STATUSTEXT (informacja z autopilota w formacie tekstowym). W celu jej zdekodowania należy wywołać funkcję 
\textit{mavlink\_msg\_statustext\_decode(...)}, która zwróci strukturę \textit{mavlink\_statustext\_t} ze znakami gotowymi do wyświetlenia lub dalszej analizy.

Wysyłanie komend jest o wiele prostsze, wymagane jest bowiem spakowanie informacji w tablicę bajtów i wysyłanie ich w odpowiedniej kolejności.

Najbardziej skomplikowaną w użyciu jest komenda ruchu, SET\_POSITION\_TARGET\_LOCAL\_NED. Funkcja, która jest odpowiedzialna za konwersję jej w tablicę bajtów nosi nazwę \textit{mavlink\_msg\_set\_position\_target\_local\_ned\_pack}. Używa ona następujących parametrów
\begin{itemize}
	\item \textit{x, y, z} -- przesunięcie w odpowiednim kierunku, w metrach
	\item \textit{vx, vy, vz} -- prędkość w odpowiednim kierunku, w $m/s$
	\item \textit{afx, afy, afz} -- przyśpieszenie w odpowiednim kierunku, w $m/s^2$ lub $N$
	\item \textit{yaw} -- obrót wokół pionowej osi, w radianach
	\item \textit{yaw\_rate} -- prędkość kątowa obrotu, w rad/s
	\item \textit{type\_mask} -- maska bitowa określająca wartości, które mają być zignorowane (wartość 1) przez autopilot. Mapowanie bitów 1: x, bit 2: y, bit 3: z, bit 4: vx, bit 5: vy, bit 6: vz, bit 7: ax, bit 8: ay, bit 9: az, bit 11: yaw, bit 12: yaw\_rate. Bit 10 jest używany do określenia jednostki przyśpieszenia
	\item \textit{coordinate\_frame} -- zdefiniowanie układu współrzędnych, w odniesieniu do którego odbywać się będzie ruch
\end{itemize}
Warto zaznaczyć, że do wykonania ruchu potrzebne jest odblokowanie wszystkich bitów związanych z przemieszczeniem LUB prędkością, kombinacja obu sposobów ruchu nie jest wspierana przez oprogramowanie ArduPilot. Parametry związane z przyśpieszeniem oraz obrotem w osi pionowej również nie są wspierane.
Do przechowywania wartości wygodnie jest używać struktury \textit{mavlink\_set\_position\_target\_local\_ned\_t}.

\section{Konfiguracja aparatury radiowej}

Aparatura radiowa FrSky Taranis X9D Plus pozwala przesyłać informacje zgrupowane w 16 kanałów. Do manualnej kontroli drona wymagane są jedynie 4 z nich, przyporządkowane dwóm głównym drążkom aparatury. Najważniejszym sygnałem jest Throttle, który odpowiada za ogólny poziom mocy wszystkich silników drona. Kolejnymi sygnałami są Roll, Pitch oraz Yaw, które są przyporządkowane osiom obrotu platformy \ref{fig:flight_modes}.

\begin{figure}[ht]
	\centering
	\includegraphics[width=12cm]{B_rotational_axes.png}
	\caption{Osi obrotu drona \cite{Bouhali2017}}
	\label{fig:flight_modes}
\end{figure}

Kanał 5 jest używany przez autopilota do zmiany trybu lotu. Funkcja ta obsługuje wybór spośród maksymalnie 6 trybów, którym przyporządkowano zakresy szerokości pulsu PWM przedstawione w oknie konfiguracyjnym \ref{fig:flight_modes}. Elementem aparatury radiowej przypisanym do kanału musiał zostać zestaw kilku przełączników - zastosowanie potencjometrów nie byłoby ergonomicznym rozwiązaniem. Wybrano dwa przełączniki 3-pozycyjne oznaczone jako SA oraz SB, które mikser aparatury dodaje do siebie - przedstawia to rysunek \ref{fig:mixer}. Zastosowane wagi: $-30$ dopasowano, by położenie dolne obu przełączników odpowiadało trybowi 1, a zmiana stanu któregokolwiek z nich skutkowała przejściem do innego zakresu szerokości pulsu PWM. Ostatecznie zestaw obu przełączników daje możliwość wyboru spośród 5 trybów lotu.
\begin{figure}[ht]
	\centering
	\includegraphics[width=10cm]{B_mixer.png}
	\caption{Mikser sygnałów w aparaturze radiowej}
	\label{fig:mixer}
\end{figure}

Kanał 6 jest przypisany do potencjometru S1 i został skonfigurowany jako sygnał ciągły sterujący wartością zadaną nachylenia kamery. W trakcie lotu  mechanizmy gimbala utrzymują tę wartość pomimo możliwych wychyleń platformy. Orientacja kamery z minimalną wartością nachylenia pozwala rejestrować obraz bezpośrednio przed dronem, z kolei maksymalna wartość nachylenia kieruje obiektyw ku ziemi.\newline
Ważne: ze względu na podłączony do kamery przewód HDMI regulacja nachylenia nie działa poprawnie -- w projekcie zasilanie silnika odpowiedzialnego za nachylenie kamery nie jest podłączone.
%może jakaś kalibracja żyroskopów gimbala jest w stanie pomóc, na pewno przydałby się bardziej giętki kabel

Kanał 7 przypisano do 3-pozycyjnego przełącznika SC. Sygnał PWM z informacją z tego kanału jest wysyłany przez autopilot do układu rekonfigurowalnego i służy do zdalnej inicjalizacji misji śledzenia.


\section{Instrukcja powtórzenia eksperymentu śledzenia}

\begin{itemize}
	\item Upewnić się, że wszystkie przewody zostały poprawnie podłączone do płyty PYNQ (zasilanie 5V, sygnały z autopilota), a sama płyta jest dobrze unieruchomiona na platformie
	\item Podłączyć pakiet LiPo
	\item Włączyć zasilanie płyty PYNQ
	\begin{itemize}
		\item jeśli układ Zynq nie jest konfigurowany z karty SD, należy skonfigurować go poprzez port JTAG
	\end{itemize}	
	\item Włączyć kamerę i upewnić się, że jest w trybie rejestracji wideo
	\item Podłączyć autopilot do komputera (kabel USB) i w aplikacji MISSION Planner sprawdzić podstawowe informacje -- obecność sygnału GPS i innych czujników, zwracane błędy. Zamknąć połączenie i odłączyć kabel
	\item Upewnić się, że nie ma przewodów lub elementów mogących stanowić zagrożenie przy pracy śmigieł
	\item Włączyć aparaturę radiową i za pomocą przełączników SA oraz SB wybrać tryb GUIDED
	\item Ustawić przełącznik SC w pozycji górnej -- wysłać sygnał rozpoczynający misję 
	\item Oczekiwać na osiągnięcie stałej wysokości przez drona przy zachowaniu pozycji stojącej i odległości kilku metrów od drona
	\item Po ruchu drona wskazującym na rozpoczęcie śledzenia, można zmienić swoje położenie i obserwować działanie systemu 
	\item Wyłączyć misję ustawieniem przełącznika SC w pozycji środkowej lub dolnej -- rozpocznie się lądowanie 
\end{itemize}


\printbibliography

\end{document}
