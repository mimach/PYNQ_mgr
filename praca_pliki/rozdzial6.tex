\chapter{Testy}

Sprawdzenie poprawności działania obu algorytmów miało przebieg dwustopniowy:
\begin{itemize}
	\item symulacja modelu behawioralnego,
	\item uruchomienie algorytmu z warstwą sterującą na dronie.
\end{itemize}

\section{Testy symulacyjne}
O ile drugi ze sposobów był traktowany jako etap finalny, zwieńczający pracę nad projektem, to testy symulacyjne przebiegały równolegle z postępem prac nad poszczególnym z algorytmów śledzących. Co więcej, za wzór symulacji obrano uprzednio stworzony model programowy w MATLABie, chociażby ze względu na wykorzystanie tych samych obrazów w celu porównania wyników.
Moduł symulacyjny stworzony w środowisku Vivado i języku SystemVerilog nie uwzględniał jedynie części logiki zawartej w Block Designie - a więc instancji procesora oraz wejściowego i wyjściowego fragmentu toru wizyjnego. Byłaby to jednak część niepotrzebna, znacznie obciążająca i spowalniająca pracę symulatora. Jako zamiennik, wystarczył zasymulowany komplet sygnałów RGB z informacją o wartości piksela, która została wczytana z pliku tekstowego. Plik przechowujący jedną ramkę obrazu, wygenerowano w MATLABie umieszczając każdy kolejny pixel w nowej linii w formacie heksadecymalnym, po dwa znaki na każdą składową R, G i B.

Podstawowej funkcją symulacji jest możliwość podejrzenia propagowanych w układzie sygnałów w przystosowanym do tego oknie. Jednak ze względu na rozrastający się poziom skomplikowania modułów z czasem postanowiono bezpośrednio porównać wybrane funkcjonalności z pracą modelu programowego. W kodzie architektury stworzono logikę zapisującą do plików określone informacje z pojedynczej iteracji algorytmu. By jednak zapis nie był realizowany na każdym zboczu narastającym zegara, potrzebne było określenie sygnałów wyzwalających - zazwyczaj były to odpowiedniki sygnałów aktywnych. Do analizy stworzono w MATLABie dodatkowy skrypt, który parsował stworzone podczas symulacji pliki, uruchamiał pojedynczy przebieg modelu programowego i porównywał wyniki, określając ilość błędów na danym etapie algorytmu dla całej ramki. Niektóre dane, z racji ograniczenia bitowej reprezentacji w architekturze, wymagały zdefiniowania akceptowalnego poziomu tolerancji błędu.

TBD: Grafika z opisem

\section{Testy na dronie}

TBD