\chapter{Podsumowanie i możliwości rozwoju pracy}

W zrealizowanym projekcie przedstawiono sprzętową realizację detekcji i śledzenia osoby w heterogenicznym układzie ZYNQ SoC, na potrzeby kontroli bezzałogowego statku powietrznego. Osiągnięto przetwarzanie obrazu o rozdzielczości $1280\times 720$ dla 60 klatek na sekundę, z prędkością \textit{60Hz} i \textit{30Hz} odpowiednio dla algorytmów MeanShift oraz HoG+SVM.

Na uwagę zasługuje warstwa najwyższa, łącząca pracę obu algorytmów i komunikująca się z autopilotem Pixhawk. Wykorzystanie protokołu MAVLink pozwala stworzyć platformę, która jest w stanie wydawać polecenia ruchu bez udziału pilota.

Autor uważa jednak, że bazując na testowanej konfiguracji sprzętowo-programowej można dokonać szeregu usprawnień, podnoszących ogólną niezawodność rozwiązania. Jednym z nich jest próba zredukowania fałszywych detekcji (HoG+SVM) poprzez zmianę wielkości bloków lub komórek. Innym pomysłem mogłoby być proste zwiększenie analizowanych skal. Kolejnym usprawnieniem, tym razem dla algorytmu MeanShift, byłoby zlikwidowanie rzadkich sytuacji utraty zbieżności ze śledzonym obszarem - co skutkuje „wędrowaniem okna”. Barierą na drodze większości zmian jest liczba dostępnych zasobów w układzie - wymagana byłaby zmiana układu na dysponujący zwłaszcza większą ilością bloków BRAM. Istnieją jednak zmiany niewymagające ingerencji w kod. Do podstawowych należałoby lepsze skonfigurowanie kamery w celu poprawy rejestrowanych (wysyłanych kablem) kolorów. Kolejną mogłaby być lepsza stabilizacja obrazu. Ostatnią, mającą największy wpływ na śledzenie, byłoby poprawienie właściwości lotnych drona.

Ponadto, zbudowana platforma stanowi ogromny potencjał dla nowych pomysłów związanych z autonomizacją dronów. Podstawowym kierunkiem mogłaby być zdolność omijania przeszkód (nadal kosztowna opcja w dronach komercyjnych). Detekcja mogłaby być realizowana przez specjalistyczne czujniki, lub nawet LIDAR.\newline 



