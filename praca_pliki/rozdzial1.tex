\chapter{Wstęp}
\label{cha:wstep}

Bezzałogowe statki powietrzne, potocznie nazywane dronami, zyskują w ostatnich latach coraz większe zainteresowanie. Według raportu portalu Business Insider z czerwca 2016 roku \cite{BInsider}, światowy rynek bezzałogowych statków powietrznych powiększy swoją wartość z niecałych 6 mld \$ w 2013 roku do prawie 13 mld \$ w roku 2024. Biorąc pod uwagę podstawowy podział maszyn - ze względu na przeznaczenie - raport ten wskazuje na rosnący trend zarówno w przypadku przemysłu obronnego, jak i cywilnego. Tak gwałtowny wzrost popularności niesie ze sobą potrzebę tworzenia nowych rozwiązań, którym sprzyja ogólny rozwój technologiczny i dostępność wielu peryferiów. 

Jedno z głównych zadań, jakie stoi przed inżynierami, to autonomizacja takich maszyn, czyli nadanie im możliwości pracy bez nadzoru człowieka. Oznacza to wyposażenie ich w odpowiednie urządzenia elektroniczne i rozwiązania służące do realizacji określonego sterowania w oparciu o zebrane informacje.

\section{Cel pracy}

Celem pracy jest implementacja systemu służącego do rozpoznania i śledzenia osoby za pomocą bezzałogowego pojazdu latającego.  Projekt zakłada stworzenie odpowiednich algorytmów na układzie reprogramowalnym oraz nawiązanie komunikacji z systemem dostępnym na dronie, pozwalając w sposób autonomiczny wydawać komendy sterujące pracą maszyny.



\section{Struktura pracy}

Rozdział drugi poświęcono opisowi dostępnych algorytmów, a następnie skontentrowano się na dokładnym przedstawieniu tych, które zostały wybrane do implementacji. W rozdziale trzecim pokrótce przedstawiono stworzenie modeli programowych, które następnie przeniesiono na system wbudowany - co zostało opisane w rozdziale 4. Część 5 zawiera informacje o konfiguracji sprzętowej oraz o wyższej warstwie decyzyjnej. Część 6 streszcza proces testów poświęconych tworzonemu rozwiązaniu. Ostatni rozdział stanowi przedstawienie wniosków i wskazuje możliwe kierunki rozwoju pracy.












